\subsection{Data Preparation} \label{data_preparation}

All of our data records are resampled and quantized before further analysis. In general, dimensionality and cardinality reduction of time series is performed to ease and accelerate data processing by providing a more compact representation of equidistant measurements \cite{liu2009uwave}.

\paragraph{Resampling:} The recorded acceleration data was resampled by means of the moving average technique, using a window size of 50 ms and step size of 30 ms.

\paragraph{Quantization:} The resampled records were then converted into time series with integer values between -16 and 16, such as suggested in related work \cite{liu2009uwave} and summarized in table \ref{table:conversion}.

\begin{table}
    \begin{center}
        \begin{tabularx}{\textwidth}{XX}
            \hline
            \textbf{Acceleration data ($a$) in $\frac{dm}{s^2}$} & \textbf{Converted value}\\
            \hline
            $a > 200$ & 16\\
            $100 < a < 200$ & 11 to 15 (five levels linearly)\\
            $0 < a < 100$ & 1 to 10 (ten levels linearly)\\
            $a = 0$ & 0\\
            $-100 < a < 0$ & -1 to - 10 (ten levels linearly)\\
            $-200 < a < -100$ & -11 to - 15 (five levels linearly)\\
            $a < -200$ & -16\\
            \hline
        \end{tabularx}
    \end{center}
    \caption{Table shows the conversion rules of the recorded acceleration data. In contrast to \cite{liu2009uwave} are
    $100\frac{dm}{s^2}$ the steps threshold and not $1g$.}
	\label{table:conversion}
\end{table}
