\subsection{Results} \label{results}

28152 different configured simulations are obtained by the presented options in section \ref{experiment}. The best
performing configurations are reaching a $F_{1}score_{\mu}$ around 0.7384. Two configurations achieve the best
$F_{1}score_{\mu}$. One simulation works with and one without the presented filter approach. Both configurations use
the same options apart from the filter. Those dominating configuration options are the following.
\begin{itemize}
    \item \textbf{$\eta$DTW}, Dynamic Time Warping combined with the $\eta$ normalization and a Sakoe-Chiba band size of
        36 \% as time series distance measure.
    \item \textbf{Mid} as window size determination.
    \item \textbf{HAveD} as threshold determination.
\end{itemize}
Hereinafter, is inspected the influence of those options by changing step by step only one part of the
dominating configuration. Afterwards, the influence of the time series filter measure on the $F_{1}score_{\mu}$ and on
the amount of nearest neighbour classification calls is examined. Finally, the performance of the different gestures
under the best performing simulation is shown.

\paragraph{Influence of Normalization} To illustrate the influence of the normalization the focus is on the set of
simulations that use \textit{Mid} as window size determination and \textit{HAveD} as threshold determination. The
Sakoe-Chiba band size is arbitrary and discussed in following paragraph on a smaller subset. Figure
\ref{fig:distance_measure_result} illustrates the influence of the normalization. DTW combined with the $\eta$
normalization reaches the best $F_{1}score_{\mu}$. The small cloud in the left corner with lower $F_{1}score_{\mu}$
values are too small Sakoe-Chiba band sizes.

\definecolor{light-gray}{gray}{0.8}

\begin{figure}
    \begin{minipage}{0.55\textwidth}
        \resizebox {\textwidth} {!} {
            \begin{tikzpicture}[spy using outlines={circle, magnification=6, connect spies}]
                \begin{axis}[
                    xmin=0,
                    xmax=1,
                    ymin=0,
                    ymax=1,
                    width=\axisdefaultwidth,
                    height=\axisdefaultwidth,
                    xlabel=$Precision_{\mu}$,
                    ylabel=$Recall_{\mu}$,
                    samples=100,
                    legend pos=outer north east]
                    \addplot[blue, only marks, mark size=1] table {../data/fig/distance_measure_result/dtw.dat};
                    \addlegendentry{DTW}
                    \addplot[red, only marks, mark size=1] table {../data/fig/distance_measure_result/ndtw.dat};
                    \addlegendentry{$\eta$DTW}
                    \addplot[green, only marks, mark size=1] table {../data/fig/distance_measure_result/n1dtw.dat};
                    \addlegendentry{$\eta '$DTW}
                    \addplot[gray, domain=0.051:1] {(0.1 * x) / (2 * x - 0.1)};
                    \addplot[gray, domain=0.11:1] {(0.2 * x) / (2 * x - 0.2)};
                    \addplot[gray, domain=0.16:1] {(0.3 * x) / (2 * x - 0.3)};
                    \addplot[gray, domain=0.21:1] {(0.4 * x) / (2 * x - 0.4)};
                    \addplot[gray, domain=0.26:1] {(0.5 * x) / (2 * x - 0.5)};
                    \addplot[gray, domain=0.31:1] {(0.6 * x) / (2 * x - 0.6)};
                    \addplot[gray, domain=0.36:1] {(0.7 * x) / (2 * x - 0.7)};
                    \addplot[gray, domain=0.41:1] {(0.8 * x) / (2 * x - 0.8)};
                    \addplot[gray, domain=0.46:1] {(0.9 * x) / (2 * x - 0.9)};
                    \coordinate (spypoint) at (axis cs:0.8413867433,0.6578651685);
                    \coordinate (magnifyglass) at (axis cs:0.2,0.8);
                \end{axis}
                \spy [size=2cm] on (spypoint)
                    in node[fill=white] at (magnifyglass);
            \end{tikzpicture}
        }
    \end{minipage}\hfill
    \begin{minipage}{0.4\textwidth}
        \caption{$Precision_{\mu}$ and $Recall_{\mu}$ of all simulations with \textit{Mid} as window determination and
        \textit{HAveD} as threshold determination, separated by the used time series distance measures DTW, $\eta$DTW
        or $\eta '$DTW. The magnifying glass is focusing on the best simulation measured on the underlying
        $F_{1}score_{\mu}$. Gray lines are illustrating the distribution of $F_{1}score_{\mu}$ in $\frac{1}{10}$ steps.}
        \label{fig:distance_measure_result}
    \end{minipage}
\end{figure}

\paragraph{Influence of Sakoe-Chiba Band Size} The influence of the Sakoe-Chiba band size on the dominating
configuration is shown on the subset of simulations that use DTW in combination with the $\eta$ normalization,
\textit{Mid} as window size determination and \textit{HAveD} as threshold determination. Figure
\ref{fig:sakoe-chiba_band_result} illustrates the influence of the Sakoe-Chiba band size on the $F_{1}score_{\mu}$ for
the best performing simulation without filter. There is a maximum at 36\% band size.

\begin{figure}
    \begin{minipage}{0.55\textwidth}
        \resizebox {\textwidth} {!} {
            \begin{tikzpicture}
                \begin{axis}[
                    xmin=0,
                    ymin=0.65,
                    xmax=200,
                    xlabel=band size in \% depending on input time series,
                    ylabel=$F_{1}score_{\mu}$,
                    width=\axisdefaultwidth,
                    height=0.75*\axisdefaultheight]
                    \addplot[blue, ultra thick] table {../data/fig/sakoe-chiba_band_result/scb.dat};
                \end{axis}
            \end{tikzpicture}
        }
    \end{minipage}\hfill
    \begin{minipage}{0.4\textwidth}
        \caption{The $F_{1}score_{\mu}$ depending on the size of the Sakoe-Chiba band size.}
        \label{fig:sakoe-chiba_band_result}
    \end{minipage}
\end{figure}

\paragraph{Influence of Threshold Determination} Three different threshold determinations approaches are tested. The
influence of the threshold determination is shown on the subset of simulations that use DTW with Sakoe-Chiba band size
36 \% in combination with the $\eta$ normalization and \textit{Mid} as window size determination. Figure
\ref{fig:threshold_result} illustrates the distribution of the three approaches. The \textit{HAveD} and the
\textit{HMidD} approaches reach the best $F_{1}score_{\mu}$ values, \textit{HAveD} a bit better as \textit{HMidD}.
Remarkable are the top $Precision_{\mu}$ values of the \textit{HMinD} threshold determination.

\begin{figure}
    \begin{minipage}{0.55\textwidth}
        \resizebox {\textwidth} {!} {
            \begin{tikzpicture}[spy using outlines={circle, magnification=6, connect spies}]
                \begin{axis}[
                    xmin=0,
                    xmax=1,
                    ymin=0,
                    ymax=1,
                    width=\axisdefaultwidth,
                    height=\axisdefaultwidth,
                    xlabel=$Precision_{\mu}$,
                    ylabel=$Recall_{\mu}$,
                    samples=100,
                    legend pos=outer north east]
                    \addplot[blue, only marks, mark size=1] table {../data/fig/threshold_result/haved.dat};
                    \addlegendentry{HAveD}
                    \addplot[red, only marks, mark size=1] table {../data/fig/threshold_result/hmidd.dat};
                    \addlegendentry{HMidD}
                    \addplot[green, only marks, mark size=1] table {../data/fig/threshold_result/hmind.dat};
                    \addlegendentry{HMinD}
                    \addplot[gray, domain=0.051:1] {(0.1 * x) / (2 * x - 0.1)};
                    \addplot[gray, domain=0.11:1] {(0.2 * x) / (2 * x - 0.2)};
                    \addplot[gray, domain=0.16:1] {(0.3 * x) / (2 * x - 0.3)};
                    \addplot[gray, domain=0.21:1] {(0.4 * x) / (2 * x - 0.4)};
                    \addplot[gray, domain=0.26:1] {(0.5 * x) / (2 * x - 0.5)};
                    \addplot[gray, domain=0.31:1] {(0.6 * x) / (2 * x - 0.6)};
                    \addplot[gray, domain=0.36:1] {(0.7 * x) / (2 * x - 0.7)};
                    \addplot[gray, domain=0.41:1] {(0.8 * x) / (2 * x - 0.8)};
                    \addplot[gray, domain=0.46:1] {(0.9 * x) / (2 * x - 0.9)};
                    \coordinate (spypoint) at (axis cs:0.8413867433,0.6578651685);
                    \coordinate (magnifyglass) at (axis cs:0.2,0.8);
                \end{axis}
                \spy [size=2cm] on (spypoint)
                    in node[fill=white] at (magnifyglass);
            \end{tikzpicture}
        }
    \end{minipage}\hfill
    \begin{minipage}{0.4\textwidth}
        \caption{$Precision_{\mu}$ and $Recall_{\mu}$ of all simulations that use $\eta$DTW with a Sakoe-Chiba band of size
        36\% depending on the input time series length and \textit{Mid} as window size determination, separated by the used
        threshold determination. The magnifying glass is focusing on the best simulation measured on the underlying
        $F_{1}score_{\mu}$. Gray lines are illustrating the distribution of $F_{1}score_{\mu}$ in $\frac{1}{10}$ steps.}
        \label{fig:threshold_result}
    \end{minipage}
\end{figure}

\paragraph{Influence of Window Size Determination} Four different window size determination approaches are tested. The
influence of the window size determination is shown on the subset of simulations that use DTW with Sakoe-Chiba band size
36 \% in combination with the $\eta$ normalization and \textit{HAveD} as threshold determination. Figure
\ref{fig:window_size_result} illustrates the distribution of the four approaches. It is clearly visible that the
\textit{Mid} window size determination reaches the best $Precision_{\mu}$, $Recall_{\mu}$ and $F_{1}score_{\mu}$
values in the experiment. The \textit{Ave} window size determination achieves also much better performance as
\textit{Max} and \textit{Min}.

\begin{figure}
    \begin{minipage}{0.55\textwidth}
        \resizebox {\textwidth} {!} {
            \begin{tikzpicture}[spy using outlines={circle, magnification=6, connect spies}]
                \begin{axis}[
                    xmin=0,
                    xmax=1,
                    ymin=0,
                    ymax=1,
                    width=\axisdefaultwidth,
                    height=\axisdefaultwidth,
                    xlabel=$Precision_{\mu}$,
                    ylabel=$Recall_{\mu}$,
                    samples=100,
                    legend pos=outer north east]
                    \addplot[blue, only marks, mark size=1] table {../data/fig/window_size_result/ave.dat};
                    \addlegendentry{Ave}
                    \addplot[red, only marks, mark size=1] table {../data/fig/window_size_result/max.dat};
                    \addlegendentry{Max}
                    \addplot[green, only marks, mark size=1] table {../data/fig/window_size_result/mid.dat};
                    \addlegendentry{Mid}
                    \addplot[violet, only marks, mark size=1] table {../data/fig/window_size_result/min.dat};
                    \addlegendentry{Min}
                    \addplot[gray, domain=0.051:1] {(0.1 * x) / (2 * x - 0.1)};
                    \addplot[gray, domain=0.11:1] {(0.2 * x) / (2 * x - 0.2)};
                    \addplot[gray, domain=0.16:1] {(0.3 * x) / (2 * x - 0.3)};
                    \addplot[gray, domain=0.21:1] {(0.4 * x) / (2 * x - 0.4)};
                    \addplot[gray, domain=0.26:1] {(0.5 * x) / (2 * x - 0.5)};
                    \addplot[gray, domain=0.31:1] {(0.6 * x) / (2 * x - 0.6)};
                    \addplot[gray, domain=0.36:1] {(0.7 * x) / (2 * x - 0.7)};
                    \addplot[gray, domain=0.41:1] {(0.8 * x) / (2 * x - 0.8)};
                    \addplot[gray, domain=0.46:1] {(0.9 * x) / (2 * x - 0.9)};
                    \coordinate (spypoint) at (axis cs:0.8413867433,0.6578651685);
                    \coordinate (magnifyglass) at (axis cs:0.2,0.8);
                \end{axis}
                \spy [size=2cm] on (spypoint)
                    in node[fill=white] at (magnifyglass);
            \end{tikzpicture}
        }
    \end{minipage}\hfill
    \begin{minipage}{0.4\textwidth}
        \caption{$Precision_{\mu}$ and $Recall_{\mu}$ of all simulations that use $\eta$DTW with a Sakoe-Chiba band of size
        36\% depending on the input time series length and \textit{HAveD} as threshold determination, separated
        by the used window size determination. The magnifying glass is focusing on the best simulation measured on the
        underlying $F_{1}score_{\mu}$. Gray lines are illustrating the distribution of $F_{1}score_{\mu}$ in
        $\frac{1}{10}$ steps.}
        \label{fig:window_size_result}
    \end{minipage}
\end{figure}

\paragraph{Influence of Time Series Filter Measure} The paragraphs and plots above are showing the effects of the
different configurations to find the best performing and dominating simulations. Interesting in this paragraph is the
influence of the different filter approaches on the dominating configuration. Figure
\ref{fig:sliding_window_filter_result} illustrates the influence of the time series filter measures on the performance
of the dominating configuration. 20 simulations are reaching a $F_{1}score_{\mu}$ value greater or equal to 0.7. Every
underlying time series measure for the filter is represented in those 20 simulations. One simulation with a filter
reaches the same $F_{1}score_{\mu}$ as the best performing simulation without filter.

\begin{figure}
    \begin{minipage}{0.55\textwidth}
        \resizebox {\textwidth} {!} {
            \begin{tikzpicture}[spy using outlines={circle, magnification=6, connect spies}]
                \begin{axis}[
                    xmin=0,
                    xmax=1,
                    ymin=0,
                    ymax=1,
                    width=\axisdefaultwidth,
                    height=\axisdefaultwidth,
                    xlabel=$Precision_{\mu}$,
                    ylabel=$Recall_{\mu}$,
                    samples=100,
                    legend pos=outer north east]
                    \addplot[blue, mark=o, only marks] table {../data/fig/sliding_window_filter_result/lnce.dat};
                    \addlegendentry{LNCE}
                    \addplot[red, mark=triangle, only marks] table {../data/fig/sliding_window_filter_result/var.dat};
                    \addlegendentry{VAR}
                    \addplot[black, mark=x, only marks] table {../data/fig/sliding_window_filter_result/nofilter.dat};
                    \addlegendentry{No Filter}
                    \addplot[gray, domain=0.051:1] {(0.1 * x) / (2 * x - 0.1)};
                    \addplot[gray, domain=0.11:1] {(0.2 * x) / (2 * x - 0.2)};
                    \addplot[gray, domain=0.16:1] {(0.3 * x) / (2 * x - 0.3)};
                    \addplot[gray, domain=0.21:1] {(0.4 * x) / (2 * x - 0.4)};
                    \addplot[gray, domain=0.26:1] {(0.5 * x) / (2 * x - 0.5)};
                    \addplot[gray, domain=0.31:1] {(0.6 * x) / (2 * x - 0.6)};
                    \addplot[gray, domain=0.36:1] {(0.7 * x) / (2 * x - 0.7)};
                    \addplot[gray, domain=0.41:1] {(0.8 * x) / (2 * x - 0.8)};
                    \addplot[gray, domain=0.46:1] {(0.9 * x) / (2 * x - 0.9)};
                    \coordinate (spypoint) at (axis cs:0.8413867433,0.6578651685);
                    \coordinate (magnifyglass) at (axis cs:0.2,0.8);
                \end{axis}
                \spy [size=2cm] on (spypoint)
                    in node[fill=white] at (magnifyglass);
            \end{tikzpicture}
        }
    \end{minipage}\hfill
    \begin{minipage}{0.4\textwidth}
    \caption{$Precision_{\mu}$ and $Recall_{\mu}$ of all simulations that use $\eta$DTW with a Sakoe-Chiba band of size
        36\% depending on the input time series length, \textit{HAveD} as threshold determination and \textit{Mid} as window
        size determination, separated by the used sliding window filter. The magnifying glass is focusing on the best
        simulation measured on the underlying $F_{1}score_{\mu}$. Gray lines are illustrating the distribution of
        $F_{1}score_{\mu}$ in $\frac{1}{10}$ steps.}
        \label{fig:sliding_window_filter_result}
    \end{minipage}
\end{figure}

\paragraph{Influence of Filter Interval Size} The filter interval size has influence on the performed
$F_{1}score_{\mu}$ and on the amount of nearest neighbour classification calls. The right plot of figure
\ref{fig:blur_factor_result} illustrates the influence of the filter interval size on the $F_{1}score_{\mu}$. Only the
LNCE estimate is reaching with the given factors the same $F_{1}score_{\mu}$ as the dominating configuration without
filter. However, the two filter types come also with small factors very close to the top $F_{1}score_{\mu}$ value.

\begin{figure}
    \begin{center}
        \resizebox {\textwidth} {!} {
            \begin{tabular}{cc}
                \resizebox {!} {\height} {
                    \begin{tikzpicture}
                        \begin{axis}[
                            legend pos=south east,
                            xmin=100,
                            xmax=300,
                            ymin=0.6,
                            ymax=0.75,
                            xlabel=filter interval size in \%,
                            ylabel=$F_{1}score_{\mu}$,
                            width=\axisdefaultwidth,
                            height=0.7*\axisdefaultheight]
                            \addplot[blue, ultra thick] table {../data/fig/blur_factor_result/lnce.dat};
                            \addlegendentry{LNCE}
                            \addplot[red, ultra thick] table {../data/fig/blur_factor_result/var.dat};
                            \addlegendentry{VAR}
                            \addplot[dotted, black, domain=100:300] {0.738393631276109};
                            \addlegendentry{No Filter}
                        \end{axis}
                    \end{tikzpicture}
                } &
                \resizebox {!} {\height} {
                    \begin{tikzpicture}
                        \begin{axis}[
                            legend pos=south east,
                            xmin=100,
                            xmax=300,
                            ymin=0,
                            ymax=5500,
                            xlabel=filter interval size in \%,
                            ylabel=\# 1NN-DTW calls,
                            width=\axisdefaultwidth,
                            height=0.7*\axisdefaultheight]
                            \addplot[blue, ultra thick] table {../data/fig/nnc_calls_result/lnce.dat};
                            \addlegendentry{LNCE}
                            \addplot[red, ultra thick] table {../data/fig/nnc_calls_result/var.dat};
                            \addlegendentry{VAR}
                            \addplot[dotted, black, domain=100:300] {4893};
                            \addlegendentry{No Filter}
                        \end{axis}
                    \end{tikzpicture}
                }
            \end{tabular}
        }
    \end{center}
    \caption{The left plot shows the $F_{1}score_{\mu}$ depending on the factor that expands the size of the filter
    interval in \%. The right plot shows the amount of 1NN-DTW calls depending on the factor that expands the size
    of the filter interval in \%.}
    \label{fig:blur_factor_result}
\end{figure}

\paragraph{Differentiation according to Gestures} Also interesting are the different results depending on the gesture.
Figure \ref{fig:gesture_result} shows the $Precision$ and $Recall$ values for
every gesture over all records on the best performing simulation.

\begin{figure}
    \begin{minipage}{0.4\textwidth}
        \resizebox {\textwidth} {!} {
            \begin{tikzpicture}
                \begin{axis}[
                    xmin=0,
                    xmax=1,
                    ymin=0,
                    ymax=1,
                    width=\axisdefaultwidth,
                    height=\axisdefaultwidth,
                    xlabel=$Precision$,
                    ylabel=$Recall$,
                    samples=100,
                    legend pos=outer north east]
                    \addplot+[
                        blue,
                        only marks,
                        nodes near coords,
                        every node near coord/.style={at={(-0.05,0)}, color=black},
                        point meta=explicit symbolic] table[x=x, y=y, meta=label] {../data/fig/gesture_result/gesture.dat};
                    \addplot[gray, domain=0.051:1] {(0.1 * x) / (2 * x - 0.1)};
                    \addplot[gray, domain=0.11:1] {(0.2 * x) / (2 * x - 0.2)};
                    \addplot[gray, domain=0.16:1] {(0.3 * x) / (2 * x - 0.3)};
                    \addplot[gray, domain=0.21:1] {(0.4 * x) / (2 * x - 0.4)};
                    \addplot[gray, domain=0.26:1] {(0.5 * x) / (2 * x - 0.5)};
                    \addplot[gray, domain=0.31:1] {(0.6 * x) / (2 * x - 0.6)};
                    \addplot[gray, domain=0.36:1] {(0.7 * x) / (2 * x - 0.7)};
                    \addplot[gray, domain=0.41:1] {(0.8 * x) / (2 * x - 0.8)};
                    \addplot[gray, domain=0.46:1] {(0.9 * x) / (2 * x - 0.9)};
                \end{axis}
            \end{tikzpicture}
        }
    \end{minipage}\hfill
    \begin{minipage}{0.55\textwidth}
        \caption{$Precision$ and $Recall$ of all gestures over all records for the simulation that uses $\eta$DTW with a
        Sakoe-Chiba band of size 36\% depending on the input time series length, \textit{HAveD} as threshold determination
        and \textit{Mid} as window size determination without any filter. Gesture $A$ is the first of all eight gestures
        and gesture $H$ is the eighth gesture. Gray lines are illustrating the distribution of $F_{1}score$ in $\frac{1}{10}$ steps.}
        \label{fig:gesture_result}
    \end{minipage}
\end{figure}
