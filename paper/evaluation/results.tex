\subsection{Results} \label{results}

In order to evaluate the influence of all model parameters that were described in Section \ref{experiment}, we performed a total number of 28152 experiments.
Figure \ref{fig:result}(a) illustrates the $Precision_{\mu}$ and $Recall_{\mu}$ values for all test runs.
A top performance of around 0.7384 $F_{1}score_{\mu}$ was achieved by parameter configurations that
used $\eta$ time series normalization, DTW with a Sakoe-Chiba band of about 18 \% time series length, \textit{mid} window size, and \textit{HAvgD} for threshold determination.

Given the best parameter configuration, we investigated the influence of the individual parameters by changing only one at a time and fixing the others, see Figure \ref{fig:result}(b,c,d).
As shown in Figure \ref{fig:result}(e), we also evaluated the performance with $VAR$, $LNCE$, and no filter.
The best results for each individual gesture is shown in Figure Figure \ref{fig:result}(f).
Further tests on the applicability of the sliding window filter as well as our interpretation of the results are presented below.

\begin{figure}
    \begin{tabular}{lclc}
        a) &
        \adjustbox{valign=t}{
            \resizebox {0.45\textwidth} {!} {
                \begin{tikzpicture}[spy using outlines={circle, magnification=6, connect spies}]
                    \begin{axis}[
                        xmin=0,
                        xmax=1,
                        ymin=0,
                        ymax=1,
                        width=\axisdefaultwidth,
                        height=\axisdefaultwidth,
                        xlabel=$Precision_{\mu}$,
                        ylabel=$Recall_{\mu}$,
                        samples=100,
                        colorbar horizontal,
                        colormap/viridis high res,
                        title=All Simulations]
                        % \addplot[only marks, scatter, scatter src=explicit, mark size=1] table[x=x,y=y,meta=fscore] {../data/fig/result2/result.dat};
                        \addplot[gray, domain=0.051:1] {(0.1 * x) / (2 * x - 0.1)};
                        \addplot[gray, domain=0.11:1] {(0.2 * x) / (2 * x - 0.2)};
                        \addplot[gray, domain=0.16:1] {(0.3 * x) / (2 * x - 0.3)};
                        \addplot[gray, domain=0.21:1] {(0.4 * x) / (2 * x - 0.4)};
                        \addplot[gray, domain=0.26:1] {(0.5 * x) / (2 * x - 0.5)};
                        \addplot[gray, domain=0.31:1] {(0.6 * x) / (2 * x - 0.6)};
                        \addplot[gray, domain=0.36:1] {(0.7 * x) / (2 * x - 0.7)};
                        \addplot[gray, domain=0.41:1] {(0.8 * x) / (2 * x - 0.8)};
                        \addplot[gray, domain=0.46:1] {(0.9 * x) / (2 * x - 0.9)};
                        \coordinate (spypoint) at (axis cs:0.8413867433,0.6578651685);
                        \coordinate (magnifyglass) at (axis cs:0.2,0.8);
                    \end{axis}
                    \spy [size=2cm] on (spypoint)
                        in node[fill=white] at (magnifyglass);
                \end{tikzpicture}
            }
        } &
        b) &
        \adjustbox{valign=t}{
            \resizebox {0.45\textwidth} {!} {
                \begin{tikzpicture}[spy using outlines={circle, magnification=6, connect spies}]
                    \begin{axis}[
                        title=Influence of Normalization,
                        xmin=0,
                        xmax=1,
                        ymin=0,
                        ymax=1,
                        width=\axisdefaultwidth,
                        height=\axisdefaultwidth,
                        xlabel=$Precision_{\mu}$,
                        ylabel=$Recall_{\mu}$,
                        samples=100,
                        legend style={at={(0.5,-0.15)}, anchor=north,legend columns=-1}]
                        \addplot[blue, only marks, mark size=1] table {../data/fig/distance_measure_result/dtw.dat};
                        \addlegendentry{no}
                        \addplot[red, only marks, mark size=1] table {../data/fig/distance_measure_result/ndtw.dat};
                        \addlegendentry{$\eta$-norm}
                        \addplot[green, only marks, mark size=1] table {../data/fig/distance_measure_result/n1dtw.dat};
                        \addlegendentry{$z$-norm}
                        \addplot[gray, domain=0.051:1] {(0.1 * x) / (2 * x - 0.1)};
                        \addplot[gray, domain=0.11:1] {(0.2 * x) / (2 * x - 0.2)};
                        \addplot[gray, domain=0.16:1] {(0.3 * x) / (2 * x - 0.3)};
                        \addplot[gray, domain=0.21:1] {(0.4 * x) / (2 * x - 0.4)};
                        \addplot[gray, domain=0.26:1] {(0.5 * x) / (2 * x - 0.5)};
                        \addplot[gray, domain=0.31:1] {(0.6 * x) / (2 * x - 0.6)};
                        \addplot[gray, domain=0.36:1] {(0.7 * x) / (2 * x - 0.7)};
                        \addplot[gray, domain=0.41:1] {(0.8 * x) / (2 * x - 0.8)};
                        \addplot[gray, domain=0.46:1] {(0.9 * x) / (2 * x - 0.9)};
                        \coordinate (spypoint) at (axis cs:0.8413867433,0.6578651685);
                        \coordinate (magnifyglass) at (axis cs:0.2,0.8);
                    \end{axis}
                    \spy [size=2cm] on (spypoint)
                        in node[fill=white] at (magnifyglass);
                \end{tikzpicture}
            }
        } \\
        c) &
        \adjustbox{valign=t}{
            \resizebox {0.45\textwidth} {!} {
                \begin{tikzpicture}[spy using outlines={circle, magnification=6, connect spies}]
                    \begin{axis}[
                        title=Influence of Threshold Determination,
                        xmin=0,
                        xmax=1,
                        ymin=0,
                        ymax=1,
                        width=\axisdefaultwidth,
                        height=\axisdefaultwidth,
                        xlabel=$Precision_{\mu}$,
                        ylabel=$Recall_{\mu}$,
                        samples=100,
                        legend style={at={(0.5,-0.15)}, anchor=north,legend columns=-1}]
                        \addplot[blue, only marks, mark size=1] table {../data/fig/threshold_result/haved.dat};
                        \addlegendentry{HAvgD}
                        \addplot[red, only marks, mark size=1] table {../data/fig/threshold_result/hmidd.dat};
                        \addlegendentry{HMidD}
                        \addplot[green, only marks, mark size=1] table {../data/fig/threshold_result/hmind.dat};
                        \addlegendentry{HMinD}
                        \addplot[gray, domain=0.051:1] {(0.1 * x) / (2 * x - 0.1)};
                        \addplot[gray, domain=0.11:1] {(0.2 * x) / (2 * x - 0.2)};
                        \addplot[gray, domain=0.16:1] {(0.3 * x) / (2 * x - 0.3)};
                        \addplot[gray, domain=0.21:1] {(0.4 * x) / (2 * x - 0.4)};
                        \addplot[gray, domain=0.26:1] {(0.5 * x) / (2 * x - 0.5)};
                        \addplot[gray, domain=0.31:1] {(0.6 * x) / (2 * x - 0.6)};
                        \addplot[gray, domain=0.36:1] {(0.7 * x) / (2 * x - 0.7)};
                        \addplot[gray, domain=0.41:1] {(0.8 * x) / (2 * x - 0.8)};
                        \addplot[gray, domain=0.46:1] {(0.9 * x) / (2 * x - 0.9)};
                        \coordinate (spypoint) at (axis cs:0.8413867433,0.6578651685);
                        \coordinate (magnifyglass) at (axis cs:0.2,0.8);
                    \end{axis}
                    \spy [size=2cm] on (spypoint)
                        in node[fill=white] at (magnifyglass);
                \end{tikzpicture}
            }
        } &
        d) &
        \adjustbox{valign=t}{
            \resizebox {0.45\textwidth} {!} {
                \begin{tikzpicture}[spy using outlines={circle, magnification=6, connect spies}]
                    \begin{axis}[
                        title=Influence of Window Size Determination,
                        xmin=0,
                        xmax=1,
                        ymin=0,
                        ymax=1,
                        width=\axisdefaultwidth,
                        height=\axisdefaultwidth,
                        xlabel=$Precision_{\mu}$,
                        ylabel=$Recall_{\mu}$,
                        samples=100,
                        legend style={at={(0.5,-0.15)}, anchor=north,legend columns=-1}]
                        \addplot[blue, only marks, mark size=1] table {../data/fig/window_size_result/ave.dat};
                        \addlegendentry{avg\vphantom{d}}
                        \addplot[red, only marks, mark size=1] table {../data/fig/window_size_result/max.dat};
                        \addlegendentry{max\vphantom{dg}}
                        \addplot[green, only marks, mark size=1] table {../data/fig/window_size_result/mid.dat};
                        \addlegendentry{mid\vphantom{g}}
                        \addplot[violet, only marks, mark size=1] table {../data/fig/window_size_result/min.dat};
                        \addlegendentry{min\vphantom{dg}}
                        \addplot[gray, domain=0.051:1] {(0.1 * x) / (2 * x - 0.1)};
                        \addplot[gray, domain=0.11:1] {(0.2 * x) / (2 * x - 0.2)};
                        \addplot[gray, domain=0.16:1] {(0.3 * x) / (2 * x - 0.3)};
                        \addplot[gray, domain=0.21:1] {(0.4 * x) / (2 * x - 0.4)};
                        \addplot[gray, domain=0.26:1] {(0.5 * x) / (2 * x - 0.5)};
                        \addplot[gray, domain=0.31:1] {(0.6 * x) / (2 * x - 0.6)};
                        \addplot[gray, domain=0.36:1] {(0.7 * x) / (2 * x - 0.7)};
                        \addplot[gray, domain=0.41:1] {(0.8 * x) / (2 * x - 0.8)};
                        \addplot[gray, domain=0.46:1] {(0.9 * x) / (2 * x - 0.9)};
                        \coordinate (spypoint) at (axis cs:0.8413867433,0.6578651685);
                        \coordinate (magnifyglass) at (axis cs:0.2,0.8);
                    \end{axis}
                    \spy [size=2cm] on (spypoint)
                        in node[fill=white] at (magnifyglass);
                \end{tikzpicture}
            }
        } \\
        e) &
        \adjustbox{valign=t}{
            \resizebox {0.45\textwidth} {!} {
                \begin{tikzpicture}[spy using outlines={circle, magnification=6, connect spies}]
                    \begin{axis}[
                        title=Influence of Time Series Filter Measure,
                        xmin=0,
                        xmax=1,
                        ymin=0,
                        ymax=1,
                        width=\axisdefaultwidth,
                        height=\axisdefaultwidth,
                        xlabel=$Precision_{\mu}$,
                        ylabel=$Recall_{\mu}$,
                        samples=100,
                        legend style={at={(0.5,-0.15)}, anchor=north,legend columns=-1}]
                        \addplot[blue, mark=o, only marks] table {../data/fig/sliding_window_filter_result/lnce.dat};
                        \addlegendentry{LNCE}
                        \addplot[red, mark=triangle, only marks] table {../data/fig/sliding_window_filter_result/var.dat};
                        \addlegendentry{VAR}
                        \addplot[black, mark=x, only marks] table {../data/fig/sliding_window_filter_result/nofilter.dat};
                        \addlegendentry{No Filter}
                        \addplot[gray, domain=0.051:1] {(0.1 * x) / (2 * x - 0.1)};
                        \addplot[gray, domain=0.11:1] {(0.2 * x) / (2 * x - 0.2)};
                        \addplot[gray, domain=0.16:1] {(0.3 * x) / (2 * x - 0.3)};
                        \addplot[gray, domain=0.21:1] {(0.4 * x) / (2 * x - 0.4)};
                        \addplot[gray, domain=0.26:1] {(0.5 * x) / (2 * x - 0.5)};
                        \addplot[gray, domain=0.31:1] {(0.6 * x) / (2 * x - 0.6)};
                        \addplot[gray, domain=0.36:1] {(0.7 * x) / (2 * x - 0.7)};
                        \addplot[gray, domain=0.41:1] {(0.8 * x) / (2 * x - 0.8)};
                        \addplot[gray, domain=0.46:1] {(0.9 * x) / (2 * x - 0.9)};
                        \coordinate (spypoint) at (axis cs:0.8413867433,0.6578651685);
                        \coordinate (magnifyglass) at (axis cs:0.2,0.8);
                    \end{axis}
                    \spy [size=2cm] on (spypoint)
                        in node[fill=white] at (magnifyglass);
                \end{tikzpicture}
            }
        } &
        f) &
        \adjustbox{valign=t}{
            \resizebox {0.45\textwidth} {!} {
                \begin{tikzpicture}
                    \begin{axis}[
                        xmin=0,
                        xmax=1,
                        ymin=0,
                        ymax=1,
                        width=\axisdefaultwidth,
                        height=\axisdefaultwidth,
                        xlabel=$Precision_{\phantom{\mu}}$,
                        ylabel=$Recall_{\phantom{\mu}}$,
                        samples=100,
                        title=Differentiation according to Gestures,
                        legend style={at={(0.5,-0.15)}, anchor=north,legend columns=-1}]
                        \addplot+[
                            blue,
                            only marks,
                            nodes near coords,
                            every node near coord/.style={at={(-0.05,0)}, color=black},
                            point meta=explicit symbolic] table[x=x, y=y, meta=label] {../data/fig/gesture_result/gesture.dat};
                        \addplot[gray, domain=0.051:1] {(0.1 * x) / (2 * x - 0.1)};
                        \addplot[gray, domain=0.11:1] {(0.2 * x) / (2 * x - 0.2)};
                        \addplot[gray, domain=0.16:1] {(0.3 * x) / (2 * x - 0.3)};
                        \addplot[gray, domain=0.21:1] {(0.4 * x) / (2 * x - 0.4)};
                        \addplot[gray, domain=0.26:1] {(0.5 * x) / (2 * x - 0.5)};
                        \addplot[gray, domain=0.31:1] {(0.6 * x) / (2 * x - 0.6)};
                        \addplot[gray, domain=0.36:1] {(0.7 * x) / (2 * x - 0.7)};
                        \addplot[gray, domain=0.41:1] {(0.8 * x) / (2 * x - 0.8)};
                        \addplot[gray, domain=0.46:1] {(0.9 * x) / (2 * x - 0.9)};
                    \end{axis}
                \end{tikzpicture}
            }
        }
    \end{tabular}
    \caption{
    $Precision_{\mu}$ and $Recall_{\mu}$ plots illustrating the performance influence of the individual model parameters.
    The magnifying glass is focusing on the results with the highest $F_{1}score_{\mu}$.
    Gray lines indicate the $F_{1}score_{\mu}$ distribution in $\frac{1}{10}$ steps.}
        \label{fig:result}
\end{figure}
\definecolor{light-gray}{gray}{0.8}


\paragraph{Normalization:}

The influence of the time series normalization is illustrated in Figure \ref{fig:result}(b). We compare $\eta$, $z$, and no normalization, with \textit{mid} window size and \textit{HAvgD} dissimilarity threshold. The best $F_{1}score_{\mu}$ was achieved by means of the $\eta$ normalization, which corresponds to the data points shown in the magnifying glass. The point cloud in the lower left corner of plot \ref{fig:result}(b) are parameter settings with rather small warping band.


\paragraph{Warping Band:}

The influence of the Sakoe-Chiba band in combination with the DTW distance is shown below in Figure \ref{fig:sakoe-chiba_band_result}.
For this experiment we selected only the dominating parameter settings, with $\eta$ normalization, \textit{mid} window size, and and \textit{HAvgD} dissimilarity threshold. The best $F_{1}score_{\mu}$ was achieved with a band with of 18 \% time series length.

\begin{figure}
    \begin{minipage}{0.55\textwidth}
        \resizebox {\textwidth} {!} {
            \begin{tikzpicture}
                \begin{axis}[
                    xmin=0,
                    ymin=0.65,
                    xmax=100,
                    xlabel=band size in \% depending on input time series,
                    ylabel=$F_{1}score_{\mu}$,
                    width=\axisdefaultwidth,
                    height=0.6*\axisdefaultheight]
                    \addplot[blue, ultra thick] table[x expr=0.5*\thisrowno{0}, y=y] {../data/fig/sakoe-chiba_band_result/scb.dat};
                \end{axis}
            \end{tikzpicture}
        }
    \end{minipage}\hfill
    \begin{minipage}{0.4\textwidth}
        \caption{Performance with varying Sakoe-Chiba band width.}
        \label{fig:sakoe-chiba_band_result}
    \end{minipage}
\end{figure}


\paragraph{Dissimilarity Threshold:}

We evaluate three different ways of determining a dissimilarity threshold, namely \textit{HMinD}, \textit{HAvgD}, and \textit{HMidD}. For our comparison in Figure \ref{fig:result}(c), we used $\eta$ normalization, \textit{mid} window size, and a warping band of 18 \% time series length. The best $F_{1}score_{\mu}$ was achieved by means of \textit{HAvgD}, shortly followed by the \textit{HMidD} approach. Comparatively high $Precision_{\mu}$ values were given by \textit{HMinD} threshold.


\paragraph{Window Size:}

The influence of the window size determination approach is shown in Figure \ref{fig:result}(d). We compare \textit{min}, \textit{max}, \textit{avg}, and \textit{mid} window size, with $\eta$ normalization, \textit{HAvgD} dissimilarity threshold, and a warping and of 18 \% time series length. The highest $Precision_{\mu}$, $Recall_{\mu}$ and $F_{1}score_{\mu}$ was achieved by the \textit{mid} window size, shortly followed by the \textit{avg} window size. Figure \ref{fig:result}(d) furthermore suggests to refrain from using \textit{max} and \textit{min} window size determination.


\paragraph{Filtering Approach:}

Given the optimal parameter setting that was determined in the previous experiments, we are now in the position to assess the influence of the filtering approach. Figure \ref{fig:result}(e) shows the performance with $VAR$, $LNCE$, and no filter. Note that only results with a
$F_{1}score_{\mu}$ of at least 0.7 are plotted. Interestingly, top performance was achieved with with and without filter. This lead is to the question of computational complexity, which is answered in following paragraph.

\paragraph{Filter Interval:}

The filter interval does not only influence the resulting $F_{1}score_{\mu}$, but also the amount of time series dissimilarity comparisons.
Figure \ref{fig:blur_factor_result} illustrates the influence of the interval size in respect to performance and computational demand.
With an appropriate filter interval of about 200 \% we are able to reduce the number of dissimilarity comparisons by one half, while still achieving relatively high performance values.

\begin{figure}
    \begin{center}
        \resizebox {\textwidth} {!} {
            \begin{tabular}{cc}
                \resizebox {!} {\height} {
                    \begin{tikzpicture}
                        \begin{axis}[
                            legend pos=south east,
                            xmin=100,
                            xmax=300,
                            ymin=0.6,
                            ymax=0.75,
                            xlabel=size of filter interval in \%,
                            ylabel=$F_{1}score_{\mu}$,
                            width=\axisdefaultwidth,
                            height=0.7*\axisdefaultheight]
                            \addplot[blue, ultra thick] table {../data/fig/blur_factor_result/lnce.dat};
                            \addlegendentry{LNCE}
                            \addplot[red, ultra thick] table {../data/fig/blur_factor_result/var.dat};
                            \addlegendentry{VAR}
                            \addplot[dotted, black, domain=100:300] {0.738393631276109};
                            \addlegendentry{No Filter}
                        \end{axis}
                    \end{tikzpicture}
                } &
                \resizebox {!} {\height} {
                    \begin{tikzpicture}
                        \begin{axis}[
                            legend pos=south east,
                            xmin=100,
                            xmax=300,
                            ymin=0,
                            ymax=5500,
                            xlabel=size of filter interval in \%,
                            ylabel=\# 1NN-DTW calls,
                            width=\axisdefaultwidth,
                            height=0.7*\axisdefaultheight]
                            \addplot[blue, ultra thick] table {../data/fig/nnc_calls_result/lnce.dat};
                            \addlegendentry{LNCE}
                            \addplot[red, ultra thick] table {../data/fig/nnc_calls_result/var.dat};
                            \addlegendentry{VAR}
                            \addplot[dotted, black, domain=100:300] {4893};
                            \addlegendentry{No Filter}
                        \end{axis}
                    \end{tikzpicture}
                }
            \end{tabular}
        }
    \end{center}
    \caption{Influence of filter interval on the $F_{1}score_{\mu}$ (left) and the amount of 1NN-DTW dissimilarity comparisons (right). An optimal tradeoff between performance and computational demand is achieved at approximately 150 \% to 200 \% filter interval.}
    \label{fig:blur_factor_result}
\end{figure}


\paragraph{Individual Gestures}

Finally, we have tested the performance for each of the examined gestures separately.
Figure \ref{fig:result}(e) shows the best $Precision$ and $Recall$ values that were achieved for each gesture.
The results demonstrate that some gestures are easier to recognize than others.
