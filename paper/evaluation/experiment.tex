\subsection{Experiment} \label{experiment}

An online sliding window application can be configured and implemented in many different ways. The following five main
configurations with their associated options are used in this paper.

\begin{itemize}
    \item The \textbf{Window Size} defines the size of the observed time series window. Four different approaches were
        tested.
        \begin{itemize}
            \item The \textbf{Max} approach works with the maximum length of the training gestures.
            \item The \textbf{Min} approach works with the minimum length of the training gestures.
            \item The \textbf{Ave} approach works with the average length of the training gestures.
            \item The \textbf{Mid} approach works with the average length between the maximum and minimum of the
                training gestures.
        \end{itemize}
    \item The \textbf{Step Size} defines the gap size between the observed time series windows. This values is
        intuitively defined to one tenth of the window size.
    \item The \textbf{Time Series Distance Measure} calculates the distance between the observed time series window and
        a training gesture. This time series distance calcualtion is the fundamental for a nearest neighbour
        classification on time series. DTW is the only approach in this experiment. Both compared time series are
        prepared by three different methods before calculating the simularity with DTW. Once with the $\eta$
        normalization, once the $\eta '$ nrmalization and once without normalization. In addition DTW is combined with
        34 different Sakoe-Chiba band sizes from 2 \% to 200 \%.
    \item The \textbf{Threshold} defines for every gesture class the upper bound of the time series distance measure
        between the observed time series window and a training gesture. A time series window is detected as a specific
        gesture if the neareast neighbour associated with the window is an instance of the same gesture class and the
        distance between those two time series is less than or equal the threshold of the gesture class. Three different
        approaches were tested as threshold determination.
        \begin{itemize}
            \item The \textbf{HMinD} approach determines the threshold of a gesture class by the half minimum distance
                of a class to all other gesture classes in the training data set.
            \item The \textbf{HAveD} approach determines the threshold of a gesture class by the half average distance
                of a class to all other gesture classes in the training data set.
            \item The \textbf{HMidD} approach determines the threshold of a gesture class by the half average of the
                minimum and maximum distance of a class to all other gesture classes in the training data set.
        \end{itemize}
    \item The \textbf{Time Series Filter Measure} is the fundamental of the presented filter approach. Two different
        time series measures are tested agains simulations without filter.
        \begin{itemize}
            \item The \textbf{LNCE} measure is the length normalized Complexity Estimate of a time series.
            \item The \textbf{VAR} measure is the sample variance of a time series.
        \end{itemize}
        Both approaches are tested with eleven different factors that increased the size of the passing filter interval
        from 100 \% to 300 \%.
\end{itemize}

An example result of a configured online gesture detection simulation is illustrated by figure \ref{fig:experiment}.
True positive class detections are shown as green rectangles, false positive class detections are shown as red
rectangles and false negative class detections are shown as blue rectangles. True negatives are white resp.
transparent. Apart from the not always avoidable small
false detections around the green rectangles, are seven gestures detected correctly and one gesture is falsely detected
in the presented example.

\begin{figure}
    \resizebox {\textwidth} {!} {
        \begin{tikzpicture}
            \begin{axis}[
                xmin=0,
                xmax=2426,
                ymin=-16,
                ymax=16,
                width=10*\axisdefaultwidth,
                height=\axisdefaultheight,
                xticklabels={,,},
                yticklabels={,,}]
                \addplot[blue, mark=none, opacity=0.4] table[x=t, y=x] {../data/fig/experimentee_result2/exp1.dat};
                \addplot[red, mark=none, opacity=0.4] table[x=t, y=y] {../data/fig/experimentee_result2/exp1.dat};
                \addplot[green, mark=none, opacity=0.4] table[x=t, y=z] {../data/fig/experimentee_result2/exp1.dat};
                \addplot+[fill, opacity=0.5, red, mark=none] coordinates {(294, -16) (307, -16) (307, 16) (294, 16)} --cycle;
                \addplot+[fill, opacity=0.5, green, mark=none] coordinates {(307, -16) (357, -16) (357, 16) (307, 16)} --cycle;
                \addplot+[fill, opacity=0.5, red, mark=none] coordinates {(357, -16) (359, -16) (359, 16) (357, 16)} --cycle;
                \addplot+[fill, opacity=0.5, red, mark=none] coordinates {(497, -16) (508, -16) (508, 16) (497, 16)} --cycle;
                \addplot+[fill, opacity=0.5, green, mark=none] coordinates {(508, -16) (562, -16) (562, 16) (508, 16)} --cycle;
                \addplot+[fill, opacity=0.5, blue, mark=none] coordinates {(562, -16) (564, -16) (564, 16) (562, 16)} --cycle;
                \addplot+[fill, opacity=0.5, red, mark=none] coordinates {(712, -16) (722, -16) (722, 16) (712, 16)} --cycle;
                \addplot+[fill, opacity=0.5, green, mark=none] coordinates {(722, -16) (777, -16) (777, 16) (722, 16)} --cycle;
                \addplot+[fill, opacity=0.5, blue, mark=none] coordinates {(777, -16) (778, -16) (778, 16) (777, 16)} --cycle;
                \addplot+[fill, opacity=0.5, blue, mark=none] coordinates {(940, -16) (945, -16) (945, 16) (940, 16)} --cycle;
                \addplot+[fill, opacity=0.5, green, mark=none] coordinates {(945, -16) (1010, -16) (1010, 16) (945, 16)} --cycle;
                \addplot+[fill, opacity=0.5, blue, mark=none] coordinates {(1010, -16) (1023, -16) (1023, 16) (1010, 16)} --cycle;
                \addplot+[fill, opacity=0.5, red, mark=none] coordinates {(1310, -16) (1316, -16) (1316, 16) (1310, 16)} --cycle;
                \addplot+[fill, opacity=0.5, green, mark=none] coordinates {(1316, -16) (1367, -16) (1367, 16) (1316, 16)} --cycle;
                \addplot+[fill, opacity=0.5, red, mark=none] coordinates {(1367, -16) (1375, -16) (1375, 16) (1367, 16)} --cycle;
                \addplot+[fill, opacity=0.5, red, mark=none] coordinates {(1681, -16) (1689, -16) (1689, 16) (1681, 16)} --cycle;
                \addplot+[fill, opacity=0.5, green, mark=none] coordinates {(1689, -16) (1746, -16) (1746, 16) (1689, 16)} --cycle;
                \addplot+[fill, opacity=0.5, blue, mark=none] coordinates {(1746, -16) (1748, -16) (1748, 16) (1746, 16)} --cycle;
                \addplot+[fill, opacity=0.5, red, mark=none] coordinates {(2082, -16) (2090, -16) (2090, 16) (2082, 16)} --cycle;
                \addplot+[fill, opacity=0.5, green, mark=none] coordinates {(2090, -16) (2146, -16) (2146, 16) (2090, 16)} --cycle;
                \addplot+[fill, opacity=0.5, red, mark=none] coordinates {(2146, -16) (2147, -16) (2147, 16) (2146, 16)} --cycle;
                \addplot+[fill, opacity=0.5, blue, mark=none] coordinates {(2311, -16) (2388, -16) (2388, 16) (2311, 16)} --cycle;
                \addplot+[fill, opacity=0.5, red, mark=none] coordinates {(2297, -16) (2362, -16) (2362, 16) (2297, 16)} --cycle;
            \end{axis}
        \end{tikzpicture}
    }
    \caption{An example result of a simulated online gesture detection on a test time series stream.}
    \label{fig:experiment}
\end{figure}
