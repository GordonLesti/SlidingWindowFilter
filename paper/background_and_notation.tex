\section{Background and Notation} \label{background_and_notation}

This section gives more background on the sliding window technique \cite{keogh2004sliding}, DTW distance measure \cite{keogh2002exact}, and time series normalization \cite{das1998rule}, which are fundamental building blocks of our conducted online gesture recognition study \cite{lesti2017filter}. Table \ref{tab:notation} introduces the notation that we use for formal problem description.

\begin{table}
    \begin{center}
        \begin{tabularx}{\textwidth}{c X}
            \hline
            \textbf{Symbol} \qquad & \textbf{Description}\\
            \hline
            $\mathbb{U}$ & a set containing all items of a domain\\
            $d$ & a distance function on $\mathbb{U}$ with $d: \mathbb{U} \times \mathbb{U} \to \mathbb{R}$\\
            $Q$ & a time series over the set $\mathbb{U}$ of size $l$ with
                $Q = (q_1, q_2, \dots, q_i, \dots, q_l), q_i \in \mathbb{U}$\\
            $Q[i,j]$ & a subsequence time series of $Q$ over the set $\mathbb{U}$ with $Q[i,j] = (q_i, q_{i+1}, \dots, q_{j})$\\
            $\mu$ & the mean of a time series $Q$ over the set $\mathbb{U}$, $\bar{q} \in \mathbb{U}$\\
            $\sigma$ & the standard deviation of a time series $Q$ over the set $\mathbb{U}$, $\sigma \in \mathbb{R}$\\
            $\eta$, $z$  & two different time series normalizations\\
            \hline
        \end{tabularx}
    \end{center}
    \caption{Notation used for formal problem description.}
	\label{tab:notation}
\end{table}

\subsection{Sliding Window Technique} \label{sliding_window_technique}
Given is a continuous time series data stream $Q$. The last subsequence of size $w$ is in the interest of the sliding
window technique. All subsequences or time series data points of $Q$ that are back even further as the window size $w$
are in this moment irrelevant. The sliding window application tries to classify this last subsequence of $Q$ with the
help of a time series classificator. This time series classificator makes its decisions based on a given time series
training set. The application triggers an event in the case of a positive classification of the subsequence and an other
application can react on this event.

The sliding window application remains idle until the time series data stream $Q$
grows. Afterwards, the process is repeated again. However, the sliding window application waits until the time series
data stream $Q$ has grown by a predefined step size $s$. If necessary, this stepwise approach can avoid the execution of
the process continuously after every new data point in $Q$. Figure \ref{fig:swt} illustrates the described process.

\tikzstyle{decision} = [diamond, draw, aspect=2, fill=white!20, text width=8em, text badly centered, node distance=2cm, inner sep=0pt]
\tikzstyle{block} = [rectangle, draw, fill=white!20, text width=5em, text centered, minimum height=4em]
\tikzstyle{line} = [draw, -latex']

\begin{figure}
    \begin{center}
        \resizebox {\textwidth} {!} {
            {\tiny
                \begin{tikzpicture}[node distance = 1.5cm, auto]
                    \node [block] (sod) {sensors or devices};
                    \node [block, right of=sod, node distance=6cm, text width=2cm] (extract) {Extract last subsequence from Q of size $w$, $Q[t-w,t]$};
                    \node [block, right of=extract, node distance=4cm, text width=2cm] (nnc) {Time series classificator};
                    \node [decision, below of=nnc] (decide) {$Q[t-w,t]$ classifiable?};
                    \node [block, left of=decide, node distance=3cm] (sleeps) {Sleep for $s$ time};
                    \node [block, below of=decide, node distance=2cm, text width=2cm] (action) {Trigger event that $Q[t-w,t]$ has been classified and sleep for $w$ time};

                    \path [line,dashed] (sod) -- node (ctss) {Continuous time series stream $Q$} (extract);
                    \path [line] (extract) -- node {$Q[t-w,t]$} (nnc);
                    \path [line] (nnc) -- (decide);
                    \path [line] (decide) -- node {no} (sleeps);
                    \path [line,dashed] (sleeps) -| (ctss);
                    \path [line] (decide) -- node {yes} (action);
                    \path [line,dashed] (action) -| (ctss);
                \end{tikzpicture}
            }
        }
    \end{center}
    \caption{Possible design for a sliding window application. The current time is stored in variable $t$. The constant
    variables $w$ for the window size and $s$ for the step size are predefined.}
    \label{fig:swt}
\end{figure}

The explained sliding window technique uses a generic time series classificator. However, in this bachelor thesis the
generic time series classificator is replaced by 1NN-DTW. A time series window $Q[t-w,t]$ is classified as instance of
class $K_i$ by 1NN-DTW if the DTW distance between $Q[t-w,t]$ and the nearest neighbour passes a predefined
class threshold $\epsilon_i$. The following subsection gives some background of DTW.

\subsection{Dynamic Time Warping} \label{dynamic_time_warping}
Dynamic Time Warping (DTW) is a widely used and robust distance measure for time series, \textit{allowing similar shapes
to match even if they are out of phase in the time axis} \cite{keogh2002exact}. The following explaination to calculate
the DTW distance is based on \cite{sart2010accelerating}.

Given are two time series $Q = (q_1, q_2, \dots, q_i, \dots, q_l)$ with length
$l$, $C = (c_1, c_2, \dots, c_j, \dots, c_k)$ with length $k$ over the domain set $\mathbb{U}$ and a distance measure
function $D$ with $D: \mathbb{U} \times \mathbb{U} \to \mathbb{R}$. Calculating the DTW distance between the two time
series $Q$ and $C$ can be achieved by calculating a matrix $M$ of size $l \times k$ with the following rule.
\begin{center} \[ M_{i, j} = \begin{cases}
    D(q_i,c_j) & \text{if } i = 1 \wedge j = 1\\
    M_{i,j-1} + D(q_i,c_j) & \text{if } i = 1 \wedge j \neq 1\\
    M_{i-1,j} + D(q_i,c_j) & \text{if } i \neq 1 \wedge j = 1\\
    min(M_{i-1,j}, M_{i-1,j-1}, M_{i,j-1}) + D(q_i,c_j) & \text{if } i \neq 1 \wedge j \neq 1
\end{cases} \] \end{center}
The DTW distance between the two time series $Q$ and $C$ is the entry $M_{l,k}$ of the resulting matrix.
\begin{center}
    $DTW(Q, C) = M_{l,k}$
\end{center}
The detection of the warping path as result of the backtracking is irrelevant for the aim of this bachelor thesis.
Figure \ref{fig:dynamictimewarping} illustrates DTW for two time series $Q$ and $C$ that contain recorded data from one
acceleration sensor. DTW shown as above has time and space complexity of $\mathcal{O}(lk)$. When ignoring the
warping path as result the algorithm can easely reduce the space complexity to $\mathcal{O}(min(l, k))$. This can be
achieved by keeping only the last important enties in space that are necessary to calculate the final entriy $M_{l,k}$
of the matrix.

\begin{figure}
    \begin{center}
        \begin{tabular}{cc}
            \resizebox {0.57\textwidth} {!} {
                \begin{tikzpicture}
                    \begin{axis}[
                        xmin=0,
                        xmax=47,
                        xlabel=time,
                        ylabel=acceleration,
                        width=\textwidth,
                        height=\axisdefaultheight,
                        reverse legend]
                        \addplot[lightgray, quiver={u=\thisrow{u}, v=\thisrow{v}}] table {../data/fig/dynamictimewarping/path.dat};
                        \addplot[red, thick, mark=none] table {../data/fig/dynamictimewarping/q.dat};
                        \addlegendentry{Q}
                        \addplot[blue, thick, mark=none] table {../data/fig/dynamictimewarping/c.dat};
                        \addlegendentry{C}
                    \end{axis}
                \end{tikzpicture}
            } & \quad
            \resizebox {0.33\textwidth} {!} {
                \begin{tabular}{ll}
                    &
                    \\[-1.55\textwidth]
                    \begin{turn}{90}
                        \begin{tikzpicture}
                            \begin{axis}[
                                xmin=0,
                                xmax=47,
                                ymin=-100,
                                ymax=0,
                                hide x axis,
                                hide y axis,
                                width=\textwidth,
                                height=\axisdefaultheight]
                                \addplot[red, ultra thick, mark=none] table {../data/fig/dynamictimewarping/q.dat};
                            \end{axis}
                        \end{tikzpicture}
                    \end{turn} \hspace*{3em} &
                    \begin{tikzpicture}
                        \begin{axis}[
                            enlargelimits=false,
                            ymin=0,
                            ymax=47,
                            hide x axis,
                            hide y axis,
                            width=\textwidth,
                            height=\textwidth,
                            colorbar,
                            colormap/viridis high res]
                            \addplot[matrix plot*,
                                mesh/cols=48,
                                point meta=explicit] table[meta=C] {../data/fig/dynamictimewarping/matrix.dat};
                            \addplot[white, ultra thick, mark=*] table {../data/fig/dynamictimewarping/matrix_path.dat};
                        \end{axis}
                    \end{tikzpicture}\\
                    &
                    \\[1em]
                    &
                    \begin{tikzpicture}
                        \begin{axis}[
                            xmin=0,
                            xmax=47,
                            ymin=-100,
                            ymax=0,
                            hide x axis,
                            hide y axis,
                            width=\textwidth,
                            height=\axisdefaultheight]
                            \addplot[blue, ultra thick, mark=none] table {../data/fig/dynamictimewarping/c.dat};
                        \end{axis}
                    \end{tikzpicture}
                \end{tabular}
            }
        \end{tabular}
    \end{center}
    \caption{Two time series $Q$ and $C$ containing recorded and compressed data from one acceleration sensor. On the
    left plot both time series graphs, the gray lines are representing the warping path of plain DTW. The right plot
    shows the associated matrix containing the distances between the time series data points. Starting in the lower left
    corner and ending in the upper right corner, the warping path is illustrated as white graph.}
    \label{fig:dynamictimewarping}
\end{figure}

\subsubsection{Sakoe-Chiba Band} \label{sakoe-chiba_band}
\cite{sakoe1978dynamic}

\subsection{Time Series Normalization} \label{time_series_normalization}
\cite{keogh2003need}


