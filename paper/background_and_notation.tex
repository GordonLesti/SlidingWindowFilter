\section{Background and Notation} \label{background_and_notation}

This section gives more background on the sliding window technique \cite{keogh2004sliding}, DTW distance measure \cite{keogh2002exact}, and time series normalization \cite{das1998rule}, which are fundamental building blocks of our conducted online gesture recognition study \cite{lesti2017filter}. Table \ref{tab:notation} introduces the notation that we use for formal problem description.

\begin{table}
    \begin{center}
        \begin{tabularx}{\textwidth}{c X}
            \hline
            \textbf{Symbol} \qquad & \textbf{Description}\\
            \hline
            $\mathbb{U}$ & a set containing all items of a domain\\
            $d$ & a distance function on $\mathbb{U}$ with $d: \mathbb{U} \times \mathbb{U} \to \mathbb{R}$\\
            $Q$ & a time series over the set $\mathbb{U}$ of size $l$ with
                $Q = (q_1, q_2, \dots, q_i, \dots, q_l), q_i \in \mathbb{U}$\\
            $Q[i,j]$ & a subsequence time series of $Q$ over the set $\mathbb{U}$ with
                $Q[i,j] = (q_i, q_{i+1}, \dots, q_{j})$\\
            $\bar{q}$ & the mean of a time series $Q$ over the set $\mathbb{U}$, $\bar{q} \in \mathbb{U}$\\
            $\sigma$ & the standard deviation of a time series $Q$ over the set $\mathbb{U}$, $\sigma \in \mathbb{R}$\\
            $\eta$, $\eta '$  & two different time series normalizations\\
            \hline
        \end{tabularx}
    \end{center}
    \caption{Notation used for formal problem description.}
	\label{tab:notation}
\end{table}

\subsection{Sliding Window Technique} \label{sliding_window_technique}
Given is a continuous time series data stream $Q$. The last subsequence of size $w$ is in the interest of the sliding
window technique. All subsequences or time series data points of $Q$ that are back even further as the window size $w$
are in this moment irrelevant. The sliding window application tries to classify this last subsequence of $Q$ with the
help of a time series classificator. This time series classificator makes its decisions based on a given time series
training set. The application triggers an event in the case of a positive classification of the subsequence and an other
application can react on this event.

The sliding window application remains idle until the time series data stream $Q$
grows. Afterwards, the process is repeated again. However, the sliding window application waits until the time series
data stream $Q$ has grown by a predefined step size $s$. If necessary, this stepwise approach can avoid the execution of
the process continuously after every new data point in $Q$.

The explained sliding window technique uses a generic time series classificator. However, in this paper the
generic time series classificator is replaced by 1NN-DTW. A time series window $Q[t-w,t]$ is classified as instance of
class $K_i$ by 1NN-DTW if the DTW distance between $Q[t-w,t]$ and the nearest neighbour passes a predefined
class threshold $\epsilon_i$.

\subsubsection{Dynamic Time Warping}
Dynamic time warping or just shorten DTW is a well known algorithm for pattern detection in time series
\cite{berndt1994using}. The following short description of the algorithm will only focus on the distance calculation and
not on the backtracking. The warping path as result of the backtracking is irrelevant for the aim of this bachelor
thesis.

Given are a time series $S$ of size $n$ and a time series $T$ of size $m$ over the set $U$. Furthermore is given a
distance measure function $D$ on set $\mathbb{U}$ with $D: \mathbb{U} \times \mathbb{U} \to \mathbb{R}$. DTW creates a
$n \times m$ grid $M$ with the the following rule.
\begin{center} \[ M_{i, j} = \begin{cases}
    D(s_i,t_j) & \text{if } i = 1 \wedge j = 1\\
    M_{i,j-1} + D(s_i,t_j) & \text{if } i = 1 \wedge j \neq 1\\
    M_{i-1,j} + D(s_i,t_j) & \text{if } i \neq 1 \wedge j = 1\\
    min(M_{i-1,j}, M_{i-1,j-1}, M_{i,j-1}) + D(s_i,t_j) & \text{if } i \neq 1 \wedge j \neq 1
\end{cases} \] \end{center}
The resulting distance between the two given time series is the entry $M_{n,m}$ of the grid.
\begin{center}
    $DTW(S, T) = M_{n,m}$
\end{center}

