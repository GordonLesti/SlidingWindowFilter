\subsection{Time Series Normalization} \label{time_series_normalization}

Literature on time series mining \cite{ding2008querying,spiegel2015diss} suggests to normalize all (sub-)sequences before measuring their pair-wise dissimilarity by means of a distance measure. There are multiple ways to normalize time series, where two common techniques \cite{das1998rule} are compared  in this study.

Given is a time series $Q = (q_1, \dots, q_n)$ of length $n$, we can compute its mean $\mu$ and standard deviation $\sigma$ as followed:

\begin{equation*}
\mu = \frac{1}{n} \sum_{i=1}^{n} q_i 	\;\;\;\;\;\;\;\;\;\;\;\;\;\;\;\; 	\sigma = \frac{1}{n} \sum_{i=1}^{n} (q_i - \mu)^2
\end{equation*}

Having defined the mean $\mu$ and standard deviation $\sigma$, we can normalize each data point $q_i$ (with $1\leq i \leq n$) of a time series $Q = (q_1, \dots, q_n)$ in one of the two following ways \cite{das1998rule}:

\begin{equation}
\eta(q_i) = q_i - \mu
\end{equation}

\begin{equation}
z(q_i) = \frac{q_i - \mu}{\sigma}
\label{eqn:zscore}
\end{equation}

Equation \ref{eqn:zscore} is commonly known as the Z-score. For the sake of simplicity we refer to $\eta$ and $z$ normalization \cite{das1998rule} for the rest of the paper.


