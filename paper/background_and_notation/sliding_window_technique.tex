\subsection{Sliding Window Technique} \label{sliding_window_technique}

Given a continuous time series stream $Q$, the sliding window technique examines the $w$ most recent data points and moves $s$ steps along the time axis as new measurements arrive, where $w$ and $s$ are referred to as window and step size. This technique has the advantage that it does not need to store the never-ending stream of data, but it also implies that measurements can only be considered for further data analysis as long as they are located within the current window. 

In most applications, each window is passed to a data processing unit, which performs some kind of time series classification, clustering, or anomaly detection. For example in online gesture recognition \cite{lesti2017filter}, one can employ a nearest neighbor classifier, which compares each window to a training set of preliminary learned time series instances. In case that the current window $Q[t-w,t]$ is similar to one of the known gestures, where similar means that the time series distance falls below a certain threshold, a corresponding action can be triggered. A popular distance measure for gestures \cite{liu2009uwave} and other warped time series is described in the following subsection.
