\subsection{Sliding Window Technique} \label{sliding_window_technique}

Given a continuous time series stream $Q$, the sliding window technique examines the $w$ most recent data points and moves $s$ steps along the time axis as new measurements arrive, where $w$ and $s$ are referred to as window and step size. This technique has the advantage that it does not need to store the never-ending stream of data, but it also implies that measurements can only be considered for further data analysis as long as they are located within the current window. 

In most applications, each window is passed to a data processing unit, which performs some kind of time series classification, clustering, or anomaly detection. For example in online gesture recognition, one could employ a nearest neighbor classifier that compares each window to a training set of preliminary learned time series instances. In case that a time series segment was classified as   

...

The sliding window application tries to classify this last subsequence of $Q$ with the
help of a time series classificator. This time series classificator makes its decisions based on a given time series
training set. The application triggers an event in the case of a positive classification of the subsequence and an other
application can react on this event.

The sliding window application remains idle until the time series data stream $Q$
grows. Afterwards, the process is repeated again. However, the sliding window application waits until the time series
data stream $Q$ has grown by a predefined step size $s$. If necessary, this stepwise approach can avoid the execution of
the process continuously after every new data point in $Q$.

The explained sliding window technique uses a generic time series classificator. However, in this paper the
generic time series classificator is replaced by 1NN-DTW. A time series window $Q[t-w,t]$ is classified as instance of
class $K_i$ by 1NN-DTW if the DTW distance between $Q[t-w,t]$ and the nearest neighbour passes a predefined
class threshold $\epsilon_i$.
