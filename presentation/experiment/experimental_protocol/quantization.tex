\subsubsection{Quantization}

\begin{frame}{Quantization}
    Quantization based on \textit{uWave} \cite{liu2009uwave}
    \pause
    \begin{block}{Compressing}
        \begin{itemize}
            \item Recorded acceleration data were compressed to an average value for a window size of 50
                ms and a step length of 30 ms
        \end{itemize}
    \end{block}
    \pause
    \begin{block}{Conversion}
        \begin{itemize}
            \item The compressed records are converted into 33 different levels
            \begin{center}
                \tiny
                \begin{tabular}{ll}
                    \textbf{Acceleration data ($a$) in $\frac{dm}{s^2}$} & \textbf{Converted value}\\
                    \hline
                    $a > 200$ & 16\\
                    $100 < a < 200$ & 11 to 15 (five levels linearly)\\
                    $0 < a < 100$ & 1 to 10 (ten levels linearly)\\
                    $a = 0$ & 0\\
                    $-100 < a < 0$ & -1 to - 10 (ten levels linearly)\\
                    $-200 < a < -100$ & -11 to - 15 (five levels linearly)\\
                    $a < -200$ & -16
                \end{tabular}
            \end{center}
        \end{itemize}
    \end{block}
\end{frame}

\begin{frame}{Quantization}{Example - raw acceleration data}
    \begin{center}
        \begin{tikzpicture}
            \pgfplotsset{every axis legend/.append style={
        		at={(0.5,1.03)},
        		anchor=south}}
            \begin{axis}[
                xmin=1,
                xmax=295,
                xlabel=time,
                ylabel=acceleration in $\frac{dm}{s^2}$,
                legend columns=4]
                \addplot[blue, ultra thick, mark=none] table[x=t, y=x] {../data/fig/quantization/raw.dat};
                \addlegendentry{x-axis}
                \addplot[red, ultra thick, mark=none] table[x=t, y=y] {../data/fig/quantization/raw.dat};
                \addlegendentry{y-axis}
                \addplot[green, ultra thick, mark=none] table[x=t, y=z] {../data/fig/quantization/raw.dat};
                \addlegendentry{z-axis}
            \end{axis}
        \end{tikzpicture}
    \end{center}
\end{frame}

\begin{frame}{Quantization}{Example - compressed acceleration data}
    \begin{center}
        \begin{tikzpicture}
            \pgfplotsset{every axis legend/.append style={
                at={(0.5,1.03)},
                anchor=south}}
            \begin{axis}[
                xmin=1,
                xmax=52,
                xlabel=time,
                ylabel=acceleration in $\frac{dm}{s^2}$,
                legend columns=4]
                \addplot[blue, ultra thick, mark=none] table[x=t, y=x] {../data/fig/quantization/compressed.dat};
                \addlegendentry{x-axis}
                \addplot[red, ultra thick, mark=none] table[x=t, y=y] {../data/fig/quantization/compressed.dat};
                \addlegendentry{y-axis}
                \addplot[green, ultra thick, mark=none] table[x=t, y=z] {../data/fig/quantization/compressed.dat};
                \addlegendentry{z-axis}
            \end{axis}
        \end{tikzpicture}
    \end{center}
\end{frame}

\begin{frame}{Quantization}{Example - compressed \& converted acceleration data}
    \begin{center}
        \begin{tikzpicture}
            \pgfplotsset{every axis legend/.append style={
                at={(0.5,1.03)},
                anchor=south}}
            \begin{axis}[
                xmin=1,
                xmax=52,
                xlabel=time,
                ylabel=converted acceleration,
                legend columns=4]
                \addplot[blue, ultra thick, mark=none] table[x=t, y=x] {../data/fig/quantization/converted.dat};
                \addlegendentry{x-axis}
                \addplot[red, ultra thick, mark=none] table[x=t, y=y] {../data/fig/quantization/converted.dat};
                \addlegendentry{y-axis}
                \addplot[green, ultra thick, mark=none] table[x=t, y=z] {../data/fig/quantization/converted.dat};
                \addlegendentry{z-axis}
            \end{axis}
        \end{tikzpicture}
    \end{center}
\end{frame}
