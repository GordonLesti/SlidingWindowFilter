\section{Background and Notation} \label{background_and_notation}
A time series $S$ with the length $l$ over the domain set $\mathbb{U}$ is a sequence of data points
$S = (s_1, s_2, \dots, s_i, \dots, s_l)$ with $s_i \in \mathbb{U}$. On the set $\mathbb{U}$ exists a distance measure
function $D$ with $D: \mathbb{U} \times \mathbb{U} \to \mathbb{R}$. A common assumption for time series is that the
containing data points are elements of the integer numbers $\mathbb{Z}$ or the real numbers $\mathbb{R}$. The distance
measure function on these common domains is just the absolute value of the difference of two elements. A fundamental
prerequisite for distance measure function on time series is often the distance measure function $D$ on the set
$\mathbb{U}$. Table~\ref{tab:notation} explains the basic variables that are used in this bachelor thesis.

\begin{table}[H]
    \begin{center}
        \begin{tabularx}{\textwidth}{c X}
            \textbf{Symbol} & \textbf{Description}\\
            \hline
            $\mathbb{U}$ & a set containing all items of a domain\\
            $D$ & a distance measure function on $\mathbb{U}$ with $D: \mathbb{U} \times \mathbb{U} \to \mathbb{R}$\\
            $S$ & a time series over the set $\mathbb{U}$ of size $l$ with
                $S = (s_1, s_2, \dots, s_i, \dots, s_l), s_i \in \mathbb{U}$\\
            $DTW$ & dynamic time warping, a distance measure function for time series\\
            $ED$ & euclidean distance, a distance measure function for time series
        \end{tabularx}
    \end{center}
    \caption{Basic variables that are used in this bachelor thesis.}
	\label{tab:notation}
\end{table}

\subsubsection{Euclidean Distance}
A basic distance measure function on time series is the euclidean distance or shorten $ED$.
\begin{center}
    $ED(S, T) = \sqrt[2]{\sum \limits_{i=1}^{n} D(s_i, t_i)^2}$
\end{center}
Two time series of the the same length are required for the euclidean distance. The smaller time series has to be
streched or the longer time series has to be compressed.

\subsection{Dynamic Time Warping} \label{dynamic_time_warping}
Dynamic Time Warping (DTW) is a widely used and robust distance measure for time series, \textit{allowing similar shapes
to match even if they are out of phase in the time axis} \cite{keogh2002exact}. The following explaination to calculate
the DTW distance is based on \cite{sart2010accelerating}.

Given are two time series $Q = (q_1, q_2, \dots, q_i, \dots, q_l)$ with length
$l$, $C = (c_1, c_2, \dots, c_j, \dots, c_k)$ with length $k$ over the domain set $\mathbb{U}$ and a distance measure
function $D$ with $D: \mathbb{U} \times \mathbb{U} \to \mathbb{R}$. Calculating the DTW distance between the two time
series $Q$ and $C$ can be achieved by calculating a matrix $M$ of size $l \times k$ with the following rule.
\begin{center} \[ M_{i, j} = \begin{cases}
    D(q_i,c_j) & \text{if } i = 1 \wedge j = 1\\
    M_{i,j-1} + D(q_i,c_j) & \text{if } i = 1 \wedge j \neq 1\\
    M_{i-1,j} + D(q_i,c_j) & \text{if } i \neq 1 \wedge j = 1\\
    min(M_{i-1,j}, M_{i-1,j-1}, M_{i,j-1}) + D(q_i,c_j) & \text{if } i \neq 1 \wedge j \neq 1
\end{cases} \] \end{center}
The DTW distance between the two time series $Q$ and $C$ is the entry $M_{l,k}$ of the resulting matrix.
\begin{center}
    $DTW(Q, C) = M_{l,k}$
\end{center}
The detection of the warping path as result of the backtracking is irrelevant for the aim of this bachelor thesis.
Figure \ref{fig:dynamictimewarping} illustrates DTW for two time series $Q$ and $C$ that contain recorded data from one
acceleration sensor. DTW shown as above has time and space complexity of $\mathcal{O}(lk)$. When ignoring the
warping path as result the algorithm can easely reduce the space complexity to $\mathcal{O}(min(l, k))$. This can be
achieved by keeping only the last important enties in space that are necessary to calculate the final entriy $M_{l,k}$
of the matrix.

\begin{figure}
    \begin{center}
        \begin{tabular}{cc}
            \resizebox {0.57\textwidth} {!} {
                \begin{tikzpicture}
                    \begin{axis}[
                        xmin=0,
                        xmax=47,
                        xlabel=time,
                        ylabel=acceleration,
                        width=\textwidth,
                        height=\axisdefaultheight,
                        reverse legend]
                        \addplot[lightgray, quiver={u=\thisrow{u}, v=\thisrow{v}}] table {../data/fig/dynamictimewarping/path.dat};
                        \addplot[red, thick, mark=none] table {../data/fig/dynamictimewarping/q.dat};
                        \addlegendentry{Q}
                        \addplot[blue, thick, mark=none] table {../data/fig/dynamictimewarping/c.dat};
                        \addlegendentry{C}
                    \end{axis}
                \end{tikzpicture}
            } & \quad
            \resizebox {0.33\textwidth} {!} {
                \begin{tabular}{ll}
                    &
                    \\[-1.55\textwidth]
                    \begin{turn}{90}
                        \begin{tikzpicture}
                            \begin{axis}[
                                xmin=0,
                                xmax=47,
                                ymin=-100,
                                ymax=0,
                                hide x axis,
                                hide y axis,
                                width=\textwidth,
                                height=\axisdefaultheight]
                                \addplot[red, ultra thick, mark=none] table {../data/fig/dynamictimewarping/q.dat};
                            \end{axis}
                        \end{tikzpicture}
                    \end{turn} \hspace*{3em} &
                    \begin{tikzpicture}
                        \begin{axis}[
                            enlargelimits=false,
                            ymin=0,
                            ymax=47,
                            hide x axis,
                            hide y axis,
                            width=\textwidth,
                            height=\textwidth,
                            colorbar,
                            colormap/viridis high res]
                            \addplot[matrix plot*,
                                mesh/cols=48,
                                point meta=explicit] table[meta=C] {../data/fig/dynamictimewarping/matrix.dat};
                            \addplot[white, ultra thick, mark=*] table {../data/fig/dynamictimewarping/matrix_path.dat};
                        \end{axis}
                    \end{tikzpicture}\\
                    &
                    \\[1em]
                    &
                    \begin{tikzpicture}
                        \begin{axis}[
                            xmin=0,
                            xmax=47,
                            ymin=-100,
                            ymax=0,
                            hide x axis,
                            hide y axis,
                            width=\textwidth,
                            height=\axisdefaultheight]
                            \addplot[blue, ultra thick, mark=none] table {../data/fig/dynamictimewarping/c.dat};
                        \end{axis}
                    \end{tikzpicture}
                \end{tabular}
            }
        \end{tabular}
    \end{center}
    \caption{Two time series $Q$ and $C$ containing recorded and compressed data from one acceleration sensor. On the
    left plot both time series graphs, the gray lines are representing the warping path of plain DTW. The right plot
    shows the associated matrix containing the distances between the time series data points. Starting in the lower left
    corner and ending in the upper right corner, the warping path is illustrated as white graph.}
    \label{fig:dynamictimewarping}
\end{figure}

\subsubsection{Sakoe-Chiba Band} \label{sakoe-chiba_band}
\cite{sakoe1978dynamic}

\subsection{Time Series Normalization} \label{time_series_normalization}
\cite{keogh2003need}


\subsection{Complexity-Invariant Distance Measure}
The complexity-invariant distance measure \cite{batista2011complexity} or CID is a distance measure for time series.

