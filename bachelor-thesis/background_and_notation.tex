\section{Background and Notation} \label{background_and_notation}
A time series $Q$ with the length $l$ over the domain set $\mathbb{U}$ is a sequence of data points
$Q = (q_1, q_2, \dots, q_i, \dots, q_l)$ with $q_i \in \mathbb{U}$. On the set $\mathbb{U}$ exists a distance measure
function $d$ with $d: \mathbb{U} \times \mathbb{U} \to \mathbb{R}$. A common assumption for time series is that the
containing data points are elements of the integer numbers $\mathbb{Z}$ or the real numbers $\mathbb{R}$. The distance
measure function on these common domains is just the absolute value of the difference of two elements. A fundamental
prerequisite for similarity measures on time series is often the distance measure function $d$ on the set
$\mathbb{U}$. Table~\ref{tab:notation} explains the basic notation that is used in this bachelor thesis.

\begin{table}
    \begin{center}
        \begin{tabularx}{\textwidth}{c X l}
            \hline
            \textbf{Symbol} \qquad & \textbf{Description} & \qquad \textbf{Section}\\
            \hline
            $\mathbb{U}$ & a set containing all items of a domain & \qquad\\
            $d$ & a distance measure function on $\mathbb{U}$ with $d: \mathbb{U} \times \mathbb{U} \to \mathbb{R}$
                & \qquad\\
            $Q$ & a time series over the set $\mathbb{U}$ of size $l$ with
                $Q = (q_1, q_2, \dots, q_i, \dots, q_l), q_i \in \mathbb{U}$ & \qquad\\
            $Q[i,j]$ & a subsequence time series of $Q$ over the set $\mathbb{U}$ with
                $Q[i,j] = (q_i, q_{i+1}, \dots, q_{j})$ & \qquad\\
            $t$ & the current time & \qquad\\
            $w$ & time series window size & \qquad \ref{sliding_window_technique}\\
            $s$ & time series step size & \qquad \ref{sliding_window_technique}\\
            DTW & Dynamic Time Warping, a similarity measure for time series & \qquad \ref{dynamic_time_warping}\\
            $\eta$, $\eta '$  & two different time series normalizations & \qquad \ref{time_series_normalization}\\
            CE & Complexity Estimate & \qquad \ref{complexity_estimate}\\
            LNCE & Length normalized Complexity Estimate & \qquad \ref{sliding_window_filter_measures}\\
            VAR & Sample Variance & \qquad \ref{sliding_window_filter_measures}\\
            \hline
        \end{tabularx}
    \end{center}
    \caption{Basic notation.}
	\label{tab:notation}
\end{table}

\subsection{Sliding Window Technique} \label{sliding_window_technique}
Given is a continuous time series data stream $Q$. The last subsequence of size $w$ is in the interest of the sliding
window technique. All subsequences or time series data points of $Q$ that are back even further as the window size $w$
are in this moment irrelevant. The sliding window application tries to classify this last subsequence of $Q$ with the
help of a time series classificator. This time series classificator makes its decisions based on a given time series
training set. The application triggers an event in the case of a positive classification of the subsequence and an other
application can react on this event.

The sliding window application remains idle until the time series data stream $Q$
grows. Afterwards, the process is repeated again. However, the sliding window application waits until the time series
data stream $Q$ has grown by a predefined step size $s$. If necessary, this stepwise approach can avoid the execution of
the process continuously after every new data point in $Q$.

The explained sliding window technique uses a generic time series classificator. However, in this paper the
generic time series classificator is replaced by 1NN-DTW. A time series window $Q[t-w,t]$ is classified as instance of
class $K_i$ by 1NN-DTW if the DTW distance between $Q[t-w,t]$ and the nearest neighbour passes a predefined
class threshold $\epsilon_i$.

\subsubsection{Dynamic Time Warping}
Dynamic time warping or just shorten DTW is a well known algorithm for pattern detection in time series
\cite{berndt1994using}. The following short description of the algorithm will only focus on the distance calculation and
not on the backtracking. The warping path as result of the backtracking is irrelevant for the aim of this bachelor
thesis.

Given are a time series $S$ of size $n$ and a time series $T$ of size $m$ over the set $U$. Furthermore is given a
distance measure function $D$ on set $\mathbb{U}$ with $D: \mathbb{U} \times \mathbb{U} \to \mathbb{R}$. DTW creates a
$n \times m$ grid $M$ with the the following rule.
\begin{center} \[ M_{i, j} = \begin{cases}
    D(s_i,t_j) & \text{if } i = 1 \wedge j = 1\\
    M_{i,j-1} + D(s_i,t_j) & \text{if } i = 1 \wedge j \neq 1\\
    M_{i-1,j} + D(s_i,t_j) & \text{if } i \neq 1 \wedge j = 1\\
    min(M_{i-1,j}, M_{i-1,j-1}, M_{i,j-1}) + D(s_i,t_j) & \text{if } i \neq 1 \wedge j \neq 1
\end{cases} \] \end{center}
The resulting distance between the two given time series is the entry $M_{n,m}$ of the grid.
\begin{center}
    $DTW(S, T) = M_{n,m}$
\end{center}

\subsection{Sliding Window Filter}

\begin{frame}<handout:0>{Sliding Window Filter}{Workflow}
    \begin{center}
        \resizebox {\textwidth} {!} {
            {\tiny
                \begin{tikzpicture}[node distance = 1.5cm, auto]
                    \node [block] (sod) {sensors or devices};
                    \node [block, right of=sod, node distance=6cm, text width=2cm] (extract) {Extract last subsequence from Q of size $w$, $Q[t-w,t]$};
                    \node [block, draw=blue, right of=extract, node distance=4cm, text width=2cm, color=white] (filter) {Time series filter};
                    \node [decision, draw=blue, below of=filter, node distance=1.5cm, color=white] (filterdecide) {$Q[t-w,t]$ can pass?};
                    \node [block, below of=filterdecide, node distance=1.5cm, text width=2cm, color=white] (nnc) {Time series classificator};
                    \node [decision, below of=nnc, node distance=1.5cm, color=white] (decide) {$Q[t-w,t]$ classifiable?};
                    \node [block, left of=decide, node distance=3cm, color=white] (sleeps) {Sleep for $s$ time};
                    \node [block, below of=decide, node distance=1.5cm, text width=2cm, color=white] (action) {Trigger event that $Q[t-w,t]$ has been classified and sleep for $w$ time};

                    \path [line,dashed] (sod) -- node (ctss) {Continuous time series stream $Q$} (extract);
                    \path [line] (extract) -- node {$Q[t-w,t]$} (filter);
                \end{tikzpicture}
            }
        }
    \end{center}
\end{frame}

\begin{frame}<handout:0>{Sliding Window Filter}{Workflow}
    \begin{center}
        \resizebox {\textwidth} {!} {
            {\tiny
                \begin{tikzpicture}[node distance = 1.5cm, auto]
                    \node [block] (sod) {sensors or devices};
                    \node [block, right of=sod, node distance=6cm, text width=2cm] (extract) {Extract last subsequence from Q of size $w$, $Q[t-w,t]$};
                    \node [block, draw=blue, right of=extract, node distance=4cm, text width=2cm] (filter) {Time series filter};
                    \node [decision, draw=blue, below of=filter, node distance=1.5cm] (filterdecide) {$Q[t-w,t]$ can pass?};
                    \node [block, below of=filterdecide, node distance=1.5cm, text width=2cm, color=white] (nnc) {Time series classificator};
                    \node [decision, below of=nnc, node distance=1.5cm, color=white] (decide) {$Q[t-w,t]$ classifiable?};
                    \node [block, left of=decide, node distance=3cm, color=white] (sleeps) {Sleep for $s$ time};
                    \node [block, below of=decide, node distance=1.5cm, text width=2cm, color=white] (action) {Trigger event that $Q[t-w,t]$ has been classified and sleep for $w$ time};

                    \path [line,dashed] (sod) -- node (ctss) {Continuous time series stream $Q$} (extract);
                    \path [line] (extract) -- node {$Q[t-w,t]$} (filter);
                    \path [line] (filter) -- (filterdecide);
                \end{tikzpicture}
            }
        }
    \end{center}
\end{frame}

\begin{frame}<handout:0>{Sliding Window Filter}{Workflow}
    \begin{center}
        \resizebox {\textwidth} {!} {
            {\tiny
                \begin{tikzpicture}[node distance = 1.5cm, auto]
                    \node [block] (sod) {sensors or devices};
                    \node [block, right of=sod, node distance=6cm, text width=2cm] (extract) {Extract last subsequence from Q of size $w$, $Q[t-w,t]$};
                    \node [block, draw=blue, right of=extract, node distance=4cm, text width=2cm] (filter) {Time series filter};
                    \node [decision, draw=blue, below of=filter, node distance=1.5cm] (filterdecide) {$Q[t-w,t]$ can pass?};
                    \node [block, below of=filterdecide, node distance=1.5cm, text width=2cm, color=white] (nnc) {Time series classificator};
                    \node [decision, below of=nnc, node distance=1.5cm, color=white] (decide) {$Q[t-w,t]$ classifiable?};
                    \node [block, left of=decide, node distance=3cm] (sleeps) {Sleep for $s$ time};
                    \node [block, below of=decide, node distance=1.5cm, text width=2cm, color=white] (action) {Trigger event that $Q[t-w,t]$ has been classified and sleep for $w$ time};

                    \path [line,dashed] (sod) -- node (ctss) {Continuous time series stream $Q$} (extract);
                    \path [line] (extract) -- node {$Q[t-w,t]$} (filter);
                    \path [line] (filter) -- (filterdecide);
                    \path [line] (filterdecide) -| node [near start] {no} (sleeps);
                    \path [line,dashed] (sleeps) -| (ctss);
                \end{tikzpicture}
            }
        }
    \end{center}
\end{frame}

\begin{frame}{Sliding Window Filter}{Workflow}
    \begin{center}
        \resizebox {\textwidth} {!} {
            {\tiny
                \begin{tikzpicture}[node distance = 1.5cm, auto]
                    \node [block] (sod) {sensors or devices};
                    \node [block, right of=sod, node distance=6cm, text width=2cm] (extract) {Extract last subsequence from Q of size $w$, $Q[t-w,t]$};
                    \node [block, draw=blue, right of=extract, node distance=4cm, text width=2cm] (filter) {Time series filter};
                    \node [decision, draw=blue, below of=filter, node distance=1.5cm] (filterdecide) {$Q[t-w,t]$ can pass?};
                    \node [block, below of=filterdecide, node distance=1.5cm, text width=2cm] (nnc) {Time series classificator};
                    \node [decision, below of=nnc, node distance=1.5cm] (decide) {$Q[t-w,t]$ classifiable?};
                    \node [block, left of=decide, node distance=3cm] (sleeps) {Sleep for $s$ time};
                    \node [block, below of=decide, node distance=1.5cm, text width=2cm] (action) {Trigger event that $Q[t-w,t]$ has been classified and sleep for $w$ time};

                    \path [line,dashed] (sod) -- node (ctss) {Continuous time series stream $Q$} (extract);
                    \path [line] (extract) -- node {$Q[t-w,t]$} (filter);
                    \path [line] (filter) -- (filterdecide);
                    \path [line] (filterdecide) -- node {yes} (nnc);
                    \path [line] (filterdecide) -| node [near start] {no} (sleeps);
                    \path [line] (nnc) -- (decide);
                    \path [line] (decide) -- node {no} (sleeps);
                    \path [line,dashed] (sleeps) -| (ctss);
                    \path [line] (decide) -- node {yes} (action);
                    \path [line,dashed] (action) -| (ctss);
                \end{tikzpicture}
            }
        }
    \end{center}
\end{frame}

\subsection{Complexity Estimate}

\begin{frame}{Complexity Estimate}
    \begin{itemize}
        \item Invented by Gustavo Enrique de Almeida Prado Alves Batista, Xiaoyue Wang and Eamonn Keogh in 2011
            \cite{batista2011complexity}
        
        \item A possible approach to measure the complexity of a time series
        
        \item Linear complexity
    \end{itemize}
\end{frame}

\begin{frame}{Complexity Estimate}{Calculation}
    \begin{block}{Given}
        \begin{itemize}
            \item A domain set $\mathbb{U}$
            
            \item A distance measure function $d$ with $d: \mathbb{U} \times \mathbb{U} \to \mathbb{R}$
            
        \end{itemize}
    \end{block}
    \begin{block}{Input}
        \begin{itemize}
            \item A time series $Q = (q_1, q_2, \dots, q_i, \dots, q_l)$ with length $l$ over the domain set
                $\mathbb{U}$
        \end{itemize}
    \end{block}
\end{frame}

\begin{frame}{Complexity Estimate}{Calculation}
    \begin{block}{Calculation}
        \begin{itemize}
            \item $CE(Q) = \sqrt[2]{\sum \limits_{i=1}^{l-1} d(q_i, q_{i + 1})^2}$
        \end{itemize}
    \end{block}
\end{frame}

\begin{frame}{Length Normalized Complexity Estimate}{Calculation}
    \begin{block}{Calculation}
        \begin{itemize}
            \item $LNCE(Q) = \frac{1}{l-1}CE(Q)$
        \end{itemize}
    \end{block}
\end{frame}

