\section{Background and Notation} \label{background_and_notation}
A time series $Q$ with the length $l$ over the domain set $\mathbb{U}$ is a sequence of data points
$Q = (q_1, q_2, \dots, q_i, \dots, q_l)$ with $q_i \in \mathbb{U}$. On the set $\mathbb{U}$ exists a distance measure
function $D$ with $D: \mathbb{U} \times \mathbb{U} \to \mathbb{R}$. A common assumption for time series is that the
containing data points are elements of the integer numbers $\mathbb{Z}$ or the real numbers $\mathbb{R}$. The distance
measure function on these common domains is just the absolute value of the difference of two elements. A fundamental
prerequisite for similarity measures on time series is often the distance measure function $D$ on the set
$\mathbb{U}$. Table~\ref{tab:notation} explains the basic notation that is used in this bachelor thesis.

\begin{table}[H]
    \begin{center}
        \begin{tabularx}{\textwidth}{c X l}
            \textbf{Symbol} \qquad & \textbf{Description} & \qquad \textbf{Section}\\
            \hline
            $\mathbb{U}$ & a set containing all items of a domain & \qquad \ref{background_and_notation}\\
            $D$ & a distance measure function on $\mathbb{U}$ with $D: \mathbb{U} \times \mathbb{U} \to \mathbb{R}$
                & \qquad \ref{background_and_notation}\\
            $Q$ & a time series over the set $\mathbb{U}$ of size $l$ with
                $Q = (q_1, q_2, \dots, q_i, \dots, q_l), q_i \in \mathbb{U}$ & \qquad \ref{background_and_notation}\\
            ED & Euclidean Distance, a similarity measure for time series & \qquad \ref{euclidean_distance}\\
            DTW & Dynamic Time Warping, a similarity measure for time series & \qquad \ref{dynamic_time_warping}\\
            CID & Complexity-Invariant Distance, a similarity measure for time series
                & \qquad \ref{complexity-invariant_distance_measure}\\
            CF & Complexity Correction Factor & \qquad \ref{complexity-invariant_distance_measure}\\
            CE & Complexity Estimate & \qquad \ref{complexity-invariant_distance_measure}\\
            CIDDTW & Complexity-Invariant Dynamic Time Warping
                & \qquad \ref{complexity-invariant_dynamic_time_warping}\\
        \end{tabularx}
    \end{center}
    \caption{Basic notation.}
	\label{tab:notation}
\end{table}

\subsection{Euclidean Distance}
The Euclidean Distance (ED) is the most straightforward similarity measure for time series \cite{ding2008querying}.
Given are two time series $Q = (q_1, q_2, \dots, q_i, \dots, q_l)$, $C = (c_1, c_2, \dots, c_j, \dots, c_l)$ with the
same length $l$ over the domain set $\mathbb{U}$ and a distance measure function $D$ with
$D: \mathbb{U} \times \mathbb{U} \to \mathbb{R}$. The ED between those two time series can be calculated with the
following formula.
\begin{center}
    $ED(Q, C) = \sqrt[2]{\sum \limits_{i=1}^{l} D(q_i, c_i)^2}$
\end{center}

\begin{figure}[H]
    \begin{center}
        \begin{tikzpicture}
            \begin{axis}[
                xlabel=time,
                ylabel=acceleration,
                width=\textwidth,
                height=\axisdefaultheight]
                \addplot[blue, mark=none] table[x=t, y=q] {background_and_notation/euclidean_distance/timeseries.dat};
                \addlegendentry{Q}
                \addplot[red, mark=none] table[x=t, y=c] {background_and_notation/euclidean_distance/timeseries.dat};
                \addlegendentry{C}
                \addplot[gray, quiver={v=\thisrow{v}}] table[x=t, y=q] {background_and_notation/euclidean_distance/timeseries.dat};
            \end{axis}
        \end{tikzpicture}
    \end{center}
    \caption{Two time series $Q$ and $C$ containing recorded data from one acceleration sensor. The square root of the
    sum of the gray distance lines illustrate the Euclidean distance between the two time series.}
    \label{fig:euclideandistance}
\end{figure}

An advantage of ED is the linear time and space complexity of $\mathcal{O}(l)$ to calculate the similarity between two
time series. The algorithm itself has even only a space complexity of $\mathcal{O}(1)$ if the space of the input time
series is ignored. Comparing two time series of different length with the ED presupposes to transform the time series
to the same length. That can be achieved by stretch the smaller time series for example. Figure
\ref{fig:euclideandistance} illustrates the Euclidean Distance between a time series $Q$ and $C$ that contain recorded
data from one acceleration sensor.

\subsubsection{Dynamic Time Warping}
Dynamic time warping or just shorten DTW is a well known algorithm for pattern detection in time series
\cite{berndt1994using}. The following short description of the algorithm will only focus on the distance calculation and
not on the backtracking. The warping path as result of the backtracking is irrelevant for the aim of this bachelor
thesis.

Given are a time series $S$ of size $n$ and a time series $T$ of size $m$ over the set $U$. Furthermore is given a
distance measure function $D$ on set $\mathbb{U}$ with $D: \mathbb{U} \times \mathbb{U} \to \mathbb{R}$. DTW creates a
$n \times m$ grid $M$ with the the following rule.
\begin{center} \[ M_{i, j} = \begin{cases}
    D(s_i,t_j) & \text{if } i = 1 \wedge j = 1\\
    M_{i,j-1} + D(s_i,t_j) & \text{if } i = 1 \wedge j \neq 1\\
    M_{i-1,j} + D(s_i,t_j) & \text{if } i \neq 1 \wedge j = 1\\
    min(M_{i-1,j}, M_{i-1,j-1}, M_{i,j-1}) + D(s_i,t_j) & \text{if } i \neq 1 \wedge j \neq 1
\end{cases} \] \end{center}
The resulting distance between the two given time series is the entry $M_{n,m}$ of the grid.
\begin{center}
    $DTW(S, T) = M_{n,m}$
\end{center}

\subsection{Complexity-Invariant Distance Measure} \label{complexity-invariant_distance_measure}
The Complexity-Invariant Distance measure \cite{batista2011complexity} (CID) is a distance measure for time
series. It is composed of the ED and a Complexity Correction Factor (CF).

% Given are a time series $S$ of size $n$ and a time series $T$ of size $m$ over the set $\mathbb{U}$. Furthermore is
% $n = m$ and a distance measure function $D$ on the set $\mathbb{U}$ with
% $D: \mathbb{U} \times \mathbb{U} \to \mathbb{R}$ is given.
% \begin{center}
%     $CID(S, T) = ED(S, T) \times CF(S, T)$
% \end{center}
% with complexity correction factor
% \begin{center}
%     $CF(S, T) = \frac{max(CE(S), CE(T))}{min(CE(S), CE(T))}$
% \end{center}
% $CE$ is a complexity estimate of a time series and a possible implementation can be
% \begin{center}
%     $CE(S) = \sqrt[2]{\sum \limits_{i=1}^{n-1} D(s_i, s_{i + 1})^2}$
% \end{center}

