\section{Background and Notation} \label{background_and_notation}
A time series $Q$ with the length $l$ over the domain set $\mathbb{U}$ is a sequence of data points
$Q = (q_1, q_2, \dots, q_i, \dots, q_l)$ with $q_i \in \mathbb{U}$. On the set $\mathbb{U}$ exists a distance measure
function $D$ with $D: \mathbb{U} \times \mathbb{U} \to \mathbb{R}$. A common assumption for time series is that the
containing data points are elements of the integer numbers $\mathbb{Z}$ or the real numbers $\mathbb{R}$. The distance
measure function on these common domains is just the absolute value of the difference of two elements. A fundamental
prerequisite for similarity measures on time series is often the distance measure function $D$ on the set
$\mathbb{U}$. Table~\ref{tab:notation} explains the basic notation that is used in this bachelor thesis.

\begin{table}
    \begin{center}
        \begin{tabularx}{\textwidth}{c X l}
            \textbf{Symbol} \qquad & \textbf{Description} & \qquad \textbf{Section}\\
            \hline
            $\mathbb{U}$ & a set containing all items of a domain & \qquad \ref{background_and_notation}\\
            $D$ & a distance measure function on $\mathbb{U}$ with $D: \mathbb{U} \times \mathbb{U} \to \mathbb{R}$
                & \qquad \ref{background_and_notation}\\
            $ADD$ & a summing function on $\mathbb{U}$ with $ADD: \mathbb{U} \times \mathbb{U} \to \mathbb{U}$
                & \qquad \ref{background_and_notation}\\
            $SMULT$ & a scalar multiplication function with $SMULT: \mathbb{R} \times \mathbb{U} \to \mathbb{U}$
                & \qquad \ref{background_and_notation}\\
            $Q$ & a time series over the set $\mathbb{U}$ of size $l$ with
                $Q = (q_1, q_2, \dots, q_i, \dots, q_l), q_i \in \mathbb{U}$ & \qquad \ref{background_and_notation}\\
            DTW & Dynamic Time Warping, a similarity measure for time series & \qquad \ref{dynamic_time_warping}\\
            CE & Complexity Estimate & \qquad \ref{complexity-invariant_distance_measure}\\
            LNCE & Length normalized Complexity Estimate & \qquad \ref{sliding_window_filter_measures}\\
            VAR & Sample Variance & \qquad \ref{sliding_window_filter_measures}\\
        \end{tabularx}
    \end{center}
    \caption{Basic notation.}
	\label{tab:notation}
\end{table}

\subsubsection{Dynamic Time Warping}
Dynamic time warping or just shorten DTW is a well known algorithm for pattern detection in time series
\cite{berndt1994using}. The following short description of the algorithm will only focus on the distance calculation and
not on the backtracking. The warping path as result of the backtracking is irrelevant for the aim of this bachelor
thesis.

Given are a time series $S$ of size $n$ and a time series $T$ of size $m$ over the set $U$. Furthermore is given a
distance measure function $D$ on set $\mathbb{U}$ with $D: \mathbb{U} \times \mathbb{U} \to \mathbb{R}$. DTW creates a
$n \times m$ grid $M$ with the the following rule.
\begin{center} \[ M_{i, j} = \begin{cases}
    D(s_i,t_j) & \text{if } i = 1 \wedge j = 1\\
    M_{i,j-1} + D(s_i,t_j) & \text{if } i = 1 \wedge j \neq 1\\
    M_{i-1,j} + D(s_i,t_j) & \text{if } i \neq 1 \wedge j = 1\\
    min(M_{i-1,j}, M_{i-1,j-1}, M_{i,j-1}) + D(s_i,t_j) & \text{if } i \neq 1 \wedge j \neq 1
\end{cases} \] \end{center}
The resulting distance between the two given time series is the entry $M_{n,m}$ of the grid.
\begin{center}
    $DTW(S, T) = M_{n,m}$
\end{center}

\subsection{Complexity-Invariant Distance Measure} \label{complexity-invariant_distance_measure}
The Complexity-Invariant Distance measure \cite{batista2011complexity} (CID) is a distance measure for time
series. It is composed of the ED and a Complexity Correction Factor (CF).

% Given are a time series $S$ of size $n$ and a time series $T$ of size $m$ over the set $\mathbb{U}$. Furthermore is
% $n = m$ and a distance measure function $D$ on the set $\mathbb{U}$ with
% $D: \mathbb{U} \times \mathbb{U} \to \mathbb{R}$ is given.
% \begin{center}
%     $CID(S, T) = ED(S, T) \times CF(S, T)$
% \end{center}
% with complexity correction factor
% \begin{center}
%     $CF(S, T) = \frac{max(CE(S), CE(T))}{min(CE(S), CE(T))}$
% \end{center}
% $CE$ is a complexity estimate of a time series and a possible implementation can be
% \begin{center}
%     $CE(S) = \sqrt[2]{\sum \limits_{i=1}^{n-1} D(s_i, s_{i + 1})^2}$
% \end{center}

