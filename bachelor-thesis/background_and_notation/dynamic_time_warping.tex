\subsection{Dynamic Time Warping} \label{dynamic_time_warping}
Dynamic Time Warping (DTW) is a widely used and robust distance measure for time series, \textit{allowing similar shapes
to match even if they are out of phase in the time axis} \cite{keogh2002exact}. The following explaination to calculate
the DTW distance is based on \cite{sart2010accelerating}.

Given are two time series $Q = (q_1, q_2, \dots, q_i, \dots, q_l)$ with length
$l$, $C = (c_1, c_2, \dots, c_j, \dots, c_k)$ with length $k$ over the domain set $\mathbb{U}$ and a distance measure
function $d$ with $d: \mathbb{U} \times \mathbb{U} \to \mathbb{R}$. Calculating the DTW distance between the two time
series $Q$ and $C$ can be achieved by calculating a matrix $M$ of size $l \times k$ with the following rule.
\begin{equation}
    M_{i, j} = \begin{cases}
        d(q_i,c_j) & \text{if } i = 1 \wedge j = 1\\
        M_{i,j-1} + d(q_i,c_j) & \text{if } i = 1 \wedge j \neq 1\\
        M_{i-1,j} + d(q_i,c_j) & \text{if } i \neq 1 \wedge j = 1\\
        min(M_{i-1,j}, M_{i-1,j-1}, M_{i,j-1}) + d(q_i,c_j) & \text{if } i \neq 1 \wedge j \neq 1
    \end{cases}
\end{equation}
The DTW distance between the two time series $Q$ and $C$ is the entry $M_{l,k}$ of the resulting matrix.
\begin{equation}
    DTW(Q, C) = M_{l,k}
\end{equation}
The detection of the warping path as result of the backtracking is irrelevant for the aim of this bachelor thesis.
Figure \ref{fig:dynamictimewarping} illustrates DTW for two time series $Q$ and $C$ that contain recorded data from one
acceleration sensor. DTW shown as above has time and space complexity of $\mathcal{O}(lk)$. When ignoring the
warping path as result the algorithm can easely reduce the space complexity to $\mathcal{O}(min(l, k))$. This can be
achieved by keeping only the last important enties in space that are necessary to calculate the final entriy $M_{l,k}$
of the matrix.

\begin{figure}
    \begin{center}
        \resizebox {\textwidth} {!} {
            \begin{tabular}{cc}
                \resizebox* {!} {0.3\textwidth} {
                    \begin{tikzpicture}
                        \begin{axis}[
                            xmin=0,
                            xmax=47,
                            xlabel=time,
                            ylabel=acceleration,
                            width=\axisdefaultwidth,
                            height=0.7*\axisdefaultheight,
                            reverse legend,
                            legend pos=south east]
                            \addplot[gray, quiver={u=\thisrow{u}, v=\thisrow{v}}] table {../data/fig/dynamictimewarping/path.dat};
                            \addplot[red, thick, mark=none] table {../data/fig/dynamictimewarping/q.dat};
                            \addlegendentry{Q}
                            \addplot[blue, thick, mark=none] table {../data/fig/dynamictimewarping/c.dat};
                            \addlegendentry{C}
                        \end{axis}
                    \end{tikzpicture}
                } & \quad
                \resizebox* {!} {0.3\textwidth} {
                    \begin{tabular}[b]{ll}
                        \begin{turn}{90}
                            \begin{tikzpicture}
                                \begin{axis}[
                                    xmin=0,
                                    xmax=47,
                                    ymin=-100,
                                    ymax=0,
                                    hide x axis,
                                    hide y axis,
                                    width=\axisdefaultwidth,
                                    height=0.7*\axisdefaultheight]
                                    \addplot[red, ultra thick, mark=none] table {../data/fig/dynamictimewarping/q.dat};
                                \end{axis}
                            \end{tikzpicture}
                        \end{turn} \hspace*{3em} &
                        \begin{tikzpicture}
                            \begin{axis}[
                                enlargelimits=false,
                                ymin=0,
                                ymax=47,
                                hide x axis,
                                hide y axis,
                                width=\axisdefaultwidth,
                                height=\axisdefaultwidth,
                                colorbar,
                                colormap/viridis high res]
                                \addplot[matrix plot*,
                                    mesh/cols=48,
                                    point meta=explicit] table[meta=C] {../data/fig/dynamictimewarping/matrix.dat};
                                \addplot[white, ultra thick, mark=*, mark size=1] table {../data/fig/dynamictimewarping/matrix_path.dat};
                            \end{axis}
                        \end{tikzpicture} \\
                        &
                        \\[1em]
                        &
                        \begin{tikzpicture}
                            \begin{axis}[
                                xmin=0,
                                xmax=47,
                                ymin=-100,
                                ymax=0,
                                hide x axis,
                                hide y axis,
                                width=\axisdefaultwidth,
                                height=0.7*\axisdefaultheight]
                                \addplot[blue, ultra thick, mark=none] table {../data/fig/dynamictimewarping/c.dat};
                            \end{axis}
                        \end{tikzpicture}
                    \end{tabular}
                }
            \end{tabular}
        }
    \end{center}
    \caption{Two time series $Q$ and $C$ containing recorded and compressed data from one acceleration sensor. On the
    left plot both time series graphs, the gray lines are representing the warping path of plain DTW. The right plot
    shows the associated matrix containing the distances between the time series data points. Starting in the lower left
    corner and ending in the upper right corner, the warping path is illustrated as white graph.}
    \label{fig:dynamictimewarping}
\end{figure}

\subsubsection{Sakoe-Chiba Band} \label{sakoe-chiba_band}
Plain DTW uses a full distance matrix to calculate the similarity between two time series as shown on the right plot of
figure \ref{fig:dynamictimewarping}. This can allow DTW to match a data point at the beginning of a time series to a
data point at the end of the compared time series. To prevent such scenarios, it is a common approach to restrict the
area of the distance matrix. Popular constraints are the Sakoe-Chiba band \cite{sakoe1978dynamic} and the Itakura
parallelogram \cite{itakura1975minimum}. This thesis will focus only on the Sakoe-Chiba band. The adjustment window or
the width of the Sakoe-Chiba band can be scaled in a general context by a percentage factor of the compared time series
length. Figure \ref{fig:scb} clarifies the influence of a 20\% Sakoe-Chiba band on DTW compared to figure
\ref{fig:dynamictimewarping} with plain DTW. The implementation of a constraint can be easily achieved by setting the
distance between two data point to positive infinity if the associated entry in the distance matrix is outside of the
restricted area. The equation \ref{eq:dtw} can be expanded by a dominating $\infty$ case if $i$ and $j$ do not pass the
condition.

\begin{figure}
    \begin{center}
        \resizebox {\textwidth} {!} {
            \begin{tabular}{cc}
                \resizebox* {!} {0.3\textwidth} {
                    \begin{tikzpicture}
                        \begin{axis}[
                            xmin=0,
                            xmax=47,
                            xlabel=time,
                            ylabel=acceleration,
                            width=\axisdefaultwidth,
                            height=0.7*\axisdefaultheight,
                            reverse legend,
                            legend pos=south east]
                            \addplot[gray, quiver={u=\thisrow{u}, v=\thisrow{v}}] table {../data/fig/scb/path.dat};
                            \addplot[red, thick, mark=none] table {../data/fig/scb/q.dat};
                            \addlegendentry{Q}
                            \addplot[blue, thick, mark=none] table {../data/fig/scb/c.dat};
                            \addlegendentry{C}
                        \end{axis}
                    \end{tikzpicture}
                } & \quad
                \resizebox* {!} {0.3\textwidth} {
                    \begin{tabular}[b]{ll}
                        \begin{turn}{90}
                            \begin{tikzpicture}
                                \begin{axis}[
                                    xmin=0,
                                    xmax=47,
                                    ymin=-100,
                                    ymax=0,
                                    hide x axis,
                                    hide y axis,
                                    width=\axisdefaultwidth,
                                    height=0.7*\axisdefaultheight]
                                    \addplot[red, ultra thick, mark=none] table {../data/fig/scb/q.dat};
                                \end{axis}
                            \end{tikzpicture}
                        \end{turn} \hspace*{3em} &
                        \begin{tikzpicture}
                            \begin{axis}[
                                enlargelimits=false,
                                ymin=0,
                                ymax=47,
                                hide x axis,
                                hide y axis,
                                width=\axisdefaultwidth,
                                height=\axisdefaultwidth,
                                colorbar,
                                colormap/viridis high res]
                                \addplot[matrix plot*,
                                    shader=flat,
                                    mesh/cols=48,
                                    point meta=explicit] table[meta=C] {../data/fig/scb/matrix.dat};
                                \addplot[white, ultra thick, mark=*, mark size=1] table {../data/fig/scb/matrix_path.dat};
                            \end{axis}
                        \end{tikzpicture}\\
                        &
                        \\[1em]
                        &
                        \begin{tikzpicture}
                            \begin{axis}[
                                xmin=0,
                                xmax=47,
                                ymin=-100,
                                ymax=0,
                                hide x axis,
                                hide y axis,
                                width=\axisdefaultwidth,
                                height=0.7*\axisdefaultheight]
                                \addplot[blue, ultra thick, mark=none] table {../data/fig/scb/c.dat};
                            \end{axis}
                        \end{tikzpicture}
                    \end{tabular}
                }
            \end{tabular}
        }
    \end{center}
    \caption{The same two time series $Q$ and $C$ from figure \ref{fig:dynamictimewarping}. On the
    left plot are both time series graphs, the gray lines are representing the warping path of DTW with a Sakoe-Chiba band
    of size 20\%. The right plot shows the associated matrix containing the distances between the time series data
    points. Starting in the lower left corner and ending in the upper right corner, the warping path is illustrated as
    a white graph.}
    \label{fig:scb}
\end{figure}

\subsubsection{Time Series Normalization}

\begin{frame}{Time Series Normalization}
    \begin{itemize}
        \item z-normalization or Z-scoring time series is common prior DTW calculation \cite{ding2008querying}
        
        \item Two different time series normalizations $\eta$ and $\eta '$ from \cite{das1998rule}
    \end{itemize}
\end{frame}

\begin{frame}{Time Series Normalization}{Calculation}
    \begin{block}{Given}
        \begin{itemize}
            \item A domain set $\mathbb{U}$
            
            \item A distance measure function $d$ with $d: \mathbb{U} \times \mathbb{U} \to \mathbb{R}$
            
        \end{itemize}
    \end{block}
    \begin{block}{Input}
        \begin{itemize}
            \item A time series $Q = (q_1, q_2, \dots, q_i, \dots, q_l)$ with length $l$ over the domain set
                $\mathbb{U}$
        \end{itemize}
    \end{block}
\end{frame}

\begin{frame}{Time Series Normalization $\eta$}{Calculation}
    \begin{block}{Calculation}
        \begin{itemize}
            \item Every data point $q$ of $Q$ will be transformed by $\eta$
            
            \item $\eta (q) = q -\bar{q}$
            
        \end{itemize}
    \end{block}
    \begin{block}{Mean of $Q$}
        \begin{itemize}
            \item $\bar{q} = \frac{1}{l} \sum \limits_{i=1}^{l} q_i$
        \end{itemize}
    \end{block}
\end{frame}

\begin{frame}<handout:0>{Time Series Normalization $\eta$}{Example}
    \begin{center}
        \resizebox {\textwidth} {!} {
            \begin{tabular}{cc}
                \resizebox* {!} {0.3\textwidth} {
                    \begin{tikzpicture}
                        \begin{axis}[
                            xmin=0,
                            xmax=47,
                            xlabel=time,
                            ylabel=acceleration,
                            width=\axisdefaultwidth,
                            height=0.7*\axisdefaultheight,
                            reverse legend,
                            legend pos=south east]
                            \addplot[red, thick, mark=none] table {../data/fig/dynamictimewarping/q.dat};
                            \addlegendentry{Q}
                            \addplot[blue, thick, mark=none] table {../data/fig/dynamictimewarping/c.dat};
                            \addlegendentry{C}
                        \end{axis}
                    \end{tikzpicture}
                } & \quad
                \resizebox* {!} {0.3\textwidth} {
                    \begin{tabular}[b]{ll}
                        \begin{turn}{90}
                            \begin{tikzpicture}
                                \begin{axis}[
                                    xmin=0,
                                    xmax=47,
                                    ymin=-100,
                                    ymax=0,
                                    hide x axis,
                                    hide y axis,
                                    width=\axisdefaultwidth,
                                    height=0.7*\axisdefaultheight]
                                    \addplot[red, ultra thick, mark=none] table {../data/fig/dynamictimewarping/q.dat};
                                \end{axis}
                            \end{tikzpicture}
                        \end{turn} \hspace*{3em} &
                        \begin{tikzpicture}
                            \begin{axis}[
                                enlargelimits=false,
                                ymin=0,
                                ymax=47,
                                hide x axis,
                                hide y axis,
                                width=\axisdefaultwidth,
                                height=\axisdefaultwidth,
                                colorbar,
                                colormap/viridis high res]
                                \addplot[matrix plot*,
                                    mesh/cols=48,
                                    point meta=explicit] table[meta=C] {../data/fig/dynamictimewarping/matrix.dat};
                            \end{axis}
                        \end{tikzpicture} \\
                        &
                        \\[1em]
                        &
                        \begin{tikzpicture}
                            \begin{axis}[
                                xmin=0,
                                xmax=47,
                                ymin=-100,
                                ymax=0,
                                hide x axis,
                                hide y axis,
                                width=\axisdefaultwidth,
                                height=0.7*\axisdefaultheight]
                                \addplot[blue, ultra thick, mark=none] table {../data/fig/dynamictimewarping/c.dat};
                            \end{axis}
                        \end{tikzpicture}
                    \end{tabular}
                }
            \end{tabular}
        }
    \end{center}
\end{frame}

\begin{frame}<handout:0>{Time Series Normalization $\eta$}{Example}
    \begin{center}
        \resizebox {\textwidth} {!} {
            \begin{tabular}{cc}
                \resizebox* {!} {0.3\textwidth} {
                    \begin{tikzpicture}
                        \begin{axis}[
                            xmin=0,
                            xmax=47,
                            xlabel=time,
                            ylabel=acceleration,
                            width=\axisdefaultwidth,
                            height=0.7*\axisdefaultheight,
                            reverse legend,
                            legend pos=south east]
                            \addplot[red, thick, mark=none] table {../data/fig/norm1/q.dat};
                            \addlegendentry{Q}
                            \addplot[blue, thick, mark=none] table {../data/fig/norm1/c.dat};
                            \addlegendentry{C}
                        \end{axis}
                    \end{tikzpicture}
                } & \quad
                \resizebox* {!} {0.3\textwidth} {
                    \begin{tabular}[b]{ll}
                        \begin{turn}{90}
                            \begin{tikzpicture}
                                \begin{axis}[
                                    xmin=0,
                                    xmax=47,
                                    ymin=-40,
                                    ymax=40,
                                    hide x axis,
                                    hide y axis,
                                    width=\axisdefaultwidth,
                                    height=0.7*\axisdefaultheight]
                                    \addplot[red, ultra thick, mark=none] table {../data/fig/norm1/q.dat};
                                \end{axis}
                            \end{tikzpicture}
                        \end{turn} \hspace*{3em} &
                        \begin{tikzpicture}
                            \begin{axis}[
                                enlargelimits=false,
                                ymin=0,
                                ymax=47,
                                hide x axis,
                                hide y axis,
                                width=\axisdefaultwidth,
                                height=\axisdefaultwidth,
                                colorbar,
                                colormap/viridis high res]
                                \addplot[matrix plot*,
                                    mesh/cols=48,
                                    point meta=explicit] table[meta=C] {../data/fig/norm1/matrix.dat};
                            \end{axis}
                        \end{tikzpicture}\\
                        &
                        \\[1em]
                        &
                        \begin{tikzpicture}
                            \begin{axis}[
                                xmin=0,
                                xmax=47,
                                ymin=-40,
                                ymax=40,
                                hide x axis,
                                hide y axis,
                                width=\axisdefaultwidth,
                                height=0.7*\axisdefaultheight]
                                \addplot[blue, ultra thick, mark=none] table {../data/fig/norm1/c.dat};
                            \end{axis}
                        \end{tikzpicture}
                    \end{tabular}
                }
            \end{tabular}
        }
    \end{center}
\end{frame}

\begin{frame}{Time Series Normalization $\eta$}{Example}
    \begin{center}
        \resizebox {\textwidth} {!} {
            \begin{tabular}{cc}
                \resizebox* {!} {0.3\textwidth} {
                    \begin{tikzpicture}
                        \begin{axis}[
                            xmin=0,
                            xmax=47,
                            xlabel=time,
                            ylabel=acceleration,
                            width=\axisdefaultwidth,
                            height=0.7*\axisdefaultheight,
                            reverse legend,
                            legend pos=south east]
                            \addplot[gray, quiver={u=\thisrow{u}, v=\thisrow{v}}] table {../data/fig/norm1/path.dat};
                            \addplot[red, thick, mark=none] table {../data/fig/norm1/q.dat};
                            \addlegendentry{Q}
                            \addplot[blue, thick, mark=none] table {../data/fig/norm1/c.dat};
                            \addlegendentry{C}
                        \end{axis}
                    \end{tikzpicture}
                } & \quad
                \resizebox* {!} {0.3\textwidth} {
                    \begin{tabular}[b]{ll}
                        \begin{turn}{90}
                            \begin{tikzpicture}
                                \begin{axis}[
                                    xmin=0,
                                    xmax=47,
                                    ymin=-40,
                                    ymax=40,
                                    hide x axis,
                                    hide y axis,
                                    width=\axisdefaultwidth,
                                    height=0.7*\axisdefaultheight]
                                    \addplot[red, ultra thick, mark=none] table {../data/fig/norm1/q.dat};
                                \end{axis}
                            \end{tikzpicture}
                        \end{turn} \hspace*{3em} &
                        \begin{tikzpicture}
                            \begin{axis}[
                                enlargelimits=false,
                                ymin=0,
                                ymax=47,
                                hide x axis,
                                hide y axis,
                                width=\axisdefaultwidth,
                                height=\axisdefaultwidth,
                                colorbar,
                                colormap/viridis high res]
                                \addplot[matrix plot*,
                                    mesh/cols=48,
                                    point meta=explicit] table[meta=C] {../data/fig/norm1/matrix.dat};
                                \addplot[white, ultra thick, mark=*, mark size=1] table {../data/fig/norm1/matrix_path.dat};
                            \end{axis}
                        \end{tikzpicture}\\
                        &
                        \\[1em]
                        &
                        \begin{tikzpicture}
                            \begin{axis}[
                                xmin=0,
                                xmax=47,
                                ymin=-40,
                                ymax=40,
                                hide x axis,
                                hide y axis,
                                width=\axisdefaultwidth,
                                height=0.7*\axisdefaultheight]
                                \addplot[blue, ultra thick, mark=none] table {../data/fig/norm1/c.dat};
                            \end{axis}
                        \end{tikzpicture}
                    \end{tabular}
                }
            \end{tabular}
        }
    \end{center}
\end{frame}

\begin{frame}{Time Series Normalization $\eta '$}{Calculation}
    \begin{block}{Calculation}
        \begin{itemize}
            \item Every data point $q$ of $Q$ will be transformed by $\eta '$
            
            \item $\eta '(q) = \frac{\eta (q)}{\sigma}$
            
        \end{itemize}
    \end{block}
    \begin{block}{Standard deviation of $Q$}
        \begin{itemize}
            \item $\sigma = \frac{1}{l-1} \sum \limits_{i=1}^{l} d(q_i, \bar{q})^2$
        \end{itemize}
    \end{block}
\end{frame}

\begin{frame}<handout:0>{Time Series Normalization $\eta '$}{Example}
    \begin{center}
        \resizebox {\textwidth} {!} {
            \begin{tabular}{cc}
                \resizebox* {!} {0.3\textwidth} {
                    \begin{tikzpicture}
                        \begin{axis}[
                            xmin=0,
                            xmax=47,
                            xlabel=time,
                            ylabel=acceleration,
                            width=\axisdefaultwidth,
                            height=0.7*\axisdefaultheight,
                            reverse legend,
                            legend pos=south east]
                            \addplot[red, thick, mark=none] table {../data/fig/dynamictimewarping/q.dat};
                            \addlegendentry{Q}
                            \addplot[blue, thick, mark=none] table {../data/fig/dynamictimewarping/c.dat};
                            \addlegendentry{C}
                        \end{axis}
                    \end{tikzpicture}
                } & \quad
                \resizebox* {!} {0.3\textwidth} {
                    \begin{tabular}[b]{ll}
                        \begin{turn}{90}
                            \begin{tikzpicture}
                                \begin{axis}[
                                    xmin=0,
                                    xmax=47,
                                    ymin=-100,
                                    ymax=0,
                                    hide x axis,
                                    hide y axis,
                                    width=\axisdefaultwidth,
                                    height=0.7*\axisdefaultheight]
                                    \addplot[red, ultra thick, mark=none] table {../data/fig/dynamictimewarping/q.dat};
                                \end{axis}
                            \end{tikzpicture}
                        \end{turn} \hspace*{3em} &
                        \begin{tikzpicture}
                            \begin{axis}[
                                enlargelimits=false,
                                ymin=0,
                                ymax=47,
                                hide x axis,
                                hide y axis,
                                width=\axisdefaultwidth,
                                height=\axisdefaultwidth,
                                colorbar,
                                colormap/viridis high res]
                                \addplot[matrix plot*,
                                    mesh/cols=48,
                                    point meta=explicit] table[meta=C] {../data/fig/dynamictimewarping/matrix.dat};
                            \end{axis}
                        \end{tikzpicture} \\
                        &
                        \\[1em]
                        &
                        \begin{tikzpicture}
                            \begin{axis}[
                                xmin=0,
                                xmax=47,
                                ymin=-100,
                                ymax=0,
                                hide x axis,
                                hide y axis,
                                width=\axisdefaultwidth,
                                height=0.7*\axisdefaultheight]
                                \addplot[blue, ultra thick, mark=none] table {../data/fig/dynamictimewarping/c.dat};
                            \end{axis}
                        \end{tikzpicture}
                    \end{tabular}
                }
            \end{tabular}
        }
    \end{center}
\end{frame}

\begin{frame}<handout:0>{Time Series Normalization $\eta '$}{Example}
    \begin{center}
        \resizebox {\textwidth} {!} {
            \begin{tabular}{cc}
                \resizebox* {!} {0.3\textwidth} {
                    \begin{tikzpicture}
                        \begin{axis}[
                            xmin=0,
                            xmax=47,
                            xlabel=time,
                            ylabel=acceleration,
                            width=\axisdefaultwidth,
                            height=0.7*\axisdefaultheight,
                            reverse legend,
                            legend pos=south east]
                            \addplot[red, thick, mark=none] table {../data/fig/norm1/q.dat};
                            \addlegendentry{Q}
                            \addplot[blue, thick, mark=none] table {../data/fig/norm1/c.dat};
                            \addlegendentry{C}
                        \end{axis}
                    \end{tikzpicture}
                } & \quad
                \resizebox* {!} {0.3\textwidth} {
                    \begin{tabular}[b]{ll}
                        \begin{turn}{90}
                            \begin{tikzpicture}
                                \begin{axis}[
                                    xmin=0,
                                    xmax=47,
                                    ymin=-40,
                                    ymax=40,
                                    hide x axis,
                                    hide y axis,
                                    width=\axisdefaultwidth,
                                    height=0.7*\axisdefaultheight]
                                    \addplot[red, ultra thick, mark=none] table {../data/fig/norm1/q.dat};
                                \end{axis}
                            \end{tikzpicture}
                        \end{turn} \hspace*{3em} &
                        \begin{tikzpicture}
                            \begin{axis}[
                                enlargelimits=false,
                                ymin=0,
                                ymax=47,
                                hide x axis,
                                hide y axis,
                                width=\axisdefaultwidth,
                                height=\axisdefaultwidth,
                                colorbar,
                                colormap/viridis high res]
                                \addplot[matrix plot*,
                                    mesh/cols=48,
                                    point meta=explicit] table[meta=C] {../data/fig/norm1/matrix.dat};
                            \end{axis}
                        \end{tikzpicture}\\
                        &
                        \\[1em]
                        &
                        \begin{tikzpicture}
                            \begin{axis}[
                                xmin=0,
                                xmax=47,
                                ymin=-40,
                                ymax=40,
                                hide x axis,
                                hide y axis,
                                width=\axisdefaultwidth,
                                height=0.7*\axisdefaultheight]
                                \addplot[blue, ultra thick, mark=none] table {../data/fig/norm1/c.dat};
                            \end{axis}
                        \end{tikzpicture}
                    \end{tabular}
                }
            \end{tabular}
        }
    \end{center}
\end{frame}

\begin{frame}<handout:0>{Time Series Normalization $\eta '$}{Example}
    \begin{center}
        \resizebox {\textwidth} {!} {
            \begin{tabular}{cc}
                \resizebox* {!} {0.3\textwidth} {
                    \begin{tikzpicture}
                        \begin{axis}[
                            xmin=0,
                            xmax=47,
                            xlabel=time,
                            ylabel=acceleration,
                            width=\axisdefaultwidth,
                            height=0.7*\axisdefaultheight,
                            reverse legend,
                            legend pos=south east]
                            \addplot[red, thick, mark=none] table {../data/fig/norm2/q.dat};
                            \addlegendentry{Q}
                            \addplot[blue, thick, mark=none] table {../data/fig/norm2/c.dat};
                            \addlegendentry{C}
                        \end{axis}
                    \end{tikzpicture}
                } & \quad
                \resizebox* {!} {0.3\textwidth} {
                    \begin{tabular}[b]{ll}
                        \begin{turn}{90}
                            \begin{tikzpicture}
                                \begin{axis}[
                                    xmin=0,
                                    xmax=47,
                                    ymin=-2,
                                    ymax=2,
                                    hide x axis,
                                    hide y axis,
                                    width=\axisdefaultwidth,
                                    height=0.7*\axisdefaultheight]
                                    \addplot[red, ultra thick, mark=none] table {../data/fig/norm2/q.dat};
                                \end{axis}
                            \end{tikzpicture}
                        \end{turn} \hspace*{3em} &
                        \begin{tikzpicture}
                            \begin{axis}[
                                enlargelimits=false,
                                ymin=0,
                                ymax=47,
                                hide x axis,
                                hide y axis,
                                width=\axisdefaultwidth,
                                height=\axisdefaultwidth,
                                colorbar,
                                colormap/viridis high res]
                                \addplot[matrix plot*,
                                    mesh/cols=48,
                                    point meta=explicit] table[meta=C] {../data/fig/norm2/matrix.dat};
                            \end{axis}
                        \end{tikzpicture}\\
                        &
                        \\[1em]
                        &
                        \begin{tikzpicture}
                            \begin{axis}[
                                xmin=0,
                                xmax=47,
                                ymin=-2,
                                ymax=2,
                                hide x axis,
                                hide y axis,
                                width=\axisdefaultwidth,
                                height=0.7*\axisdefaultheight]
                                \addplot[blue, ultra thick, mark=none] table {../data/fig/norm2/c.dat};
                            \end{axis}
                        \end{tikzpicture}
                    \end{tabular}
                }
            \end{tabular}
        }
    \end{center}
\end{frame}

\begin{frame}{Time Series Normalization $\eta '$}{Example}
    \begin{center}
        \resizebox {\textwidth} {!} {
            \begin{tabular}{cc}
                \resizebox* {!} {0.3\textwidth} {
                    \begin{tikzpicture}
                        \begin{axis}[
                            xmin=0,
                            xmax=47,
                            xlabel=time,
                            ylabel=acceleration,
                            width=\axisdefaultwidth,
                            height=0.7*\axisdefaultheight,
                            reverse legend,
                            legend pos=south east]
                            \addplot[gray, quiver={u=\thisrow{u}, v=\thisrow{v}}] table {../data/fig/norm2/path.dat};
                            \addplot[red, thick, mark=none] table {../data/fig/norm2/q.dat};
                            \addlegendentry{Q}
                            \addplot[blue, thick, mark=none] table {../data/fig/norm2/c.dat};
                            \addlegendentry{C}
                        \end{axis}
                    \end{tikzpicture}
                } & \quad
                \resizebox* {!} {0.3\textwidth} {
                    \begin{tabular}[b]{ll}
                        \begin{turn}{90}
                            \begin{tikzpicture}
                                \begin{axis}[
                                    xmin=0,
                                    xmax=47,
                                    ymin=-2,
                                    ymax=2,
                                    hide x axis,
                                    hide y axis,
                                    width=\axisdefaultwidth,
                                    height=0.7*\axisdefaultheight]
                                    \addplot[red, ultra thick, mark=none] table {../data/fig/norm2/q.dat};
                                \end{axis}
                            \end{tikzpicture}
                        \end{turn} \hspace*{3em} &
                        \begin{tikzpicture}
                            \begin{axis}[
                                enlargelimits=false,
                                ymin=0,
                                ymax=47,
                                hide x axis,
                                hide y axis,
                                width=\axisdefaultwidth,
                                height=\axisdefaultwidth,
                                colorbar,
                                colormap/viridis high res]
                                \addplot[matrix plot*,
                                    mesh/cols=48,
                                    point meta=explicit] table[meta=C] {../data/fig/norm2/matrix.dat};
                                \addplot[white, ultra thick, mark=*, mark size=1] table {../data/fig/norm2/matrix_path.dat};
                            \end{axis}
                        \end{tikzpicture}\\
                        &
                        \\[1em]
                        &
                        \begin{tikzpicture}
                            \begin{axis}[
                                xmin=0,
                                xmax=47,
                                ymin=-2,
                                ymax=2,
                                hide x axis,
                                hide y axis,
                                width=\axisdefaultwidth,
                                height=0.7*\axisdefaultheight]
                                \addplot[blue, ultra thick, mark=none] table {../data/fig/norm2/c.dat};
                            \end{axis}
                        \end{tikzpicture}
                    \end{tabular}
                }
            \end{tabular}
        }
    \end{center}
\end{frame}

