\subsection{Dynamic Time Warping} \label{dynamic_time_warping}
Dynamic Time Warping (DTW) is a much more robust distance measure for time series than ED, it is allowing
similar shapes to match even if they are out of phase in the time axis \cite{keogh2002exact}. The following explaination
to calculate the DTW distance is a very basic and simple one that corresponds in the result to the method of
\textit{Using dynamic time warping to find patterns in time series} \cite{berndt1994using}.

Given are two time series $Q = (q_1, q_2, \dots, q_i, \dots, q_l)$ with length
$l$, $C = (c_1, c_2, \dots, c_j, \dots, c_k)$ with length $k$ over the domain set $\mathbb{U}$ and a distance measure
function $D$ with $D: \mathbb{U} \times \mathbb{U} \to \mathbb{R}$. Calculating the DTW distance between the two time
series $Q$ and $C$ can be achieved by calculating a matrix $M$ of size $l \times k$ with the following rule.
\begin{center} \[ M_{i, j} = \begin{cases}
    D(q_i,c_j) & \text{if } i = 1 \wedge j = 1\\
    M_{i,j-1} + D(q_i,c_j) & \text{if } i = 1 \wedge j \neq 1\\
    M_{i-1,j} + D(q_i,c_j) & \text{if } i \neq 1 \wedge j = 1\\
    min(M_{i-1,j}, M_{i-1,j-1}, M_{i,j-1}) + D(q_i,c_j) & \text{if } i \neq 1 \wedge j \neq 1
\end{cases} \] \end{center}
The DTW distance between the two time series $Q$ and $C$ is the entry $M_{l,k}$ of the resulting matrix.
\begin{center}
    $DTW(Q, C) = M_{l,k}$
\end{center}
The warping path as result of the backtracking is irrelevant for the aim of this bachelor thesis. Figure
\ref{fig:dynamictimewarping} illustrates DTW for two time series $Q$ and $C$ that contain recorded data from one
acceleration sensor. DTW shown as above has time and space complexity of $\mathcal{O}(lk)$. When ignoring the
warping path as result the algorithm can easely reduce the space complexity to $\mathcal{O}(min(l, k))$. This can be
achieved by keeping only the last important enties in space that are necessary to calculate the final entriy $M_{l,k}$
of the matrix.

\begin{figure}[H]
    \begin{center}
        \begin{tikzpicture}
            \begin{axis}[
                xmin=1,
                xmax=90,
                xlabel=time,
                ylabel=acceleration,
                width=\textwidth,
                height=\axisdefaultheight,
                reverse legend]
                \addplot[gray, quiver={u=\thisrow{u}, v=\thisrow{v}}] table[x=t, y=c] {background_and_notation/dynamic_time_warping/path.dat};
                \addplot[blue, mark=none] table[x=t, y=q] {background_and_notation/dynamic_time_warping/timeseries.dat};
                \addlegendentry{Q}
                \addplot[red, mark=none] table[x=t, y=c] {background_and_notation/dynamic_time_warping/timeseries.dat};
                \addlegendentry{C}
            \end{axis}
        \end{tikzpicture}
    \end{center}
    \caption{Two time series $Q$ and $C$ containing recorded data from one acceleration sensor. The gray lines represent
    the warping path of DTW.}
    \label{fig:dynamictimewarping}
\end{figure}
