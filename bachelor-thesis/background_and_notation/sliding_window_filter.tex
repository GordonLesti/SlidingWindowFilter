\subsection{Sliding Window Filter} \label{sliding_window_filter}
This section explains a sliding window filter and the way to embed the filter into a sliding window application as
described in section \ref{sliding_window_technique}. A time series filter is in general a function that takes a time
series as argument and returns true or false. One possible underlying inner functionality approach of the filter are
presented later in this section. The previous version of a sliding window application is extracting the current time
series window and passes it directly to a time series classificator. This approach is extended by embedding a time
series filter directly after the extraction and ahead of the time series classificator. The time series filter has the
task to prune the amount of time series windows that reach the classificator. Time series windows which would be
assessed as unclassifiable by the classificator should be blocked by the filter as much as possible. Figure
\ref{fig:swf} illustrates the way of embedding the time series filter into the workflow of a sliding window application.

\begin{figure}
    \begin{center}
        \resizebox {\textwidth} {!} {
            {\tiny
                \begin{tikzpicture}[node distance = 1.5cm, auto]
                    \node [block] (sod) {sensors or devices};
                    \node [block, right of=sod, node distance=6cm, text width=2cm] (extract) {Extract last subsequence from Q of size $w$, $Q[t-w,t]$};
                    \node [block, draw=blue, right of=extract, node distance=4cm, text width=2cm] (filter) {Time series filter};
                    \node [decision, draw=blue, below of=filter] (filterdecide) {$Q[t-w,t]$ can pass?};
                    \node [block, below of=filterdecide, node distance=2cm, text width=2cm] (nnc) {Time series classificator};
                    \node [decision, below of=nnc] (decide) {$Q[t-w,t]$ classifiable?};
                    \node [block, left of=decide, node distance=3cm] (sleeps) {Sleep for $s$ time};
                    \node [block, below of=decide, node distance=2cm, text width=2cm] (action) {Trigger event that $Q[t-w,t]$ has been classified and sleep for $w$ time};

                    \path [line,dashed] (sod) -- node (ctss) {Continous time series stream $Q$} (extract);
                    \path [line] (extract) -- node {$Q[t-w,t]$} (filter);
                    \path [line] (filter) -- (filterdecide);
                    \path [line] (filterdecide) -- node {yes} (nnc);
                    \path [line] (filterdecide) -| node [near start] {no} (sleeps);
                    \path [line] (nnc) -- (decide);
                    \path [line] (decide) -- node {no} (sleeps);
                    \path [line,dashed] (sleeps) -| (ctss);
                    \path [line] (decide) -- node {yes} (action);
                    \path [line,dashed] (action) -| (ctss);
                \end{tikzpicture}
            }
        }
    \end{center}
    \caption{Extended design for a sliding window application as in figure \ref{fig:swt}, plus the additional filter
    highlighted in blue. The current time is stored in variable $c$. The variables $w$ for the window size and $s$ for
    the step size are predefined.}
    \label{fig:swf}
\end{figure}

The integration of the time series filter into a sliding window application is described above. An underlying inner
functionality approach of a time series filter is the usage of a simple time series measures function. The argument of
such a time series measure function should be one time series and the result should be an element of $\mathbb{R}$.
Linear time and memory complexity are basic requirements for a measure to ensure an acceptable performance of the
filter. A filter instance based on a suitable time series measure has access to the same training set of time series as
the classificator. The measure executed on every time series in the training set results in a maximum and a minimum
value. Both values together are creating an interval. This interval is called filter interval. Every measure value of the
training set is inside of the boundaries of the filter interval. The approach of a measure based filter is that many
classifiable time series windows should have a measure value inside of the interval and many unclassifiable time series
windows should have a measure value outside of the interval boundaries. Incoming time series windows with a measure
function value inside the interval boundaries can pass to the classification. All other windows are blocked by the
filter. A factor can expand the filter interval to avoid the mistakenly blocking of classifiable time series windows.
This factor is called blur factor and is expressed as a percentage number. Assumed that the left boundary of the filter
interval is 10 and the right boundary is 20. A blur factor of 160\% will result in a new filter interval of 7 and 23.
The filter interval will be expanded artificially on the left and the right side. Possible time series measure functions
are mentioned in \ref{complexity_estimate} and \ref{sample_variance}.
