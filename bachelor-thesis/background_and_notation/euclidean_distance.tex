\subsection{Euclidean Distance}
The Euclidean Distance (ED) is the most straightforward similarity measure for time series \cite{ding2008querying}.
Given are two time series $Q = (q_1, q_2, \dots, q_i, \dots, q_l)$, $C = (c_1, c_2, \dots, c_j, \dots, c_l)$ with the
same length $l$ over the domain set $\mathbb{U}$ and a distance measure function $D$ with
$D: \mathbb{U} \times \mathbb{U} \to \mathbb{R}$. The ED between those two time series can be calculated with the
following formula.
\begin{center}
    $ED(Q, C) = \sqrt[2]{\sum \limits_{i=1}^{l} D(q_i, c_i)^2}$
\end{center}

\begin{figure}[H]
    \begin{center}
        \begin{tikzpicture}
            \begin{axis}[
                xlabel=time,
                ylabel=acceleration,
                width=\textwidth,
                height=\axisdefaultheight]
                \addplot[blue, mark=none] table[x=t, y=q] {background_and_notation/euclidean_distance/timeseries.dat};
                \addlegendentry{Q}
                \addplot[red, mark=none] table[x=t, y=c] {background_and_notation/euclidean_distance/timeseries.dat};
                \addlegendentry{C}
                \addplot[gray, quiver={v=\thisrow{v}}] table[x=t, y=q] {background_and_notation/euclidean_distance/timeseries.dat};
            \end{axis}
        \end{tikzpicture}
    \end{center}
    \caption{Two time series $Q$ and $C$ containing recorded data from one acceleration sensor. The square root of the
    sum of the gray distance lines illustrate the Euclidean distance between the two time series.}
    \label{fig:euclideandistance}
\end{figure}

An advantage of ED is the linear time and space complexity of $\mathcal{O}(l)$ to calculate the similarity between two
time series. The algorithm itself has even only a space complexity of $\mathcal{O}(1)$ if the space of the input time
series is ignored. Comparing two time series of different length with the ED presupposes to transform the time series
to the same length. That can be achieved by stretch the smaller time series for example. Figure
\ref{fig:euclideandistance} illustrates the Euclidean Distance between a time series $Q$ and $C$ that contain recorded
data from one acceleration sensor.
