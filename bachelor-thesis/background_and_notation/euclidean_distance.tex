\subsection{Euclidean Distance}
The Euclidean Distance (ED) is the most straightforward similarity measure for time series \cite{ding2008querying}.
Given are two time series $Q = (q_1, q_2, \dots, q_i, \dots, q_l)$, $R = (r_1, r_2, \dots, r_i, \dots, r_l)$ with the
same length $l$ over the domain set $\mathbb{U}$ and a distance measure function $D$ with
$D: \mathbb{U} \times \mathbb{U} \to \mathbb{R}$. The ED between those two time series can be calculated with the
following formula.
\begin{center}
    $ED(Q, R) = \sqrt[2]{\sum \limits_{i=1}^{n} D(q_i, r_i)^2}$
\end{center}
An advantage of ED is the linear complexity to calculate the similarity between two time series. Comparing two time
series of different length with the ED presupposes to transform the time series to the same length. That can be achieved
by stretch the smaller time series for example. Figure \ref{fig:euclideandistance} illustrates the Euclidean Distance
between a time series $Q$ and $R$ that contain recorded data from one acceleration sensor.

\begin{figure}[H]
    \begin{center}
        \begin{tikzpicture}
            \begin{axis}[
                xlabel=time,
                ylabel=acceleration,
                width=\textwidth,
                height=\axisdefaultheight]
                \addplot[blue, mark=none] table[x=t, y=q] {background_and_notation/euclidean_distance/timeseries.dat};
                \addlegendentry{Q}
                \addplot[red, mark=none] table[x=t, y=r] {background_and_notation/euclidean_distance/timeseries.dat};
                \addlegendentry{R}
                \addplot[gray, quiver={v=\thisrow{v}}] table[x=t, y=q] {background_and_notation/euclidean_distance/timeseries.dat};
            \end{axis}
        \end{tikzpicture}
    \end{center}
    \caption{Two time series $Q$ and $R$ containing recorded data from one acceleration sensor. The square root of the
    sum of the gray distance lines illustrate the Euclidean distance between the two time series.}
    \label{fig:euclideandistance}
\end{figure}
