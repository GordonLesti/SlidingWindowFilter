\subsubsection{Sakoe-Chiba Band} \label{sakoe-chiba_band}
Plain DTW uses a full distance matrix to calculate the similarity between two time series as shown on the right plot of
figure \ref{fig:dynamictimewarping}. This can allow DTW to match a data point at the beginning of a time series to a
data point at the end of the compared time series. To prevent such scenarios, it is a common approach to restrict the
area of the distance matrix. Popular constraints are the Sakoe-Chiba band \cite{sakoe1978dynamic} and the Itakura
parallelogram \cite{itakura1975minimum}. This thesis will focus only on the Sakoe-Chiba band. The adjustment window or
the width of the Sakoe-Chiba band can be scaled in a general context by a percentage factor of the compared time series
length. Figure \ref{fig:scb} clarifies the influence of a 20\% Sakoe-Chiba band on DTW compared to figure
\ref{fig:dynamictimewarping} with plain DTW. The maximum Sakoe-Chiba band size are 200\%, but this would result in plain
DTW. The implementation of a constraint can be easily achieved by setting the distance between two data points to
positive infinity if the associated entry in the distance matrix is outside of the restricted area. The equation
\ref{eq:dtw} can be expanded by a dominating $\infty$ case if $i$ and $j$ do not pass the condition.

\begin{figure}
    \begin{center}
        \resizebox {\textwidth} {!} {
            \begin{tabular}{cc}
                \resizebox* {!} {0.3\textwidth} {
                    \begin{tikzpicture}
                        \begin{axis}[
                            xmin=0,
                            xmax=47,
                            xlabel=time,
                            ylabel=acceleration,
                            width=\axisdefaultwidth,
                            height=0.7*\axisdefaultheight,
                            reverse legend,
                            legend pos=south east]
                            \addplot[gray, quiver={u=\thisrow{u}, v=\thisrow{v}}] table {../data/fig/scb/path.dat};
                            \addplot[red, thick, mark=none] table {../data/fig/scb/q.dat};
                            \addlegendentry{Q}
                            \addplot[blue, thick, mark=none] table {../data/fig/scb/c.dat};
                            \addlegendentry{C}
                        \end{axis}
                    \end{tikzpicture}
                } & \quad
                \resizebox* {!} {0.3\textwidth} {
                    \begin{tabular}[b]{ll}
                        \begin{turn}{90}
                            \begin{tikzpicture}
                                \begin{axis}[
                                    xmin=0,
                                    xmax=47,
                                    ymin=-100,
                                    ymax=0,
                                    hide x axis,
                                    hide y axis,
                                    width=\axisdefaultwidth,
                                    height=0.7*\axisdefaultheight]
                                    \addplot[red, ultra thick, mark=none] table {../data/fig/scb/q.dat};
                                \end{axis}
                            \end{tikzpicture}
                        \end{turn} \hspace*{3em} &
                        \begin{tikzpicture}
                            \begin{axis}[
                                enlargelimits=false,
                                ymin=0,
                                ymax=47,
                                hide x axis,
                                hide y axis,
                                width=\axisdefaultwidth,
                                height=\axisdefaultwidth,
                                colorbar,
                                colormap/viridis high res]
                                \addplot[matrix plot*,
                                    shader=flat,
                                    mesh/cols=48,
                                    point meta=explicit] table[meta=C] {../data/fig/scb/matrix.dat};
                                \addplot[white, ultra thick, mark=*, mark size=1] table {../data/fig/scb/matrix_path.dat};
                            \end{axis}
                        \end{tikzpicture}\\
                        &
                        \\[1em]
                        &
                        \begin{tikzpicture}
                            \begin{axis}[
                                xmin=0,
                                xmax=47,
                                ymin=-100,
                                ymax=0,
                                hide x axis,
                                hide y axis,
                                width=\axisdefaultwidth,
                                height=0.7*\axisdefaultheight]
                                \addplot[blue, ultra thick, mark=none] table {../data/fig/scb/c.dat};
                            \end{axis}
                        \end{tikzpicture}
                    \end{tabular}
                }
            \end{tabular}
        }
    \end{center}
    \caption{The same two time series $Q$ and $C$ from figure \ref{fig:dynamictimewarping}. On the
    left plot are both time series graphs, the gray lines are representing the warping path of DTW with a Sakoe-Chiba band
    of size 20\%. The right plot shows the associated matrix containing the distances between the time series data
    points. Starting in the lower left corner and ending in the upper right corner, the warping path is illustrated as
    a white graph.}
    \label{fig:scb}
\end{figure}
