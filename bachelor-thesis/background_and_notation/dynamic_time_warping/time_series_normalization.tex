\subsubsection{Time Series Normalization} \label{time_series_normalization}
When using DTW to determine the similarity of two time series, it is a common approach to z-normalize or Z-score the
time series in front of the calculation \cite{ding2008querying}. In \cite{das1998rule} two different time series
normalizations are presented. Both time series normalizations $\eta$ and $\eta '$ are explained in the following
section.

Given is a time series $Q = (q_1, q_2, \dots, q_i, \dots, q_l)$ with length $l$ over the domain set $\mathbb{U}$ and a
distance measure function $d$ with $d: \mathbb{U} \times \mathbb{U} \to \mathbb{R}$. The first time series normalization
$\eta$ transforms every data
point of the given time series $Q$ as follows.
\begin{equation}
    \eta (q) = q -\bar{q}
\end{equation}
where $\bar{q}$ is the mean of the time series $Q$
\begin{equation}
    \bar{q} = \frac{1}{l} \sum \limits_{i=1}^{l} q_i
\end{equation}
Figure \ref{fig:norm1} shows the impact of the $\eta$ normalization on DTW compared to plain DTW in figure
\ref{fig:dynamictimewarping}.

\begin{figure}
    \begin{center}
        \resizebox {\textwidth} {!} {
            \begin{tabular}{cc}
                \resizebox* {!} {0.3\textwidth} {
                    \begin{tikzpicture}
                        \begin{axis}[
                            xmin=0,
                            xmax=47,
                            xlabel=time,
                            ylabel=acceleration,
                            width=\axisdefaultwidth,
                            height=0.7*\axisdefaultheight,
                            reverse legend,
                            legend pos=south east]
                            \addplot[gray, quiver={u=\thisrow{u}, v=\thisrow{v}}] table {../data/fig/norm1/path.dat};
                            \addplot[red, thick, mark=none] table {../data/fig/norm1/q.dat};
                            \addlegendentry{Q}
                            \addplot[blue, thick, mark=none] table {../data/fig/norm1/c.dat};
                            \addlegendentry{C}
                        \end{axis}
                    \end{tikzpicture}
                } & \quad
                \resizebox* {!} {0.3\textwidth} {
                    \begin{tabular}[b]{ll}
                        \begin{turn}{90}
                            \begin{tikzpicture}
                                \begin{axis}[
                                    xmin=0,
                                    xmax=47,
                                    ymin=-40,
                                    ymax=40,
                                    hide x axis,
                                    hide y axis,
                                    width=\axisdefaultwidth,
                                    height=0.7*\axisdefaultheight]
                                    \addplot[red, ultra thick, mark=none] table {../data/fig/norm1/q.dat};
                                \end{axis}
                            \end{tikzpicture}
                        \end{turn} \hspace*{3em} &
                        \begin{tikzpicture}
                            \begin{axis}[
                                enlargelimits=false,
                                ymin=0,
                                ymax=47,
                                hide x axis,
                                hide y axis,
                                width=\axisdefaultwidth,
                                height=\axisdefaultwidth,
                                colorbar,
                                colormap/viridis high res]
                                \addplot[matrix plot*,
                                    mesh/cols=48,
                                    point meta=explicit] table[meta=C] {../data/fig/norm1/matrix.dat};
                                \addplot[white, ultra thick, mark=*, mark size=1] table {../data/fig/norm1/matrix_path.dat};
                            \end{axis}
                        \end{tikzpicture}\\
                        &
                        \\[1em]
                        &
                        \begin{tikzpicture}
                            \begin{axis}[
                                xmin=0,
                                xmax=47,
                                ymin=-40,
                                ymax=40,
                                hide x axis,
                                hide y axis,
                                width=\axisdefaultwidth,
                                height=0.7*\axisdefaultheight]
                                \addplot[blue, ultra thick, mark=none] table {../data/fig/norm1/c.dat};
                            \end{axis}
                        \end{tikzpicture}
                    \end{tabular}
                }
            \end{tabular}
        }
    \end{center}
    \caption{The same two time series $Q$ and $C$ from figure \ref{fig:dynamictimewarping} normalized by $\eta$. On the
    left plot are both time series graphs, the gray lines are representing the warping path of plain DTW. The right plot
    shows the associated matrix containing the distances between the time series data points. Starting in the lower left
    corner and ending in the upper right corner, the warping path is illustrated as a white graph.}
    \label{fig:norm1}
\end{figure}

The second time series normalization $\eta '$ transforms every data point of the given time series $Q$ as follows.
\begin{equation}
    \eta '(q) = \frac{\eta (q)}{\sigma}
\end{equation}
where $\sigma$ is the standard deviation of the time series $Q$
\begin{equation}
    \sigma = \frac{1}{l-1} \sum \limits_{i=1}^{l} d(q_i, \bar{q})^2
\end{equation}
Figure \ref{fig:norm2} shows the impact of the $\eta '$ normalization on DTW compared to plain DTW in
figure \ref{fig:dynamictimewarping}.

\begin{figure}
    \begin{center}
        \resizebox {\textwidth} {!} {
            \begin{tabular}{cc}
                \resizebox* {!} {0.3\textwidth} {
                    \begin{tikzpicture}
                        \begin{axis}[
                            xmin=0,
                            xmax=47,
                            xlabel=time,
                            ylabel=acceleration,
                            width=\axisdefaultwidth,
                            height=0.7*\axisdefaultheight,
                            reverse legend,
                            legend pos=south east]
                            \addplot[gray, quiver={u=\thisrow{u}, v=\thisrow{v}}] table {../data/fig/norm2/path.dat};
                            \addplot[red, thick, mark=none] table {../data/fig/norm2/q.dat};
                            \addlegendentry{Q}
                            \addplot[blue, thick, mark=none] table {../data/fig/norm2/c.dat};
                            \addlegendentry{C}
                        \end{axis}
                    \end{tikzpicture}
                } & \quad
                \resizebox* {!} {0.3\textwidth} {
                    \begin{tabular}[b]{ll}
                        \begin{turn}{90}
                            \begin{tikzpicture}
                                \begin{axis}[
                                    xmin=0,
                                    xmax=47,
                                    ymin=-2,
                                    ymax=2,
                                    hide x axis,
                                    hide y axis,
                                    width=\axisdefaultwidth,
                                    height=0.7*\axisdefaultheight]
                                    \addplot[red, ultra thick, mark=none] table {../data/fig/norm2/q.dat};
                                \end{axis}
                            \end{tikzpicture}
                        \end{turn} \hspace*{3em} &
                        \begin{tikzpicture}
                            \begin{axis}[
                                enlargelimits=false,
                                ymin=0,
                                ymax=47,
                                hide x axis,
                                hide y axis,
                                width=\axisdefaultwidth,
                                height=\axisdefaultwidth,
                                colorbar,
                                colormap/viridis high res]
                                \addplot[matrix plot*,
                                    mesh/cols=48,
                                    point meta=explicit] table[meta=C] {../data/fig/norm2/matrix.dat};
                                \addplot[white, ultra thick, mark=*, mark size=1] table {../data/fig/norm2/matrix_path.dat};
                            \end{axis}
                        \end{tikzpicture}\\
                        &
                        \\[1em]
                        &
                        \begin{tikzpicture}
                            \begin{axis}[
                                xmin=0,
                                xmax=47,
                                ymin=-2,
                                ymax=2,
                                hide x axis,
                                hide y axis,
                                width=\axisdefaultwidth,
                                height=0.7*\axisdefaultheight]
                                \addplot[blue, ultra thick, mark=none] table {../data/fig/norm2/c.dat};
                            \end{axis}
                        \end{tikzpicture}
                    \end{tabular}
                }
            \end{tabular}
        }
    \end{center}
    \caption{The same two time series $Q$ and $C$ from figure \ref{fig:dynamictimewarping} normalized by $\eta '$. On the
    left are plot both time series graphs, the gray lines are representing the warping path of plain DTW. The right plot
    shows the associated matrix containing the distances between the time series data points. Starting in the lower left
    corner and ending in the upper right corner, the warping path is illustrated as a white graph.}
    \label{fig:norm2}
\end{figure}
