\subsection{Complexity-Invariant Distance Measure} \label{complexity-invariant_distance_measure}
The Complexity-Invariant Distance measure \cite{batista2011complexity} (CID) is a distance measure for time
series. It is composed of the ED and a Complexity Correction Factor (CF). Given are two time series
$Q = (q_1, q_2, \dots, q_i, \dots, q_l)$, $C = (c_1, c_2, \dots, c_j, \dots, c_l)$ with the same length $l$ over the
domain set $\mathbb{U}$ and a distance measure function $D$ with $D: \mathbb{U} \times \mathbb{U} \to \mathbb{R}$.
The CID of two time series $Q$ and $C$ can be calculated by.
\begin{center}
    $CID(Q, C) = ED(Q, C) \times CF(Q, C)$
\end{center}
The CF of two time series $Q$ and $C$ can be calculated by.
\begin{center}
     $CF(Q, C) = \frac{max(CE(Q), CE(C))}{min(CE(Q), CE(C))}$
\end{center}
CF is always a value greater or equal than 1. The more $CE(Q)$ is different from $CE(C)$, the more $CF(Q, C)$ is
increasing and the distance $CID(Q, C)$ between the time series is increasing. The Complexity Estimate (CE) is a measure
for time series and the authors of \textit{A Complexity-Invariant Distance Measure for Time Series}
\cite{batista2011complexity} introduced one possible approach of a CE implementation for a time series $Q$.
\begin{center}
    $CE(Q) = \sqrt[2]{\sum \limits_{i=1}^{l-1} D(q_i, q_{i + 1})^2}$
\end{center}

The advantages and disadvantage of CID are inherited by ED. Time and space complexity are $\mathcal{O}(l)$, however the
algorithm itself has even only a space complexity of $\mathcal{O}(1)$ if the space of the input time series is ignored.
Comparing two time series of different length with the CID presupposes to transform the time series to the same length.

\subsubsection{Complexity-Invariant Dynamic Time Warping} \label{complexity-invariant_dynamic_time_warping}
Strictly speaking is the CF himself a similarity measure for time series, however certainly better suited as factor in
combination with an other similarity measure. Complexity-Invariant Dynamic Time Warping (CIDDTW) is the combination of
CF with DTW \cite{batista2011complexity}.

Given are two time series $Q = (q_1, q_2, \dots, q_i, \dots, q_l)$ with length $l$,
$C = (c_1, c_2, \dots, c_j, \dots, c_k)$ with length $k$ over the domain set $\mathbb{U}$ and a distance measure
function $D$ with $D: \mathbb{U} \times \mathbb{U} \to \mathbb{R}$. The CIDDTW distance between time series $Q$ and $C$
can be calculated with the following formula.

\begin{center}
    $CIDDTW(Q, C) = DTW(Q, C) \times CF(Q, C)$
\end{center}

