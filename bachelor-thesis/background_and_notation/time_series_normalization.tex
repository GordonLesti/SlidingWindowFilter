\subsection{Time Series Normalization} \label{time_series_normalization}
\cite{keogh2003need}

\begin{figure}
    \begin{center}
        \begin{tabular}{cc}
            \resizebox {0.6\textwidth} {!} {
                \begin{tikzpicture}
                    \begin{axis}[
                        xmin=0,
                        xmax=47,
                        xlabel=time,
                        ylabel=acceleration,
                        width=\textwidth,
                        height=\axisdefaultheight,
                        reverse legend]
                        \addplot[lightgray, quiver={u=\thisrow{u}, v=\thisrow{v}}] table {../data/fig/norm1/path.dat};
                        \addplot[red, thick, mark=none] table {../data/fig/norm1/q.dat};
                        \addlegendentry{Q}
                        \addplot[blue, thick, mark=none] table {../data/fig/norm1/c.dat};
                        \addlegendentry{C}
                    \end{axis}
                \end{tikzpicture}
            } & \qquad
            \resizebox {0.3\textwidth} {!} {
                \begin{tabular}{ll}
                    &
                    \\[-1.5\textwidth]
                    &
                    \begin{tikzpicture}
                        \begin{axis}[
                            xmin=0,
                            xmax=47,
                            ymin=-40,
                            ymax=40,
                            hide x axis,
                            hide y axis,
                            width=\textwidth,
                            height=\axisdefaultheight]
                            \addplot[red, ultra thick, mark=none] table {../data/fig/norm1/q.dat};
                        \end{axis}
                    \end{tikzpicture} \\
                    \begin{turn}{90}
                        \begin{tikzpicture}
                            \begin{axis}[
                                xmin=0,
                                xmax=47,
                                ymin=-40,
                                ymax=40,
                                hide x axis,
                                hide y axis,
                                width=\textwidth,
                                height=\axisdefaultheight]
                                \addplot[blue, ultra thick, mark=none] table {../data/fig/norm1/c.dat};
                            \end{axis}
                        \end{tikzpicture}
                    \end{turn} &
                    \begin{tikzpicture}
                        \begin{axis}[
                            enlargelimits=false,
                            ymin=0,
                            ymax=47,
                            hide x axis,
                            hide y axis,
                            width=\textwidth,
                            height=\textwidth,
                            colorbar,
                            colormap/viridis high res]
                            \addplot[matrix plot*,
                                mesh/cols=48,
                                point meta=explicit] table[x=y, y=x, meta=C] {../data/fig/norm1/matrix.dat};
                            \addplot[white, ultra thick, mark=*] table[x=y, y=x] {../data/fig/norm1/matrix_path.dat};
                        \end{axis}
                    \end{tikzpicture}
                \end{tabular}
            }
        \end{tabular}
    \end{center}
    \caption{Two time series $Q$ and $C$ containing recorded and compressed data from one acceleration sensor. On the
    left plot both time series graphs, the gray lines are representing the warping path of plain DTW. The right plot
    shows the associated matrix containing the distances between the time series data points. Starting in the lower left
    corner and ending in the upper right corner, the warping path is illustrated as white graph.}
    \label{fig:norm1}
\end{figure}
