\section{Conclusion and Future Work} \label{conclusion_and_future_work}
The results of the experiment are showing that the sliding window filter can block a huge amount of useless 1NN-DTW
calls without losing accuracy. It is possible to block more than 13\% of the 1NN-DTW calls without losing accuracy. By
losing just a not noticeable amount of accuracy it is possible to block more than 39\% of the 1NN-DTW calls. Finding the
right configuration for a sliding window application and the included filter was the biggest challenge. The dominating
configuration has worked very well for the domain of accelerometer based gesture detection. More than 72\% of the
gestures have been found right and only less than 4\% of the gestures have been wrongly detected under the best
configuration. Impressive numbers under an amendable experiment setup.

To rerun the experiment with an improved setup would probably result in more impressive consequences. Maybe not again
with accelerometer based gesture detection. The setup was pretty time consuming for a huge amount of records during a
bachelor thesis. Maybe something web based as mentioned in this
post\footnote{https://gordonlesti.com/touch-signature-identification-with-javascript/}. This has the advantage that more
experimentees would participate.

Rerunning the evaluation of this experiment with data by other domains would be interesting to see if the approach of a
sliding window filter works also in other typical time series areas.

The time series filter as presented in this thesis has been used only to block perhaps unclassifiable time series
windows to reduce the execution of a cost intensive nearest neighbour classificator. That happend on the basis of the
filter interval over all items of the training data independent from the different classes. It should be possible to
create filter intervals for every class in the training set and to prune classes as candidate for the classificator.
