\documentclass[runningheads,a4paper]{llncs}

\usepackage{amsmath}
\usepackage{amssymb}
\usepackage{float}
\usepackage{pgfplots}
\usepackage{tabularx}
\usepackage{textcomp}
\usepackage{url}
\newcommand{\keywords}[1]{\par\addvspace\baselineskip
\noindent\keywordname\enspace\ignorespaces#1}
\pgfplotsset{compat=1.11}
\setcounter{tocdepth}{3}
\setcounter{secnumdepth}{3}
\usetikzlibrary{spy}

\begin{document}
    \graphicspath{ {../img/} }
    \mainmatter
    \title{A Sliding Window Filter for Time Series Streams}
    \subtitle{Bachelor thesis\\
    \textnormal{\small{Supervisor: Stephan Spiegel}}}
    \titlerunning{A Sliding Window Filter for Time Series Streams}
    \author{Gordon Lesti\\
    313249\\
    Course of studies: Bachelor of Computer Science\\
    gordon.lesti@campus.tu-berlin.de\\}
    \authorrunning{Gordon Lesti}
    \institute{Technische Universit\"at Berlin\\
    Fakult\"at IV Elektrotechnik und Informatik\\
    Fachgebiet AOT\\
    Prof. Dr. Sahin Albayrak\\
    \url{http://www.aot.tu-berlin.de/}}
    \toctitle{A Sliding Window Filter for Time Series Streams}
    \tocauthor{Gordon Lesti}
    \maketitle

    \begin{abstract}
A lot of devices with various sensors in our everyday life are producing constantly a diverse kind of data. Obvious
examples are smartphones, wearable devices like smartwatches and fitness tracker braclets. A considerable part of the
produced data comes continuously and can be interpreted as time series streams. For example acceleration or vitality
data. A real time evaluation of the time series stream presupposes possibly a constantly running nearest neighbour
classification on the most recent time series window with a distance measure like dynamic time warping.

This bachelor thesis explains the approach of a filter with linear complexity for time series in front of a nearest
neighbour classification for time series. The added value of a filter with cheap running time is to reduce the execution
of a expensive nearest neighbour classification in the case of unclassifiable time series windows. On results of an
experimental example with gesture detection via acceleration data will be shown that filters can reduce the execution of
nearest neighbour classifications without great loss of accuracy.
\keywords{Time Series, Distance Measures, Classification}
\end{abstract}

    \tableofcontents
    \newpage
    \section{Introduction}
The scenario is an application that gets constantly data in form of a time series stream from one or more sensors. The
application itself has in general no interest in a long term evaluation of the data. Only the most recent time series
window has to be classified in the evaluation. The result of the evaluation will be stored or processed by an other
application, the time series window moves on and has to be evaluated again. This process is repeating continuously.
Dynamic Time Warping (DTW) in combination with a 1-Nearest-Neighbour (1NN) classification for the evaluation is an
obvious approach for such a scenario. The downside, this approach is computationally too demanding for many realtime
applications \cite{xi2006fast}. This disadvantage becomes even more tragic under the assumption that perhaps a large
amout of incoming time series windows may cross a certain threshold to their nearest neighbour to be classified. Shall
mean that not every incoming time series window can be matched to a class.

This bachelor thesis explains the approach of a filter for time series in front of the 1NN in combination with DTW
(1NN-DTW) to reduce the execution of 1NN-DTW on unclassifiable time series windows. The condition for such a filter is
linear complexity.

    \section{Background and Notation} \label{background_and_notation}

This section gives more background on the sliding window technique \cite{keogh2004sliding}, DTW distance measure \cite{keogh2002exact}, and time series normalization \cite{das1998rule}, which are fundamental building blocks of our conducted online gesture recognition study \cite{lesti2017filter}. Table \ref{tab:notation} introduces the notation that we use for formal problem description.

\begin{table}
    \begin{center}
        \begin{tabularx}{\textwidth}{c X}
            \hline
            \textbf{Symbol} \qquad & \textbf{Description}\\
            \hline
            $Q$ & a time series of size $n$ with $Q = (q_1, q_2, \dots, q_i, \dots, q_n)$\\
            $Q[i,j]$ & a subsequence time series of $Q$ with $Q[i,j] = (q_i, q_{i+1}, \dots, q_{j})$\\
            $t$ & the current time\\
            $\mu$ & the mean of a time series $Q$\\
            $\sigma$ & the standard deviation of a time series $Q$\\
            $\eta$, $z$  & two different time series normalizations\\
            \hline
        \end{tabularx}
    \end{center}
    \caption{Notation used for formal problem description.}
	\label{tab:notation}
\end{table}

\subsection{Sliding Window Technique} \label{sliding_window_technique}
Given is a continuous time series data stream $Q$. The last subsequence of size $w$ is in the interest of the sliding
window technique. All subsequences or time series data points of $Q$ that are back even further as the window size $w$
are in this moment irrelevant. The sliding window application tries to classify this last subsequence of $Q$ with the
help of a time series classificator. This time series classificator makes its decisions based on a given time series
training set. The application triggers an event in the case of a positive classification of the subsequence and an other
application can react on this event.

The sliding window application remains idle until the time series data stream $Q$
grows. Afterwards, the process is repeated again. However, the sliding window application waits until the time series
data stream $Q$ has grown by a predefined step size $s$. If necessary, this stepwise approach can avoid the execution of
the process continuously after every new data point in $Q$. Figure \ref{fig:swt} illustrates the described process.

\tikzstyle{decision} = [diamond, draw, aspect=2, fill=white!20, text width=8em, text badly centered, node distance=2cm, inner sep=0pt]
\tikzstyle{block} = [rectangle, draw, fill=white!20, text width=5em, text centered, minimum height=4em]
\tikzstyle{line} = [draw, -latex']

\begin{figure}
    \begin{center}
        \resizebox {\textwidth} {!} {
            {\tiny
                \begin{tikzpicture}[node distance = 1.5cm, auto]
                    \node [block] (sod) {sensors or devices};
                    \node [block, right of=sod, node distance=6cm, text width=2cm] (extract) {Extract last subsequence from Q of size $w$, $Q[t-w,t]$};
                    \node [block, right of=extract, node distance=4cm, text width=2cm] (nnc) {Time series classificator};
                    \node [decision, below of=nnc] (decide) {$Q[t-w,t]$ classifiable?};
                    \node [block, left of=decide, node distance=3cm] (sleeps) {Sleep for $s$ time};
                    \node [block, below of=decide, node distance=2cm, text width=2cm] (action) {Trigger event that $Q[t-w,t]$ has been classified and sleep for $w$ time};

                    \path [line,dashed] (sod) -- node (ctss) {Continuous time series stream $Q$} (extract);
                    \path [line] (extract) -- node {$Q[t-w,t]$} (nnc);
                    \path [line] (nnc) -- (decide);
                    \path [line] (decide) -- node {no} (sleeps);
                    \path [line,dashed] (sleeps) -| (ctss);
                    \path [line] (decide) -- node {yes} (action);
                    \path [line,dashed] (action) -| (ctss);
                \end{tikzpicture}
            }
        }
    \end{center}
    \caption{Possible design for a sliding window application. The current time is stored in variable $t$. The constant
    variables $w$ for the window size and $s$ for the step size are predefined.}
    \label{fig:swt}
\end{figure}

The explained sliding window technique uses a generic time series classificator. However, in this bachelor thesis the
generic time series classificator is replaced by 1NN-DTW. A time series window $Q[t-w,t]$ is classified as instance of
class $K_i$ by 1NN-DTW if the DTW distance between $Q[t-w,t]$ and the nearest neighbour passes a predefined
class threshold $\epsilon_i$. The following subsection gives some background of DTW.

\subsection{Dynamic Time Warping} \label{dynamic_time_warping}
Dynamic Time Warping (DTW) is a widely used and robust distance measure for time series, \textit{allowing similar shapes
to match even if they are out of phase in the time axis} \cite{keogh2002exact}. The following explaination to calculate
the DTW distance is based on \cite{sart2010accelerating}.

Given are two time series $Q = (q_1, q_2, \dots, q_i, \dots, q_l)$ with length
$l$, $C = (c_1, c_2, \dots, c_j, \dots, c_k)$ with length $k$ over the domain set $\mathbb{U}$ and a distance measure
function $D$ with $D: \mathbb{U} \times \mathbb{U} \to \mathbb{R}$. Calculating the DTW distance between the two time
series $Q$ and $C$ can be achieved by calculating a matrix $M$ of size $l \times k$ with the following rule.
\begin{center} \[ M_{i, j} = \begin{cases}
    D(q_i,c_j) & \text{if } i = 1 \wedge j = 1\\
    M_{i,j-1} + D(q_i,c_j) & \text{if } i = 1 \wedge j \neq 1\\
    M_{i-1,j} + D(q_i,c_j) & \text{if } i \neq 1 \wedge j = 1\\
    min(M_{i-1,j}, M_{i-1,j-1}, M_{i,j-1}) + D(q_i,c_j) & \text{if } i \neq 1 \wedge j \neq 1
\end{cases} \] \end{center}
The DTW distance between the two time series $Q$ and $C$ is the entry $M_{l,k}$ of the resulting matrix.
\begin{center}
    $DTW(Q, C) = M_{l,k}$
\end{center}
The detection of the warping path as result of the backtracking is irrelevant for the aim of this bachelor thesis.
Figure \ref{fig:dynamictimewarping} illustrates DTW for two time series $Q$ and $C$ that contain recorded data from one
acceleration sensor. DTW shown as above has time and space complexity of $\mathcal{O}(lk)$. When ignoring the
warping path as result the algorithm can easely reduce the space complexity to $\mathcal{O}(min(l, k))$. This can be
achieved by keeping only the last important enties in space that are necessary to calculate the final entriy $M_{l,k}$
of the matrix.

\begin{figure}
    \begin{center}
        \begin{tabular}{cc}
            \resizebox {0.57\textwidth} {!} {
                \begin{tikzpicture}
                    \begin{axis}[
                        xmin=0,
                        xmax=47,
                        xlabel=time,
                        ylabel=acceleration,
                        width=\textwidth,
                        height=\axisdefaultheight,
                        reverse legend]
                        \addplot[lightgray, quiver={u=\thisrow{u}, v=\thisrow{v}}] table {../data/fig/dynamictimewarping/path.dat};
                        \addplot[red, thick, mark=none] table {../data/fig/dynamictimewarping/q.dat};
                        \addlegendentry{Q}
                        \addplot[blue, thick, mark=none] table {../data/fig/dynamictimewarping/c.dat};
                        \addlegendentry{C}
                    \end{axis}
                \end{tikzpicture}
            } & \quad
            \resizebox {0.33\textwidth} {!} {
                \begin{tabular}{ll}
                    &
                    \\[-1.55\textwidth]
                    \begin{turn}{90}
                        \begin{tikzpicture}
                            \begin{axis}[
                                xmin=0,
                                xmax=47,
                                ymin=-100,
                                ymax=0,
                                hide x axis,
                                hide y axis,
                                width=\textwidth,
                                height=\axisdefaultheight]
                                \addplot[red, ultra thick, mark=none] table {../data/fig/dynamictimewarping/q.dat};
                            \end{axis}
                        \end{tikzpicture}
                    \end{turn} \hspace*{3em} &
                    \begin{tikzpicture}
                        \begin{axis}[
                            enlargelimits=false,
                            ymin=0,
                            ymax=47,
                            hide x axis,
                            hide y axis,
                            width=\textwidth,
                            height=\textwidth,
                            colorbar,
                            colormap/viridis high res]
                            \addplot[matrix plot*,
                                mesh/cols=48,
                                point meta=explicit] table[meta=C] {../data/fig/dynamictimewarping/matrix.dat};
                            \addplot[white, ultra thick, mark=*] table {../data/fig/dynamictimewarping/matrix_path.dat};
                        \end{axis}
                    \end{tikzpicture}\\
                    &
                    \\[1em]
                    &
                    \begin{tikzpicture}
                        \begin{axis}[
                            xmin=0,
                            xmax=47,
                            ymin=-100,
                            ymax=0,
                            hide x axis,
                            hide y axis,
                            width=\textwidth,
                            height=\axisdefaultheight]
                            \addplot[blue, ultra thick, mark=none] table {../data/fig/dynamictimewarping/c.dat};
                        \end{axis}
                    \end{tikzpicture}
                \end{tabular}
            }
        \end{tabular}
    \end{center}
    \caption{Two time series $Q$ and $C$ containing recorded and compressed data from one acceleration sensor. On the
    left plot both time series graphs, the gray lines are representing the warping path of plain DTW. The right plot
    shows the associated matrix containing the distances between the time series data points. Starting in the lower left
    corner and ending in the upper right corner, the warping path is illustrated as white graph.}
    \label{fig:dynamictimewarping}
\end{figure}

\subsubsection{Sakoe-Chiba Band} \label{sakoe-chiba_band}
\cite{sakoe1978dynamic}

\subsection{Time Series Normalization} \label{time_series_normalization}
\cite{keogh2003need}



    \section{Sliding Window Filter} \label{sliding_window_filter}
A scenario as described in section \ref{introduction} is given.

    \subsection{Experiment} \label{experiment}

The proposed sliding window filter has several model parameters that need to be carefully tuned in order to achieve optimal performance. Depending on the application domain we need to select an appropriate window and step size, time series normalization, dissimilarity threshold, and filter criterion. In the following we describe all parameter settings that were assessed in our empirical study: 

\begin{itemize}

\item 
The \textbf{window size} determines the number of most recent measurements contained in the examined time series subsequences. We tested four different sizes that were learned from the training gesture, including \textbf{min}, \textbf{max}, and \textbf{avg} length as well as the \textbf{mid}-point of the range. \\
      
\item 
The \textbf{step size} defines the gap between consecutive time series windows. As default setting we use one tenth of the window size. \\
        
\item 
For online gesture recognition we employ the nearest neighbor classifier in combination with the DTW distance, where we evaluate 34 different Sakoe-Chiba \textbf{band} sizes, ranging from 2 \% to 200 \%. 
Prior to pair-wise comparing sliding windows and training gestures, the corresponding time series should be normalized. We evaluate $\eta$, $z$, and no \textbf{normalization}. \\

\item 
The dissimilarity \textbf{threshold} defines the time series distance at which a sliding window and a training gesture are considered to belong to the same class. We determine the threshold for an individual class by measuring the distances between all samples of that particular class and all instances of other classes. In our empirical study we evaluate the threshold influence for: (i) one half of the minimum distance - \textbf{HMinD}, (ii) one half of the average distance - \textbf{HAvgD}, and (iii) one half of the midpoint distance - \textbf{HAvgD}. \\  

\item 
The \textbf{filter criterion} is an essential part of our proposed approach. In our empirical study we evaluate the performance of the two filter criteria, namely the sample variance \textbf{VAR} and the length normalized complexity estimate \textbf{LNCE} of a time series. Both filters are tested with different factors that increase the size of the filter interval from 100 \% to 300 \%.

\end{itemize}

Figure \ref{fig:experiment} visualizes the online gesture recognition results for a sample time series stream processed by our proposed sliding window filter, 
after selecting the above described model parameters with help of the recorded training gestures.

\begin{figure}
    \resizebox {\textwidth} {!} {
        \begin{tikzpicture}
            \begin{axis}[
                xmin=0,
                xmax=2426,
                ymin=-16,
                ymax=16,
                width=10*\axisdefaultwidth,
                height=\axisdefaultheight,
                xticklabels={,,},
                yticklabels={,,}]
                \addplot[blue, mark=none, opacity=0.4] table[x=t, y=x] {../data/fig/experimentee_result2/exp1.dat};
                \addplot[red, mark=none, opacity=0.4] table[x=t, y=y] {../data/fig/experimentee_result2/exp1.dat};
                \addplot[green, mark=none, opacity=0.4] table[x=t, y=z] {../data/fig/experimentee_result2/exp1.dat};
                \addplot+[fill, opacity=0.5, red, mark=none] coordinates {(294, -16) (307, -16) (307, 16) (294, 16)} --cycle;
                \addplot+[fill, opacity=0.5, green, mark=none] coordinates {(307, -16) (357, -16) (357, 16) (307, 16)} --cycle;
                \addplot+[fill, opacity=0.5, red, mark=none] coordinates {(357, -16) (359, -16) (359, 16) (357, 16)} --cycle;
                \addplot+[fill, opacity=0.5, red, mark=none] coordinates {(497, -16) (508, -16) (508, 16) (497, 16)} --cycle;
                \addplot+[fill, opacity=0.5, green, mark=none] coordinates {(508, -16) (562, -16) (562, 16) (508, 16)} --cycle;
                \addplot+[fill, opacity=0.5, blue, mark=none] coordinates {(562, -16) (564, -16) (564, 16) (562, 16)} --cycle;
                \addplot+[fill, opacity=0.5, red, mark=none] coordinates {(712, -16) (722, -16) (722, 16) (712, 16)} --cycle;
                \addplot+[fill, opacity=0.5, green, mark=none] coordinates {(722, -16) (777, -16) (777, 16) (722, 16)} --cycle;
                \addplot+[fill, opacity=0.5, blue, mark=none] coordinates {(777, -16) (778, -16) (778, 16) (777, 16)} --cycle;
                \addplot+[fill, opacity=0.5, blue, mark=none] coordinates {(940, -16) (945, -16) (945, 16) (940, 16)} --cycle;
                \addplot+[fill, opacity=0.5, green, mark=none] coordinates {(945, -16) (1010, -16) (1010, 16) (945, 16)} --cycle;
                \addplot+[fill, opacity=0.5, blue, mark=none] coordinates {(1010, -16) (1023, -16) (1023, 16) (1010, 16)} --cycle;
                \addplot+[fill, opacity=0.5, red, mark=none] coordinates {(1310, -16) (1316, -16) (1316, 16) (1310, 16)} --cycle;
                \addplot+[fill, opacity=0.5, green, mark=none] coordinates {(1316, -16) (1367, -16) (1367, 16) (1316, 16)} --cycle;
                \addplot+[fill, opacity=0.5, red, mark=none] coordinates {(1367, -16) (1375, -16) (1375, 16) (1367, 16)} --cycle;
                \addplot+[fill, opacity=0.5, red, mark=none] coordinates {(1681, -16) (1689, -16) (1689, 16) (1681, 16)} --cycle;
                \addplot+[fill, opacity=0.5, green, mark=none] coordinates {(1689, -16) (1746, -16) (1746, 16) (1689, 16)} --cycle;
                \addplot+[fill, opacity=0.5, blue, mark=none] coordinates {(1746, -16) (1748, -16) (1748, 16) (1746, 16)} --cycle;
                \addplot+[fill, opacity=0.5, red, mark=none] coordinates {(2082, -16) (2090, -16) (2090, 16) (2082, 16)} --cycle;
                \addplot+[fill, opacity=0.5, green, mark=none] coordinates {(2090, -16) (2146, -16) (2146, 16) (2090, 16)} --cycle;
                \addplot+[fill, opacity=0.5, red, mark=none] coordinates {(2146, -16) (2147, -16) (2147, 16) (2146, 16)} --cycle;
                \addplot+[fill, opacity=0.5, blue, mark=none] coordinates {(2311, -16) (2388, -16) (2388, 16) (2311, 16)} --cycle;
                \addplot+[fill, opacity=0.5, red, mark=none] coordinates {(2297, -16) (2362, -16) (2362, 16) (2297, 16)} --cycle;
            \end{axis}
        \end{tikzpicture}
    }
    \caption{Visualized results of online gesture recognition for a sample time series stream. We highlight true positives in green, false positives in red, false negatives in blue, and true negatives in transparent. 
    Although we see short false detection intervals before or after true positives, seven out of eight gestures were assigned to the correct class label.}
    \label{fig:experiment}
\end{figure}

    \section{Conclusion and Future Work} \label{conclusion_and_future_work}
The results of the experiment are showing that the sliding window filter can block a huge amount of useless 1NN-DTW
calls without losing accuracy. During the experiment it was possible to block more than 13\% of the 1NN-DTW calls without losing accuracy. By
losing just a not noticeable amount of accuracy it was possible to block more than 39\% of the 1NN-DTW calls. Finding the
right configuration for a sliding window application and the included filter was the biggest challenge. The dominating
configuration has worked very well for the domain of accelerometer based gesture detection. More than 72\% of the
gestures were found right and only less than 4\% of the gestures were wrongly detected under the best
configuration. Impressive numbers under an amendable experiment setup.

To rerun the experiment with an improved setup would probably result in more impressive consequences. Maybe a rerun of the experiment should not work again
with accelerometer based gesture detection. The setup was pretty time consuming for a huge amount of records during a
bachelor thesis. Maybe something web based as mentioned in this
post\footnote{https://gordonlesti.com/touch-signature-identification-with-javascript/} would be more effective. This has the advantage that more
experimentees would participate.

Rerunning the evaluation of this experiment with data by other domains would be interesting. Hopefully the approach of a
sliding window filter works also in other typical time series areas.

The time series filter as presented in this thesis was blocking perhaps unclassifiable time series
windows to reduce the execution of a cost intensive nearest neighbour classificator. That happend on the basis of the
filter interval over all items of the training data independent from the different classes. It should be possible to
create filter intervals for every class in the training set and to prune classes as candidate for the classificator.

    \newpage
    \begin{thebibliography}{1}
    \bibitem{berndt1994using} Berndt, Donald J and Clifford, James "Using Dynamic Time Warping to Find Patterns in
    Time Series." KDD workshop (1994): 359--370
    \bibitem{batista2011complexity} Batista, Gustavo EAPA and Wang, Xiaoyue and Keogh, Eamonn J "A
    Complexity-Invariant Distance Measure for Time Series." SDM (2011): 699--710
    \bibitem{liu2009uwave} Liu, Jiayang and Zhong, Lin and Wickramasuriya, Jehan and Vasudevan, Venu "uWave:
    Accelerometer-based personalized gesture recognition and its applications" Pervasive and Mobile Computing
    (2009): 657--675
\end{thebibliography}

\end{document}
