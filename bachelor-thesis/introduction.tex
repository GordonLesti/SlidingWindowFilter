\section{Introduction} \label{introduction}
The scenario used for this thesis is an application which constantly receives data in form of a time series stream from
one or more sensors. The application itself does not have a general interest in a long term evaluation of the data. Only
the most recent time series window has to be classified in the evaluation. The result of the evaluation can be stored or
processed by an other application, the time series window moves on and has to be evaluated again. This process is
continuously repeated with every times series window being readable only once. Dynamic Time Warping (DTW) in combination
with a 1-Nearest-Neighbour (1NN) classification for the evaluation is an obvious approach for such a scenario. The
disadvantage is that this approach is computationally too demanding for many realtime applications \cite{xi2006fast}.
This disadvantage becomes even more tragic under the assumption that perhaps a large amout of incoming time series
windows is unclassifiable due to a too large distance to the nearest neighbour.

This bachelor thesis explains the approach of a sliding window filter for time series ahead of the 1NN in combination
with DTW (1NN-DTW) to reduce the execution of 1NN-DTW on unclassifiable time series windows. The condition for such a
filter is linear complexity. Furthermore the combination of the filter ahead of the 1NN-DTW should perform with a
similar accuracy as 1NN-DTW on its own.

An experiment that simulates an above described scenario was carried out to expose the additional benefit of a filter
ahead of 1NN-DTW under the given conditions.%TODO double of

The bachelor thesis is organized as follows. Section \ref{background_and_notation} will explain the sliding window
technique and basics around time series similarity measures that are used in this bachelor thesis. Furthermore
the sliding window filter is described at the end of the same section. The description of the experiment and the
evaluation of the results are contained in section \ref{experiment}. Conclusions are offered and future work
is suggested in section \ref{conclusion_and_future_work}.
