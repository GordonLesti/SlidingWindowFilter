\section{Introduction} \label{introduction}
The scenario is an application that gets constantly data in form of a time series stream from one or more sensors. The
application itself has in general no interest in a long term evaluation of the data. Only the most recent time series
window has to be classified in the evaluation. The result of the evaluation will be stored or processed by an other
application, the time series window moves on and has to be evaluated again. This process is repeating continuously and
every time series window can be read only once. Dynamic Time Warping (DTW) in combination with a 1-Nearest-Neighbour
(1NN) classification for the evaluation is an obvious approach for such a scenario. The downside, this approach is
computationally too demanding for many realtime applications \cite{xi2006fast}. This disadvantage becomes even more
tragic under the assumption that perhaps a large amout of incoming time series windows may cross a certain threshold to
their nearest neighbour to be classified. Shall mean that not every incoming time series window can be matched to a
class.

This bachelor thesis explains the approach of a Sliding Window Filter for time series in front of the 1NN in combination
with DTW (1NN-DTW) to reduce the execution of 1NN-DTW on unclassifiable time series windows. The condition for such a
filter is linear complexity. Furthermore should the combination of the filter in front of the 1NN-DTW perform with a
similar accuracy as 1NN-DTW stand alone.

An experiment that simulates an above described scenario was carried out to expose the surplus value
of a filter in front of 1NN-DTW under the given conditions.

The rest of this bachelor thesis is organized  as follows. Section \ref{background_and_notation} will focus on basics
around time series and similarity measures that are used in this bachelor thesis. The Sliding Window Filter will be
explained in section \ref{sliding_window_filter}. Section \ref{experiment} contains the description of the experiment
and the evaluation of the results. Conclusions will be offered and future work suggested in section
\ref{conclusion_and_future_work}.
