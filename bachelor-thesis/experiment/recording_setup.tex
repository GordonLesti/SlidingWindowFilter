\subsection{Recording Setup} \label{recording_setup}
A Wii Remote\texttrademark~Plus controller is able to observe acceleration data in three axis via sensors. Those
acceleration data plus all button events on the controller can be send via
Bluetooth\footnote{https://www.bluetooth.com/} to another device. The other device was a computer that has executed the
recording software. Via a graphical user interface or shorten GUI the recording software instructed the experimentee to
perform gestures or physical activities. The experiment was a sequence of fullscreen slides that changed by performing a
gesture with the controller. Section \ref{data_recording_instructions} explains the specific sequence of slides. The
recording software observed all acceleration data and \textit{B} button events during the whole experiment and saved it
to a file. Marking the beginning and the end of a gesture has been done with the \textit{B} button. The \textit{B}
button is on the bottom of the controller and triggered by the forefinger.

The recording software has been implemented in Python\footnote{https://www.python.org/} and is available at
GitHub\footnote{https://github.com/GordonLesti/SlidingWindowFilter-experiment}. The main requirements for the recording
software are the open-source device driver for Nintendo Wii remotes,
xwiimote\footnote{https://github.com/dvdhrm/xwiimote} by David Herrmann and the language bindings for the xwiimote
package, xwiimote-bindings\footnote{https://github.com/dvdhrm/xwiimote-bindings} by Nicolas Adenis-Lamarre. The last
requirement was the main reason to implement the recording software in Python. For the GUI was choosen
TkInter\footnote{https://wiki.python.org/moin/TkInter}.

The first version of the recording software had the major problem to control the GUI and the
processing of the Bluetooth data in one loop with a linear flow. That caused a stagnation in the Bluetooth data queue
while loading new images in the GUI. This problem could be solved by two communicating threads and a preloading of the
images. Unfortunately, all existing records become unusable for the evaluation.

\subsubsection{File Format} \label{file_format}
Interesting for the evaluation of the experiment are the acceleration data in three axis and the \textit{B} button down
and up events to mark the beginning and the end of a gesture. This results in three different kinds of events, incoming
acceleration data, the \textit{B} button is pressed down and the \textit{B} button is released up. The controller is
sending the acceleration data with an average gap of five milliseconds. Every event has been marked with the time
in milliseconds that have passed since the starting of the experiment. The representing line in a recorded file for a
acceleration data event is starting with the time in milliseconds followed by the acceleration data of all three axis.
The acceleration data of a axis is transfered as integer value with the unit decimetre per second squared. A \textit{B}
button down event is represented by a line also starting with the time in milliseconds followed by the keyword
\textit{START} and a counter for the gesture. The \textit{B} button up event is the same with the keyword \textit{END}
instead of \textit{START}. An example excerpt of a recorded file can look like the following.

\medskip
\noindent
{\it Example excerpt of a recorded file}
\begin{verbatim}
3045 20 19 74
3050 16 12 70
3055 START 1
3055 14 13 88
3060 0 11 76
\end{verbatim}
\noindent
{\small Line 1, 2, 4 and 5 are representations of acceleration data and line 3 marks the beginning of the first gesture
triggered by a \textit{B} button down event.}

\medskip

The recorded files are evaluated in section \ref{evaluation}. All
recorded files used in the evaluation are available on
GitHub\footnote{https://github.com/GordonLesti/SlidingWindowFilter-evaluator/tree/v1.0.1/src/main/resources}.

