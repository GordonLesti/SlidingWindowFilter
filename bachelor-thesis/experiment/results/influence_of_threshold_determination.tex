\subsubsection{Influence of Threshold Determination} \label{influence_of_threshold_determination}
115 different simulations have been tested that use $\eta$DTW with a Sakoe-Chiba band of size 36\% depending on the
input time series length and \textit{Mid} as window size determination. Three different offical approaches have been
tested for the threshold determination, \textit{HAveD}, \textit{HMidD} and \textit{HMinD}. Furthermore two unoffical
approaches to double check the performance of the offical approaches, \textit{Peak} with a factor of 1.1 and 1.2. Every
approach has respectively 23 different simulations. Figure \ref{fig:threshold_result} illustrates the distribution of
the five subsets. The \textit{HAveD} and the \textit{HMidD} approaches are reaching the best $F_{1}score_{\mu}$ values,
\textit{HAveD} a bit better as \textit{HMidD}. Surprising is the poor performance of both \textit{Peak} approaches,
their low $Precision_{\mu}$ values are the reason for this. Remarkable are the top $Precision_{\mu}$ values of the
\textit{HMinD} threshold determination.

\begin{figure}
    \begin{center}
        \begin{tabular}{cc}
            \resizebox {0.45\textwidth} {!} {
                \begin{tikzpicture}
                    \begin{axis}[
                        legend pos=south west,
                        xmin=0.4,
                        xmax=1,
                        ymin=0.1,
                        ymax=0.7,
                        width=\axisdefaultwidth,
                        height=\axisdefaultwidth,
                        xlabel=$Precision_{\mu}$,
                        ylabel=$Recall_{\mu}$,
                        samples=100]
                        \addplot[blue, only marks, mark size=1] table {../data/fig/threshold_result/haved.dat};
                        \addlegendentry{HAveD}
                        \addplot[red, only marks, mark size=1] table {../data/fig/threshold_result/hmidd.dat};
                        \addlegendentry{HMidD}
                        \addplot[green, only marks, mark size=1] table {../data/fig/threshold_result/hmind.dat};
                        \addlegendentry{HMinD}
                        \addplot[violet, only marks, mark size=1] table {../data/fig/threshold_result/peak110.dat};
                        \addlegendentry{Peak (1.1)}
                        \addplot[cyan, only marks, mark size=1] table {../data/fig/threshold_result/peak120.dat};
                        \addlegendentry{Peak (1.2)}
                        \addplot[lightgray, domain=0.11:1] {(0.2 * x) / (2 * x - 0.2)};
                        \addplot[lightgray, domain=0.16:1] {(0.3 * x) / (2 * x - 0.3)};
                        \addplot[lightgray, domain=0.21:1] {(0.4 * x) / (2 * x - 0.4)};
                        \addplot[lightgray, domain=0.26:1] {(0.5 * x) / (2 * x - 0.5)};
                        \addplot[lightgray, domain=0.31:1] {(0.6 * x) / (2 * x - 0.6)};
                        \addplot[lightgray, domain=0.36:1] {(0.7 * x) / (2 * x - 0.7)};
                        \addplot[lightgray, domain=0.41:1] {(0.8 * x) / (2 * x - 0.8)};
                    \end{axis}
                \end{tikzpicture}
            } &
            \resizebox {0.45\textwidth} {!} {
                \begin{tikzpicture}
                    \begin{axis}[
                        xmin=0,
                        xmax=1,
                        ymin=0,
                        ymax=1,
                        width=\axisdefaultwidth,
                        height=\axisdefaultwidth,
                        xlabel=$Precision_{\mu}$,
                        ylabel=$Recall_{\mu}$,
                        samples=100]
                        \addplot[blue, only marks, mark size=0.4] table {../data/fig/threshold_result/haved.dat};
                        \addplot[red, only marks, mark size=0.4] table {../data/fig/threshold_result/hmidd.dat};
                        \addplot[green, only marks, mark size=0.4] table {../data/fig/threshold_result/hmind.dat};
                        \addplot[violet, only marks, mark size=0.4] table {../data/fig/threshold_result/peak110.dat};
                        \addplot[cyan, only marks, mark size=0.4] table {../data/fig/threshold_result/peak120.dat};
                        \addplot[lightgray, domain=0.051:1] {(0.1 * x) / (2 * x - 0.1)};
                        \addplot[lightgray, domain=0.11:1] {(0.2 * x) / (2 * x - 0.2)};
                        \addplot[lightgray, domain=0.16:1] {(0.3 * x) / (2 * x - 0.3)};
                        \addplot[lightgray, domain=0.21:1] {(0.4 * x) / (2 * x - 0.4)};
                        \addplot[lightgray, domain=0.26:1] {(0.5 * x) / (2 * x - 0.5)};
                        \addplot[lightgray, domain=0.31:1] {(0.6 * x) / (2 * x - 0.6)};
                        \addplot[lightgray, domain=0.36:1] {(0.7 * x) / (2 * x - 0.7)};
                        \addplot[lightgray, domain=0.41:1] {(0.8 * x) / (2 * x - 0.8)};
                        \addplot[lightgray, domain=0.46:1] {(0.9 * x) / (2 * x - 0.9)};
                    \end{axis}
                \end{tikzpicture}
            }
        \end{tabular}
    \end{center}
    \caption{$Precision_{\mu}$ and $Recall_{\mu}$ of all simulations that use $\eta$DTW with a Sakoe-Chiba band of size
    36\% depending on the input time series length and \textit{Mid} as window size determination, separated by the used
    threshold determination. The left plot is just a zoomed version of the right plot. Gray lines are illustrating the
    distribution of $F_{1}score_{\mu}$ in $\frac{1}{10}$ steps.}
    \label{fig:threshold_result}
\end{figure}
