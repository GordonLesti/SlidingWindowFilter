\paragraph{Influence of Blur Factor} \label{influence_of_blur_factor}
The blur factor for a Sliding Window Filter is expanding the interval that is accepting time series windows. A too small
interval can lead to many false positive classifications and a too large interval may leads to too much NNC-DTW calls.
The underlying time series measure for a Sliding Window Filter is affecting the best performing blur factor for the
filter. Eleven different blur factors have been tested for all three underlying time series measures. The influence of
the blur factor for the filter will be tested for all three underlying time series measures and compared to the
dominating configuration without filter. The right plot of figure \ref{fig:blur_factor_result} illustrates the influence
of the blur factor on the $F_{1}score_{\mu}$. Only the LNCE estimate is reaching with the given blur factors the same
$F_{1}score_{\mu}$ as the dominating configuration without filter. However all two filter types come also with small
blur factors very close to the top $F_{1}score_{\mu}$ value.

\begin{figure}
    \begin{center}
        \begin{tabular}{cc}
            \resizebox {0.5\textwidth} {!} {
                \begin{tikzpicture}
                    \begin{axis}[
                        legend pos=south east,
                        xmin=100,
                        xmax=300,
                        ymin=0.6,
                        ymax=0.75,
                        xlabel=filter interval size in \%,
                        ylabel=$F_{1}score_{\mu}$,
                        width=\textwidth,
                        height=\axisdefaultheight]
                        \addplot[blue] table {../data/fig/blur_factor_result/lnce.dat};
                        \addlegendentry{LNCE}
                        \addplot[red] table {../data/fig/blur_factor_result/var.dat};
                        \addlegendentry{VAR}
                        \addplot[dotted, black, domain=100:300] {0.738393631276109};
                        \addlegendentry{No Filter}
                    \end{axis}
                \end{tikzpicture}
            } &
            \resizebox {0.5\textwidth} {!} {
                \begin{tikzpicture}
                    \begin{axis}[
                        legend pos=south east,
                        xmin=100,
                        xmax=300,
                        ymin=0,
                        ymax=5500,
                        xlabel=filter interval size in \%,
                        ylabel=\#(nnc),
                        width=\textwidth,
                        height=\axisdefaultheight]
                        \addplot[blue] table {../data/fig/nnc_calls_result/lnce.dat};
                        \addlegendentry{LNCE}
                        \addplot[red] table {../data/fig/nnc_calls_result/var.dat};
                        \addlegendentry{VAR}
                        \addplot[dotted, black, domain=100:300] {4893};
                        \addlegendentry{No Filter}
                    \end{axis}
                \end{tikzpicture}
            }
        \end{tabular}
    \end{center}
    \caption{The left plot shows the $F_{1}score_{\mu}$ depending on the blur factor that expands the size of the filter
    interval in \%. The right plot shows the amount of NNC-DTW calls depending on the blur factor that expands the size
    of the filter interval in \%.}
    \label{fig:blur_factor_result}
\end{figure}

The left plot of figure \ref{fig:blur_factor_result} and also the table \ref{tab:result} shows that the presented
filters can reach the same $F_{1}score_{\mu}$ value as the dominating simulation without filter. The question remains,
how often are the presented filters blocking the NNC-DTW calls. The right plot of figure \ref{fig:blur_factor_result}
illustrates the influence of the blur factor on the amount of NNC-DTW calls. All two filter types remain clearly below
the amount of NNC-DTW calls of the dominating configuration without filter.
