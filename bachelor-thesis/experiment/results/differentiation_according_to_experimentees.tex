\subsubsection{Differentiation according to Experimentees} \label{differentiation_according_to_experimentees}
Many records achieved high $F_{1}score_{\mu}$s under the dominating simulation. However, there are also records that
did work that well under the best performing configuration. Figure \ref{fig:experimentee_result} shows the distribution
of the different records from different experimentees.

\begin{figure}
    \begin{center}
        \begin{tabular}{cc}
            \resizebox {0.45\textwidth} {!} {
                \begin{tikzpicture}
                    \begin{axis}[
                        xmin=0.2,
                        xmax=1,
                        ymin=0.2,
                        ymax=1,
                        width=\axisdefaultwidth,
                        height=\axisdefaultwidth,
                        xlabel=$Precision_{\mu}$,
                        ylabel=$Recall_{\mu}$,
                        samples=100]
                        \addplot+[
                            blue,
                            only marks,
                            nodes near coords,
                            every node near coord/.style={at={(0.12,0.17)}, color=gray},
                            point meta=explicit symbolic,
                            mark size=0.5] table[x=x, y=y, meta=label] {../data/fig/experimentee_result/experimentee.dat};
                        \addplot[lightgray, domain=0.16:1] {(0.3 * x) / (2 * x - 0.3)};
                        \addplot[lightgray, domain=0.21:1] {(0.4 * x) / (2 * x - 0.4)};
                        \addplot[lightgray, domain=0.26:1] {(0.5 * x) / (2 * x - 0.5)};
                        \addplot[lightgray, domain=0.31:1] {(0.6 * x) / (2 * x - 0.6)};
                        \addplot[lightgray, domain=0.36:1] {(0.7 * x) / (2 * x - 0.7)};
                        \addplot[lightgray, domain=0.41:1] {(0.8 * x) / (2 * x - 0.8)};
                        \addplot[lightgray, domain=0.46:1] {(0.9 * x) / (2 * x - 0.9)};
                    \end{axis}
                \end{tikzpicture}
            } &
            \resizebox {0.45\textwidth} {!} {
                \begin{tikzpicture}
                    \begin{axis}[
                        xmin=0,
                        xmax=1,
                        ymin=0,
                        ymax=1,
                        width=\axisdefaultwidth,
                        height=\axisdefaultwidth,
                        xlabel=$Precision_{\mu}$,
                        ylabel=$Recall_{\mu}$,
                        samples=100]
                        \addplot+[
                            blue,
                            only marks,
                            nodes near coords,
                            every node near coord/.style={at={(-0.05,-0.01)}, color=gray},
                            point meta=explicit symbolic,
                            mark size=0.4] table[x=x, y=y, meta=label] {../data/fig/experimentee_result/experimentee.dat};
                        \addplot[lightgray, domain=0.051:1] {(0.1 * x) / (2 * x - 0.1)};
                        \addplot[lightgray, domain=0.11:1] {(0.2 * x) / (2 * x - 0.2)};
                        \addplot[lightgray, domain=0.16:1] {(0.3 * x) / (2 * x - 0.3)};
                        \addplot[lightgray, domain=0.21:1] {(0.4 * x) / (2 * x - 0.4)};
                        \addplot[lightgray, domain=0.26:1] {(0.5 * x) / (2 * x - 0.5)};
                        \addplot[lightgray, domain=0.31:1] {(0.6 * x) / (2 * x - 0.6)};
                        \addplot[lightgray, domain=0.36:1] {(0.7 * x) / (2 * x - 0.7)};
                        \addplot[lightgray, domain=0.41:1] {(0.8 * x) / (2 * x - 0.8)};
                        \addplot[lightgray, domain=0.46:1] {(0.9 * x) / (2 * x - 0.9)};
                    \end{axis}
                \end{tikzpicture}
            }
        \end{tabular}
    \end{center}
    \caption{$Precision_{\mu}$ and $Recall_{\mu}$ of all 14 records from 14 different experimentees for the simulation
    that uses $\eta$DTW with a Sakoe-Chiba band of size 36\% depending on the input time series length, \textit{HAveD}
    as threshold determination and \textit{Mid} as window size determination without any filter. The left plot is just a
    zoomed version of the right plot. Gray lines are illustrating the distribution of $F_{1}score_{\mu}$ in
    $\frac{1}{10}$ steps.}
    \label{fig:experimentee_result}
\end{figure}
