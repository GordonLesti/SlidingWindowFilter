\subsubsection{File Format}
Interesting for the evaluation of the experiment are the acceleration data in three axis and the \textit{B} button down
and up events to mark the beginning and the end of a gesture. The controller is sending the acceleration data with an
average distance of five milliseconds. This results in three different kinds of events, incoming acceleration data, the
\textit{B} button is pressed down and the \textit{B} button is released up. Every event has been marked with the time in
milliseconds that have passed since the starting of the experiment. The representing line in a recorded file for a
acceleration data event is starting with the time in milliseconds followed by the acceleration data of all three axis.
The acceleration data of a axis comes as integer value with the unit
$\frac{dm}{s^2}$\footnote{Decimetre per second squared}. A \textit{B} button down event is represented by a line also
starting with the time in milliseconds followed by the keyword \textit{START} and a counter for the gesture. The
\textit{B} button up event is the same with the keyword \textit{END} instead of \textit{START}. An example excerpt of a
recorded file can look like this.\\\\
\verb!3045 20 19 74!\\
\verb!3050 16 12 70!\\
\verb!3055 START 1!\\
\verb!3055 14 13 88!\\
\verb!3060 0 11 76!\\

Line 1, 2, 4 and 5 are representations of acceleration data and line 3 marks the beginning of the first gesture
triggered by a \textit{B} button down event. The recorded files will be evaluated in section \ref{evaluation}.
