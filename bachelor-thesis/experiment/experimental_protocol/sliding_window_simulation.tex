\subsubsection{Sliding Window Simulation} \label{sliding_window_simulation}
A sliding window simulation walks with a given window size and step size over the simulated acceleration time series
stream of an experimentee and tries to identify as many gestures as possible. The simulated acceleration time series
stream of an experimentee starts after the slide (j) of figure \ref{fig:slides} and ends in slide (r) of figure
\ref{fig:slides}. The sliding window simulation is not able to read the labels of an gesture in the simulated
acceleration time series. Furthermore a sliding window simulation gets the gestures of an experimentee from slide (b)
to (i) in figure \ref{fig:slides} as training data.

Such a simulation is not free of parameters. Two parameters are already mentioned, the window size and the steps
size. In addition to parameters being the choice of the distance measure for time series, the threshold that marks
unclassifiable windows and the measure used for the sliding window filter with a blur factor. All those parameters and
their options are explained and mentioned in the following section. Furthermore a performance measure to compare the
different configured simulations is explained.

\paragraph{Window Size Determination} \label{window_size_determination}
The determination of the used window size is not trivial. An oversized or too small window can lead to many false
negative classifications. Four pretty basic approaches based on the length of the training data have been tested.

\begin{itemize}
    \item \textbf{Max:} Stands for maximum, this approach takes the length from the longest time series of the training
        data.
    \item \textbf{Min:} Stands for minimum, this approach takes the length from the shortest time series of the training
        data.
    \item \textbf{Ave:} Stands for average, this approach takes the average length from all time series of the training
        data.
    \item \textbf{Mid:} Stands for middle, this approach takes the average length from the longest and the shortest time
        series of the training data.
\end{itemize}

\paragraph{Step Size Determination} \label{step_size_determination}
The Step Size Determination is also not trivial. A oversized step size is very likely producing more false negative
classifications, but leads to a reduced execution of 1NN calls. Too small step sizes on the other hand will overuse the
1NN calls. In this experiment the step size was set to one tenth of the window size in view of an huge set of combinable
parameters for a simulation.

\paragraph{Time Series Distance Measures} \label{time_series_distance_measures}
DTW has proven in \cite{liu2009uwave} that it works for time series as accelerometer based gesture detection. The
following distance measure functions for time series have been tested. Every different version for the distanc measure
was tested with 34 different sizes of the Sakoe-Chiba band.

\begin{itemize}
    \item \textbf{DTW:} Plain DTW as explained in section \ref{dynamic_time_warping}.
    \item \textbf{$\eta$DTW:} DTW with $\eta$ normalization as explained in section \ref{time_series_normalization}.
    \item \textbf{$\eta '$DTW:} DTW with $\eta '$ normalization as explained in section \ref{time_series_normalization}.
\end{itemize}

\paragraph{Threshold Determination} \label{threshold_determination}
It is assumed that a large number of incoming time series windows should be unclassifiable. This can only be determined
by specifying a threshold as upper bound for the distance between a time series window and its nearest neighbour. An
intuitive approach for the threshold determinations would include knowledge about the distances between the instances of
a class. This approach can not be tested in this experiment, cause the protocol has the great weakness to contain only
one instance for every class in the training data as already mentioned in section \ref{protocol_review}. The following
threshold determinations have been tested.

\begin{itemize}
    \item \textbf{HMinD:} Stands for half minimum distance, this approach determinates the threshold for a class by the
        half minimum distance of the class to all other classes in the training data.
    \item \textbf{HAveD:} Stands for half average distance, this approach determinates the threshold for a class by the
        half average distance of the class to all other classes in the training data.
    \item \textbf{HMidD:} Stands for half middle distance, this approach determinates the threshold for a class by the
        half average of the minimum and maximum distance of the class to all other classes in the training data.
    \item \textbf{Peak:} Stands for peaking, this approach fakes already generated knowledge about the distance
        threshold by setting the threshold to the by a small factor increased distance between the only existing
        training class instance and the test instance that has to be found. Two more intuitively chosen factors will be
        tested, $\frac{11}{10}$ and $\frac{12}{10}$. The peaking approach exists only to judge the performance of other
        threshold determinations.
\end{itemize}

\paragraph{Sliding Window Filter Measures} \label{sliding_window_filter_measures}
The Sliding Window Filter has already been explained in section \ref{sliding_window_filter}. The following time series
measures have been tested as underlying measures for the Sliding Window Filter.

\begin{itemize}
    \item \textbf{CE:} The Complexity Estimate as explained in section \ref{complexity-invariant_distance_measure}.
    \item \textbf{ACE:} The Average Complexity Estimate is CE divided by the length of the time series reduced by one.
    Given is a time series $Q = (q_1, q_2, \dots, q_i, \dots, q_l)$ with length $l > 1$ over the domain set $\mathbb{U}$
    and the measure CE. The ACE of a time series $Q$ can be calculated by the following formula.
    \begin{center}
        $ACE(Q) = \frac{1}{l - 1}CE(Q)$
    \end{center}
    \item \textbf{VAR:} The Sample Variance based on \cite{chan1983algorithms}. Given is a time series
    $Q = (q_1, q_2, \dots, q_i, \dots, q_l)$ with length $l > 0$ over the domain set $\mathbb{U}$, a distance measure
    function $D$ with $D: \mathbb{U} \times \mathbb{U} \to \mathbb{R}$, a summing function $ADD$ with
    $ADD: \mathbb{U} \times \mathbb{U} \to \mathbb{U}$ and a scalar multiplication function $SMULT$ with
    $SMULT: \mathbb{R} \times \mathbb{U} \to \mathbb{U}$. The Variance of a time series $Q$ can be calculated by the
    following formula.
    \begin{center}
        $VAR(Q) = \frac{1}{l}\sum \limits_{i=1}^{l} D(q_i, \bar{q})^2$
    \end{center}
    where
    \begin{center}
        $\bar{q} = SMULT(\frac{1}{l-1}, \sum \limits_{i=1}^{l} q_i)$
    \end{center}
    The sigma sum sign $\sum$ in the last formula is used under the contaxt of the $ADD$ function between elements of
    $\mathbb{U}$.
\end{itemize}
Every mentioned estimates has been tested as underlying measures for the Sliding Window Filter combined with a blur
factor of $\frac{10}{10}$, $\frac{11}{10}$, $\frac{12}{10}$, $\frac{13}{10}$, $\frac{14}{10}$ and $\frac{15}{10}$.

\begin{frame}{Performance Measure}
    \begin{center}
        \begin{itemize}
            \item A sliding window simulation is a multi-class classificator
            \pause
            \item Every gesture is a class with:
            \pause
            \begin{itemize}
                \item \textbf{tp:} True positve
                \pause
                \item \textbf{tn:} True negative
                \pause
                \item \textbf{fp:} False positive
                \pause
                \item \textbf{fn:} False negative
            \end{itemize}
            \pause
            \item Multi-class classifier performance measures mentioned in \cite{sokolova2009systematic}:
            \pause
            \begin{itemize}
                \item $Precision_{\mu}$
                \pause
                \item $Recall_{\mu}$
                \pause
                \item $F_{\beta}score_{\mu}$
            \end{itemize}
        \end{itemize}
    \end{center}
\end{frame}

\begin{frame}{Performance Measure}{$Precision_{\mu}$}
    \begin{center}
        \begin{itemize}
            \item Amount of correct matches in relation to all matches
            \pause
            \item $Precision_{\mu} = \frac{\sum \limits_{i=1}^{l} tp_i}{\sum \limits_{i=1}^{l} (tp_i + fp_i)}$
        \end{itemize}
    \end{center}
\end{frame}

\begin{frame}{Performance Measure}{$Recall_{\mu}$}
    \begin{center}
        \begin{itemize}
            \item Amount of correct matches in relation to all possible correct matches
            \pause
            \item $Recall_{\mu} = \frac{\sum \limits_{i=1}^{l} tp_i}{\sum \limits_{i=1}^{l} (tp_i + fn_i)}$
        \end{itemize}
    \end{center}
\end{frame}

\begin{frame}{Performance Measure}{$F_{\beta}score_{\mu}$}
    \begin{center}
        \begin{itemize}
            \item $Precision_{\mu}$ and $Recall_{\mu}$ in relation
            \pause
            \item $F_{\beta}score_{\mu} = \frac{(\beta^2 + 1)Precision_{\mu} Recall_{\mu}}{\beta^2 Precision_{\mu} + Recall_{\mu}}$
        \end{itemize}
    \end{center}
\end{frame}

