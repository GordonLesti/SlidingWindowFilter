\subsubsection{Sliding Window Simulation} \label{sliding_window_simulation}
A sliding window simulation walks with a given window size and step size over the simulated acceleration time series
stream of an experimentee and tries to identify as many gestures as possible. The simulated acceleration time series
stream of an experimentee starts after the slide (j) of figure \ref{fig:slides} and ends in slide (r) of figure
\ref{fig:slides}. The sliding window simulation is not able to read the labels of an gesture in the simulated
acceleration time series. Furthermore a sliding window simulation gets the gestures of an experimentee from slide (b)
to (i) in figure \ref{fig:slides} as training data.

Such a simulation is not free of parameters. Two parameters are already mentioned, the window size and the steps
size. In addition to parameters being the choice of the distance measure for time series, the threshold that marks
unclassifiable windows and the measure used for the sliding window filter with a blur factor. All those parameters and
their options are explained and mentioned in the following section. Furthermore a performance measure to compare the
different configured simulations is explained.

\paragraph{Window Size Determination} \label{window_size_determination}
The determination of the used window size is not trivial. An oversized or too small window can lead to many false
negative classifications. Four pretty basic approaches based on the length of the training data have been tested.

\begin{itemize}
    \item \textbf{Maximum:} takes the length from the longest time series of the training data.
    \item \textbf{Minimum:} takes the length from the shortest time series of the training data.
    \item \textbf{Average:} takes the average length from all time series of the training data.
    \item \textbf{Middle:} takes the average length from the longest and the shortest time series of the training data.
\end{itemize}

\begin{frame}{Step Size Determination}{Options}
    \begin{itemize}
        \item One tenth of the window size
    \end{itemize}
\end{frame}

\begin{frame}{Time Series Distance Measures}{Options}
    \begin{block}{Normalization}
        \begin{itemize}
            \item \textbf{DTW:} Plain DTW
            \pause
            \item \textbf{$\eta$DTW:} DTW with $\eta$ normalization
            \pause
            \item \textbf{$\eta '$DTW:} DTW with $\eta '$ normalization
            \pause
        \end{itemize}
    \end{block}
    \begin{block}{Sakoe-Chiba band}
        \begin{itemize}
            \item 34 different sizes
        \end{itemize}
    \end{block}
\end{frame}

\paragraph{Threshold Determination} \label{threshold_determination}
It is assumed that a large number of incoming time series windows should be unclassifiable. This can only be determined
by specifying a threshold as upper bound for the distance between a time series window and its nearest neighbour. An
intuitive approach for the threshold determinations would include knowledge about the distances between the instances of
a class. This approach can not be tested in this experiment, because the recording instructions do have the great weakness
to contain only one instance for every class in the training data as mentioned in section \ref{instructions_review}. The
following threshold determinations were tested.

\begin{itemize}
    \item \textbf{HMinD:} Stands for half minimum distance, this approach determinates the threshold for a class by the
        half minimum distance of the class to all other classes in the training data.
    \item \textbf{HAveD:} Stands for half average distance, this approach determinates the threshold for a class by the
        half average distance of the class to all other classes in the training data.
    \item \textbf{HMidD:} Stands for half middle distance, this approach determinates the threshold for a class by the
        half average of the minimum and maximum distance of the class to all other classes in the training data.
    \item \textbf{Peak:} Stands for peaking, this approach fakes already generated knowledge about the distance
        threshold by setting the threshold to the by a small factor increased distance between the only existing
        training class instance and the test instance that has to be found. Two more intuitively chosen factors were
        tested, $\frac{11}{10}$ and $\frac{12}{10}$. The peaking approach exists only to judge the performance of other
        threshold determinations.
\end{itemize}

\paragraph{Sliding Window Filter Measures} \label{sliding_window_filter_measures}
The Sliding Window Filter has already been explained in section \ref{sliding_window_filter}. The following time series
measures have been tested as underlying measures for the Sliding Window Filter.

\begin{itemize}
    \item \textbf{LNCE:} Stands for length normalized CE. LNCE is CE divided by the length of the time series reduced by
    one. Given is a time series $Q = (q_1, q_2, \dots, q_i, \dots, q_l)$ with length $l > 1$ over the domain set
    $\mathbb{U}$ and the measure CE. The LNCE of a time series $Q$ can be calculated by the following formula.
    \begin{center}
        $LNCE(Q) = \frac{1}{l - 1}CE(Q)$
    \end{center}
    \item \textbf{VAR:} The Sample Variance based on \cite{chan1983algorithms}. Given is a time series
    $Q = (q_1, q_2, \dots, q_i, \dots, q_l)$ with length $l > 0$ over the domain set $\mathbb{U}$ and a distance measure
    function $d$ with $d: \mathbb{U} \times \mathbb{U} \to \mathbb{R}$. The Variance of a time series $Q$ can be
    calculated by the following formula.
    \begin{center}
        $VAR(Q) = \frac{1}{l}\sum \limits_{i=1}^{l} d(q_i, \bar{q})^2$
    \end{center}
    where
    \begin{center}
        $\bar{q} = \frac{1}{l} \sum \limits_{i=1}^{l} q_i$
    \end{center}
\end{itemize}
The passing interval of every filter has been tested with eleven different factors that increases the interval from
100\% to maximum 300\%.

\paragraph{Performance Measure} \label{performance_measure}
The presented parameters for a Sliding Window Simulation are resulting in a huge amount of different configured
simulations. Every configured simulation tries to detect gestures in the test data stream for every experimentee. It can
be argued how to compare the performance of those simulations. Have misrecognized gestures in the assessment more weight
than correctly recognized or the other way around? It is no easy decision to weight the mistakes of a simulation against
the success of a simulation.

All containing data points of a supposed detected gesture will be labeled by a simulation. Those labels can be compared
to the original labels that have been made by the experimentee. A simulation is a multi-class classificator for the
gesture classes $C_{GesA}, C_{GesB}, \dots, C_{GesH}$. Common performance measures for multi-class classificator are
$Average Accuracy$, $Precision_{\mu}$, $Recall_{\mu}$ and $Fscore_{\mu}$ as mentioned in\cite{sokolova2009systematic}.

