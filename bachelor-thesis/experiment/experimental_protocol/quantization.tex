\subsubsection{Quantization} \label{quantization}
All recorded data has been quantized before evaluation. The authors of \cite{liu2009uwave} have two reasons for that,
the length of the time series will be reduced for DTW in order to improve computation efficiency and the recorded time
series will be transformed with a stable time gab between the data points. The quantization process of
\cite{liu2009uwave} has been taken over. The hole quantization process of a recorded gesture is
illustrated by figure \ref{fig:quantization}.

\paragraph{Compressing} The recorded acceleration data has been compressed to an average value for a window size of 50
ms and a step length of 30 ms.

\paragraph{Conversion} The compressed records will be converted into 33 different levels that will be summarized in
table \ref{table:conversion}.

\begin{table}
    \begin{center}
        \begin{tabular}{l l}
            \textbf{Acceleration data ($a$) in $\frac{dm}{s^2}$} & \qquad \textbf{Converted value}\\
            \hline
            $a > 200$ & \qquad 16\\
            $100 < a < 200$ & \qquad 11 to 15 (five levels linearly)\\
            $0 < a < 100$ & \qquad 1 to 10 (ten levels linearly)\\
            $a = 0$ & \qquad 0\\
            $-100 < a < 0$ & \qquad -1 to - 10 (ten levels linearly)\\
            $-200 < a < -100$ & \qquad -11 to - 15 (five levels linearly)\\
            $a < -200$ & \qquad -16\\
        \end{tabular}
    \end{center}
    \caption{Table shows the conversion rules of the recorded acceleration data. In contrast to \cite{liu2009uwave} are
    $100\frac{dm}{s^2}$ the steps threshold and not $1g$.}
	\label{table:conversion}
\end{table}

\begin{figure}
    \begin{center}
        \begin{tabular}{ccc}
            \resizebox {0.3\textwidth} {!} {
                \begin{tikzpicture}
                    \begin{axis}[
                        xmin=1,
                        xmax=295,
                        xlabel=time,
                        ylabel=acceleration in $\frac{dm}{s^2}$]
                        \addplot[blue, ultra thick, mark=none] table[x=t, y=x] {experiment/experimental_protocol/quantization/raw.dat};
                        \addplot[red, ultra thick, mark=none] table[x=t, y=y] {experiment/experimental_protocol/quantization/raw.dat};
                        \addplot[green, ultra thick, mark=none] table[x=t, y=z] {experiment/experimental_protocol/quantization/raw.dat};
                    \end{axis}
                \end{tikzpicture}
            } &
            \resizebox {0.3\textwidth} {!} {
                \begin{tikzpicture}
                    \pgfplotsset{every axis legend/.append style={
                		at={(0.5,1.03)},
                		anchor=south}}
                    \begin{axis}[
                        xmin=1,
                        xmax=52,
                        xlabel=time,
                        ylabel=acceleration in $\frac{dm}{s^2}$,
                        legend columns=4]
                        \addplot[blue, ultra thick, mark=none] table[x=t, y=x] {experiment/experimental_protocol/quantization/compressed.dat};
                        \addlegendentry{x-axis}
                        \addplot[red, ultra thick, mark=none] table[x=t, y=y] {experiment/experimental_protocol/quantization/compressed.dat};
                        \addlegendentry{y-axis}
                        \addplot[green, ultra thick, mark=none] table[x=t, y=z] {experiment/experimental_protocol/quantization/compressed.dat};
                        \addlegendentry{z-axis}
                    \end{axis}
                \end{tikzpicture}
            } &
            \resizebox {0.3\textwidth} {!} {
                \begin{tikzpicture}
                    \begin{axis}[
                        xmin=1,
                        xmax=52,
                        xlabel=time,
                        ylabel=converted acceleration]
                        \addplot[blue, ultra thick, mark=none] table[x=t, y=x] {experiment/experimental_protocol/quantization/converted.dat};
                        \addplot[red, ultra thick, mark=none] table[x=t, y=y] {experiment/experimental_protocol/quantization/converted.dat};
                        \addplot[green, ultra thick, mark=none] table[x=t, y=z] {experiment/experimental_protocol/quantization/converted.dat};
                    \end{axis}
                \end{tikzpicture}
            }
        \end{tabular}
    \end{center}
    \caption{The left plot shows the raw recorded gesture for all three axis. On the middle plot are the compressed
    acceleration datas of the gesture. The right plot shows the converted and compressed gesture.}
    \label{fig:quantization}
\end{figure}
