\paragraph{Performance Measure} \label{performance_measure}
The presented parameters for a sliding window simulation are resulting in a huge amount of different configured
simulations. Every configured simulation tries to detect gestures in the test data stream for every experimentee. It can
be argued how to compare the performance of those simulations. Have misrecognized gestures in the assessment more weight
than correctly recognized or the other way around? It is no easy decision to weigh the mistakes of a simulation against
the success of a simulation.

All containing data points of a supposed detected gesture are labeled by a simulation. Those labels can be compared
to the original labels that were made by the experimentee. This results in true positive ($tp$), true negative
($tn$), false positive ($fp$) and false negative ($fn$) labels for every gesture $GesA, GesB, \dots, GesH$. Seen from
this point of view, a simulation is a multi-class classificator for the gesture classes
$K_{GesA}, K_{GesB}, \dots, K_{GesH}$. Common performance measures for multi-class classificator are $Precision_{\mu}$,
$Recall_{\mu}$ and $F_{\beta}score_{\mu}$ as mentioned in \cite{sokolova2009systematic}.

\begin{equation}
    Precision_{\mu} = \frac{\sum \limits_{i=1}^{l} tp_i}{\sum \limits_{i=1}^{l} (tp_i + fp_i)}
\end{equation}
\begin{equation}
    Recall_{\mu} = \frac{\sum \limits_{i=1}^{l} tp_i}{\sum \limits_{i=1}^{l} (tp_i + fn_i)}
\end{equation}
\begin{equation}
    F_{\beta}score_{\mu} = \frac{(\beta^2 + 1)Precision_{\mu} Recall_{\mu}}{\beta^2 Precision_{\mu} + Recall_{\mu}}
\end{equation}

The $F_{1}score_{\mu}$ is used in the following subsection to rank the different simulations.
