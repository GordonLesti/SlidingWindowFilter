\subsection{Discussion} \label{discussion} \label{instructions_review}
The presented recording instructions in section \ref{data_recording_instructions} combined with the recording software
in section \ref{recording_setup} have been created in front of this bachelor thesis and the experiment was performed
while writing this thesis. Unfortunately do the instructions have some weaknesses that occurred during the experiment or
while evaluating the produced data.

\paragraph{Training Data Quantity} The instructions are containing only 8 instances of training gestures for 8 different
classes. That made the determination of a threshold for a class very difficult. It would have been better to produce at
least two or better three instances of a gesture for every class.

\paragraph{Gesture Illustration Size} It was confusing for some experimentess that the gestures on the slides (j) to (q)
had an other size as the gestures on slide (b) to (i) of figure \ref{fig:slides}. The recording had to be repeated,
cause the experimentess performed the gesture scaled to the illustration size on the slide.

Changing the recording instructions during the bachelor thesis due to the above mentioned weaknesses was no option. It
is assumed that more instances for every class in the training data set would result in a better gesture detection.
However, a bigger training set would also expand the passing filter interval in a more natural way. This would result
in smaller blur factors to expand the passing interval of the filter.
