\paragraph{Influence of Sakoe-Chiba Band Size} \label{influence_of_sakoe-chiba_band_size}
The positive influence of the Sakoe-Chiba band size for the performance of a simulation in the experiment is verified by
table \ref{tab:result}. The most dominating configurations are using a band size with 36\% of the input time series.
It appears not effective to compare all different tested band sizes on a huge subset as done above. Therefore, the
influence of the Sakoe-Chiba band size will be demonstrated by changing the band size on the best performing simulation
without a filter. This configuration can be found in row two of table \ref{tab:result}. Figure
\ref{fig:sakoe-chiba_band_result} illustrates the influence of the Sakoe-Chiba band size on the $F_{1}score_{\mu}$ for
the best performing simulation without filter. There is a maximum at 36\% band size.

\begin{figure}
    \begin{center}
        \resizebox {\textwidth} {!} {
            \begin{tabular}{cc}
                \resizebox {!} {\height} {
                    \begin{tikzpicture}
                        \begin{axis}[
                            xmin=0,
                            ymin=0.65,
                            xmax=80,
                            xlabel=band size in \% depending on input time series,
                            ylabel=$F_{1}score_{\mu}$,
                            width=\axisdefaultwidth,
                            height=0.7*\axisdefaultheight]
                            \addplot[blue, ultra thick] table {../data/fig/sakoe-chiba_band_result/scb.dat};
                        \end{axis}
                    \end{tikzpicture}
                } &
                \resizebox {!} {\height} {
                    \begin{tikzpicture}
                        \begin{axis}[
                            xmin=0,
                            xmax=200,
                            ymin=0,
                            ymax=1,
                            xlabel=band size in \% depending on input time series,
                            ylabel=$F_{1}score_{\mu}$,
                            width=\axisdefaultwidth,
                            height=0.7*\axisdefaultheight]
                            \addplot[blue, ultra thick] table {../data/fig/sakoe-chiba_band_result/scb.dat};
                        \end{axis}
                    \end{tikzpicture}
                }
            \end{tabular}
        }
    \end{center}
    \caption{The $F_{1}score_{\mu}$ depending on the size of the Sakoe-Chiba band. The left plot is a zoomed and scaled
    version of the right plot.}
    \label{fig:sakoe-chiba_band_result}
\end{figure}
