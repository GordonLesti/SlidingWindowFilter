\subsubsection{Sliding Window Simulation} \label{sliding_window_simulation}
A Sliding Window Simulation walks with a given window size and step size over the simulated acceleration time series
stream of an experimentee and tries to identify as many gestures as possible. The simulated acceleration time series
stream of an experimentee starts after the slide (j) of figure \ref{fig:slides} and ends in slide (r) of figure
\ref{fig:slides}. The Sliding Window Simulation is not able to read the labels of an gesture in the simulated
acceleration time series. Furthermore a Sliding Window Simulation gets the gestures of an experimentee from slide (b)
to (i) in figure \ref{fig:slides} as training data.

Such a simulation is not free of parameters. Two parameters have already been mentioned, the window size and the steps
size. In addition as parameters are the choice of the distance measure for time series, the threshold that marks
unclassifiable windows and the Sliding Window Filter with a blur factor. All those parameters and their options will be
explained and mentioned in the following section. Furthermore a performance measure to compare the different configured
simulations will be explained.

\paragraph{Window Size Determination} \label{window_size_determination}
The determination of the used window size is not trivial. An oversized or too small window can lead to many false
negative classifications. Four pretty basic approaches based on the length of the training data have been tested.

\begin{itemize}
    \item \textbf{Maximum:} takes the length from the longest time series of the training data.
    \item \textbf{Minimum:} takes the length from the shortest time series of the training data.
    \item \textbf{Average:} takes the average length from all time series of the training data.
    \item \textbf{Middle:} takes the average length from the longest and the shortest time series of the training data.
\end{itemize}

\begin{frame}{Step Size Determination}{Options}
    \begin{itemize}
        \item One tenth of the window size
    \end{itemize}
\end{frame}

\begin{frame}{Time Series Distance Measures}{Options}
    \begin{block}{Normalization}
        \begin{itemize}
            \item \textbf{DTW:} Plain DTW
            \pause
            \item \textbf{$\eta$DTW:} DTW with $\eta$ normalization
            \pause
            \item \textbf{$\eta '$DTW:} DTW with $\eta '$ normalization
            \pause
        \end{itemize}
    \end{block}
    \begin{block}{Sakoe-Chiba band}
        \begin{itemize}
            \item 34 different sizes
        \end{itemize}
    \end{block}
\end{frame}

