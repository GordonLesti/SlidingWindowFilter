\subsubsection{Differentiation according to Experimentees} \label{differentiation_according_to_experimentees}
Many records achieved high $F_{1}score_{\mu}$s under the dominating simulation. However, there are also records that
did work that well under the best performing configuration. Figure \ref{fig:experimentee_result} shows the distribution
of the different records from different experimentees.

\begin{figure}
    \begin{center}
        \resizebox {\textwidth} {!} {
            \begin{tabular}{cc}
                \resizebox {!} {\height} {
                    \begin{tikzpicture}
                        \begin{axis}[
                            xmin=0.2,
                            xmax=1,
                            ymin=0.2,
                            ymax=1,
                            width=\axisdefaultwidth,
                            height=\axisdefaultwidth,
                            xlabel=$Precision_{\mu}$,
                            ylabel=$Recall_{\mu}$,
                            samples=100]
                            \addplot+[
                                blue,
                                only marks,
                                nodes near coords,
                                every node near coord/.style={at={(0.12,0.17)}, color=black},
                                point meta=explicit symbolic] table[x=x, y=y, meta=label] {../data/fig/experimentee_result/experimentee.dat};
                            \addplot[gray, domain=0.16:1] {(0.3 * x) / (2 * x - 0.3)};
                            \addplot[gray, domain=0.21:1] {(0.4 * x) / (2 * x - 0.4)};
                            \addplot[gray, domain=0.26:1] {(0.5 * x) / (2 * x - 0.5)};
                            \addplot[gray, domain=0.31:1] {(0.6 * x) / (2 * x - 0.6)};
                            \addplot[gray, domain=0.36:1] {(0.7 * x) / (2 * x - 0.7)};
                            \addplot[gray, domain=0.41:1] {(0.8 * x) / (2 * x - 0.8)};
                            \addplot[gray, domain=0.46:1] {(0.9 * x) / (2 * x - 0.9)};
                        \end{axis}
                    \end{tikzpicture}
                } &
                \resizebox {!} {\height} {
                    \begin{tikzpicture}
                        \begin{axis}[
                            xmin=0,
                            xmax=1,
                            ymin=0,
                            ymax=1,
                            width=\axisdefaultwidth,
                            height=\axisdefaultwidth,
                            xlabel=$Precision_{\mu}$,
                            ylabel=$Recall_{\mu}$,
                            samples=100]
                            \addplot+[
                                blue,
                                only marks,
                                nodes near coords,
                                every node near coord/.style={at={(-0.05,0)}, color=black},
                                point meta=explicit symbolic] table[x=x, y=y, meta=label] {../data/fig/experimentee_result/experimentee.dat};
                            \addplot[gray, domain=0.051:1] {(0.1 * x) / (2 * x - 0.1)};
                            \addplot[gray, domain=0.11:1] {(0.2 * x) / (2 * x - 0.2)};
                            \addplot[gray, domain=0.16:1] {(0.3 * x) / (2 * x - 0.3)};
                            \addplot[gray, domain=0.21:1] {(0.4 * x) / (2 * x - 0.4)};
                            \addplot[gray, domain=0.26:1] {(0.5 * x) / (2 * x - 0.5)};
                            \addplot[gray, domain=0.31:1] {(0.6 * x) / (2 * x - 0.6)};
                            \addplot[gray, domain=0.36:1] {(0.7 * x) / (2 * x - 0.7)};
                            \addplot[gray, domain=0.41:1] {(0.8 * x) / (2 * x - 0.8)};
                            \addplot[gray, domain=0.46:1] {(0.9 * x) / (2 * x - 0.9)};
                        \end{axis}
                    \end{tikzpicture}
                }
            \end{tabular}
        }
    \end{center}
    \caption{$Precision_{\mu}$ and $Recall_{\mu}$ of all 14 records from 14 different experimentees for the simulation
    that uses $\eta$DTW with a Sakoe-Chiba band of size 36\% depending on the input time series length, \textit{HAveD}
    as threshold determination and \textit{Mid} as window size determination without any filter. The left plot is just a
    zoomed version of the right plot. Gray lines are illustrating the distribution of $F_{1}score_{\mu}$ in
    $\frac{1}{10}$ steps.}
    \label{fig:experimentee_result}
\end{figure}

\begin{figure}
    \begin{center}
        \resizebox {\textwidth} {!} {
            \begin{tikzpicture}
                \begin{axis}[
                    xmin=0,
                    xmax=2426,
                    ymin=-16,
                    ymax=16,
                    title={\Huge $exp_1$},
                    width=8\textwidth,
                    height=\axisdefaultheight,
                    xticklabels={,,},
                    yticklabels={,,}]
                    \addplot[blue, mark=none, opacity=0.4] table[x=t, y=x] {../data/fig/experimentee_result2/exp1.dat};
                    \addplot[red, mark=none, opacity=0.4] table[x=t, y=y] {../data/fig/experimentee_result2/exp1.dat};
                    \addplot[green, mark=none, opacity=0.4] table[x=t, y=z] {../data/fig/experimentee_result2/exp1.dat};
                    \addplot+[fill, opacity=0.5, red, mark=none] coordinates
                        {(300, -16) (306, -16) (306, 16) (300, 16)} --cycle;
                    \addplot+[fill, opacity=0.5, green, mark=none] coordinates
                        {(307, -16) (355, -16) (355, 16) (307, 16)} --cycle;
                    \addplot+[fill, opacity=0.5, red, mark=none] coordinates
                        {(356, -16) (361, -16) (361, 16) (356, 16)} --cycle;
                    \addplot+[fill, opacity=0.5, red, mark=none] coordinates
                        {(505, -16) (507, -16) (507, 16) (505, 16)} --cycle;
                    \addplot+[fill, opacity=0.5, green, mark=none] coordinates
                        {(508, -16) (563, -16) (563, 16) (508, 16)} --cycle;
                    \addplot+[fill, opacity=0.5, red, mark=none] coordinates
                        {(564, -16) (566, -16) (566, 16) (564, 16)} --cycle;
                    \addplot+[fill, opacity=0.5, green, mark=none] coordinates
                        {(722, -16) (777, -16) (777, 16) (722, 16)} --cycle;
                    \addplot+[fill, opacity=0.5, red, mark=none] coordinates
                        {(778, -16) (783, -16) (783, 16) (778, 16)} --cycle;
                    \addplot+[fill, opacity=0.5, blue, mark=none] coordinates
                        {(940, -16) (950, -16) (950, 16) (940, 16)} --cycle;
                    \addplot+[fill, opacity=0.5, green, mark=none] coordinates
                        {(951, -16) (1012, -16) (1012, 16) (951, 16)} --cycle;
                    \addplot+[fill, opacity=0.5, blue, mark=none] coordinates
                        {(1013, -16) (1018, -16) (1018, 16) (1013, 16)} --cycle;
                    \addplot+[fill, opacity=0.5, red, mark=none] coordinates
                        {(1312, -16) (1315, -16) (1315, 16) (1312, 16)} --cycle;
                    \addplot+[fill, opacity=0.5, green, mark=none] coordinates
                        {(1316, -16) (1366, -16) (1366, 16) (1316, 16)} --cycle;
                    \addplot+[fill, opacity=0.5, red, mark=none] coordinates
                        {(1367, -16) (1373, -16) (1373, 16) (1367, 16)} --cycle;
                    \addplot+[fill, opacity=0.5, red, mark=none] coordinates
                        {(1685, -16) (1688, -16) (1688, 16) (1685, 16)} --cycle;
                    \addplot+[fill, opacity=0.5, green, mark=none] coordinates
                        {(1689, -16) (1746, -16) (1746, 16) (1689, 16)} --cycle;
                    \addplot+[fill, opacity=0.5, blue, mark=none] coordinates
                        {(1747, -16) (1747, -16) (1747, 16) (1747, 16)} --cycle;
                    \addplot+[fill, opacity=0.5, red, mark=none] coordinates
                        {(2082, -16) (2089, -16) (2089, 16) (2082, 16)} --cycle;
                    \addplot+[fill, opacity=0.5, green, mark=none] coordinates
                        {(2090, -16) (2143, -16) (2143, 16) (2090, 16)} --cycle;
                    \addplot+[fill, opacity=0.5, blue, mark=none] coordinates
                        {(2144, -16) (2145, -16) (2145, 16) (2144, 16)} --cycle;
                    \addplot+[fill, opacity=0.5, blue, mark=none] coordinates
                        {(2311, -16) (2347, -16) (2347, 16) (2311, 16)} --cycle;
                    \addplot+[fill, opacity=0.5, red, mark=none] coordinates
                        {(2317, -16) (2378, -16) (2378, 16) (2317, 16)} --cycle;
                \end{axis}
            \end{tikzpicture}
        }
        \resizebox {\textwidth} {!} {
            \begin{tikzpicture}
                \begin{axis}[
                    xmin=0,
                    xmax=2426,
                    ymin=-16,
                    ymax=16,
                    title={\Huge $exp_2$},
                    width=8\textwidth,
                    height=\axisdefaultheight,
                    xticklabels={,,},
                    yticklabels={,,}]
                    \addplot[blue, mark=none, opacity=0.4] table[x=t, y=x] {../data/fig/experimentee_result2/exp2.dat};
                    \addplot[red, mark=none, opacity=0.4] table[x=t, y=y] {../data/fig/experimentee_result2/exp2.dat};
                    \addplot[green, mark=none, opacity=0.4] table[x=t, y=z] {../data/fig/experimentee_result2/exp2.dat};
                    \addplot+[fill, opacity=0.5, blue, mark=none] coordinates
                        {(314, -16) (361, -16) (361, 16) (314, 16)} --cycle;
                    \addplot+[fill, opacity=0.5, red, mark=none] coordinates
                        {(336, -16) (401, -16) (401, 16) (336, 16)} --cycle;
                    \addplot+[fill, opacity=0.5, blue, mark=none] coordinates
                        {(568, -16) (574, -16) (574, 16) (568, 16)} --cycle;
                    \addplot+[fill, opacity=0.5, green, mark=none] coordinates
                        {(575, -16) (635, -16) (635, 16) (575, 16)} --cycle;
                    \addplot+[fill, opacity=0.5, red, mark=none] coordinates
                        {(636, -16) (640, -16) (640, 16) (636, 16)} --cycle;
                    \addplot+[fill, opacity=0.5, blue, mark=none] coordinates
                        {(773, -16) (824, -16) (824, 16) (773, 16)} --cycle;
                \end{axis}
            \end{tikzpicture}
        }
    \end{center}
    \caption{}
    \label{fig:experimentee_result2}
\end{figure}
