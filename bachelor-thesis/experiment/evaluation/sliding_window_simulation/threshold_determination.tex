\paragraph{Threshold Determination} \label{threshold_determination}
It is assumed that a large number of incoming time series windows should be unclassifiable. This can only be determined
by specifying a threshold as upper bound for the distance between a time series window and its nearest neighbour. An
intuitive approach for the threshold determinations would include knowledge about the distances between the instances of
a class. This approach can not be tested in this experiment, cause the protocol has the great weakness to contain only
one instance for every class in the training data as already mentioned in section \ref{protocol_review}. The following
threshold determinations have been tested.

\begin{itemize}
    \item \textbf{Half Minimum Distance:} This approach determinates the threshold for a class by the half minimum
    distance of the class to all other classes in the training data.
    \item \textbf{Half Average Distance:} This approach determinates the threshold for a class by the half average
    distance of the class to all other classes in the training data.
    \item \textbf{Half Middle Distance:} This approach determinates the threshold for a class by the half average of the
    minimum and maximum distance of the class to all other classes in the training data.
    \item \textbf{Cheating:} This approach fakes already generated knowledge about the distance threshold by setting the
    threshold to the by a small factor increased distance between the only existing training class instance and the test
    instance that has to be found. Two more intuitively chosen factors will be tested, $\frac{11}{10}$ and
    $\frac{12}{10}$. The cheating approach exists only to judge the performance of other threshold determinations.
\end{itemize}
