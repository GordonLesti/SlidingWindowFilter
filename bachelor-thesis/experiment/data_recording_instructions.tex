\subsection{Data Recording Instructions} \label{data_recording_instructions}
The recording instructions of the experiment are containing 16 slides plus one slide for the welcoming and two slides
for thanking the experimentee and saying goodbye. The slides are changing after every performed gesture. A gesture can
be performed by pressing the \textit{B} button, moving the controller and releasing the \textit{B} button after the
gesture. As mentioned in section \ref{experiment} does the experiment consists of two parts, the recording of 8 training
gestures and the repetition of those gestures mixed with physical activities. All slides of the experiment are shown in
figure~\ref{fig:slides}.

\begin{figure}
    \begin{center}
        \begin{tabular}{cccc}
            \frame{\includegraphics[width=0.24\textwidth]{1.png}} &
            \frame{\includegraphics[width=0.24\textwidth]{2.png}} &
            \frame{\includegraphics[width=0.24\textwidth]{3.png}} &
            \frame{\includegraphics[width=0.24\textwidth]{4.png}} \\
            (a) \vspace{0.5ex} & (b) \vspace{0.5ex} & (c) \vspace{0.5ex} & (d) \vspace{0.5ex} \\
            \frame{\includegraphics[width=0.24\textwidth]{5.png}} &
            \frame{\includegraphics[width=0.24\textwidth]{6.png}} &
            \frame{\includegraphics[width=0.24\textwidth]{7.png}} &
            \frame{\includegraphics[width=0.24\textwidth]{8.png}} \\
            (e) \vspace{0.5ex} & (f) \vspace{0.5ex} & (g) \vspace{0.5ex} & (h) \vspace{0.5ex} \\
            \frame{\includegraphics[width=0.24\textwidth]{9.png}} &
            \frame{\includegraphics[width=0.24\textwidth]{10.png}} &
            \frame{\includegraphics[width=0.24\textwidth]{11.png}} &
            \frame{\includegraphics[width=0.24\textwidth]{12.png}} \\
            (i) \vspace{0.5ex} & (j) \vspace{0.5ex} & (k) \vspace{0.5ex} & (l) \vspace{0.5ex} \\
            \frame{\includegraphics[width=0.24\textwidth]{13.png}} &
            \frame{\includegraphics[width=0.24\textwidth]{14.png}} &
            \frame{\includegraphics[width=0.24\textwidth]{15.png}} &
            \frame{\includegraphics[width=0.24\textwidth]{16.png}} \\
            (m) \vspace{0.5ex} & (n) \vspace{0.5ex} & (o) \vspace{0.5ex} & (p) \vspace{0.5ex} \\
            \frame{\includegraphics[width=0.24\textwidth]{17.png}} &
            \frame{\includegraphics[width=0.24\textwidth]{18.png}} &
            \frame{\includegraphics[width=0.24\textwidth]{19.png}} & \\
            (q) & (r) & (s) & \\
        \end{tabular}
    \end{center}
    \caption{The slides that are guiding the experimentees.}
    \label{fig:slides}
\end{figure}

Slide (a) of figure \ref{fig:slides} has the task to welcome the experimentee and is later used in experimental protocol
to mark the start of the recording. The slides (b) to (i) of figure \ref{fig:slides} have the task to create training
data for 1NN-DTW. Physical activities mixed with the same gestures from (b) to (i) are on the slides (j) to (q) of
figure \ref{fig:slides}. The recorded acceleration data from slide (j) to (q) of figure \ref{fig:slides} will simulate
the time series stream in section \ref{experimental_protocol}. Slide (r) of figure \ref{fig:slides} and the last
insignificant gesture have the task to prevent an abrupt ending of the record. The last slide (s) of figure
\ref{fig:slides} is just closing the recording applciation.

\subsubsection{Gesture Notation} \label{gesture_notation}
The \textit{clockwise circle} gesture of slide (b) and (j) in figure \ref{fig:slides} is abbreviated with $GesA$, the
\textit{flipped Z} gesture on slide (c) and (k) of figure \ref{fig:slides} is abbreviated with $GesB$ and so on until
the \textit{W} gesture of slide (i) and (q) in figure \ref{fig:slides} shortened with $GesH$.

As the experiment was executed by different experimentees, a instance of gesture $GesC$ performed by the fourth
experimentee on slide (d) of figure \ref{fig:slides} will be abbreviated as $exp_{4}.GesC_{1}$. The same gesture by the
same experimentee on slide (l) of figure \ref{fig:slides} will be abbreviated as $exp_{4}.GesC_{2}$.

\subsubsection{Instructions Review} \label{instructions_review}
The presented recording instructions in section \ref{data_recording_instructions} combined with the recording software
in section \ref{recording} has been created in front of this bachelor thesis and the experiment was performed while
writing this thesis. Unfortunately do the instructions have some weaknesses that occurred during the experiment or while
evaluating the produced data.

\paragraph{Training Data Quantity} The instructions are containing only 8 instances of training gestures for 8 different
classes. That made the determination of a threshold for a class very difficult. It would have been better to produce at
least two or better three instances of a gesture for every class.

\paragraph{Gesture Illustration Size} It was confusing for some experimentess that the gestures on the slides (j) to (q)
had an other size as the gestures on slide (b) to (i) of figure \ref{fig:slides}. The recording had to be repeated,
cause the experimentess performed the gesture scaled to the illustration size on the slide.

\paragraph{Similar Gestures} While designing the recording instructions and the containing gestures care was taken to
medium complex gestures that differ well from one another. This has worked well for the most part, but the evaluation
has shown that gesture \textit{GesD} and \textit{GesH} are very similar. It would have been better to flip gesture
\textit{GesH} horizontally.

Changing the recording instructions during the bachelor thesis due to the above mentioned weaknesses was no option.

