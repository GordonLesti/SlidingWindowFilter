\subsubsection{Instructions Review} \label{instructions_review}
The presented recording instructions in section \ref{data_recording_instructions} combined with the recording software
in section \ref{recording} has been created in front of this bachelor thesis and the experiment was performed while
writing this thesis. Unfortunately do the instructions have some weaknesses that occurred during the experiment or while
evaluating the produced data.

\paragraph{Training Data Quantity} The instructions are containing only 8 instances of training gestures for 8 different
classes. That made the determination of a threshold for a class very difficult. It would have been better to produce at
least two or better three instances of a gesture for every class.

\paragraph{Gesture Illustration Size} It was confusing for some experimentess that the gestures on the slides (j) to (q)
had an other size as the gestures on slide (b) to (i) of figure \ref{fig:slides}. The recording had to be repeated,
cause the experimentess performed the gesture scaled to the illustration size on the slide.

\paragraph{Similar Gestures} While designing the recording instructions and the containing gestures care was taken to
medium complex gestures that differ well from one another. This has worked well for the most part, but the evaluation
has shown that gesture \textit{GesD} and \textit{GesH} are very similar. It would have been better to flip gesture
\textit{GesH} horizontally.

Changing the recording instructions during the bachelor thesis due to the above mentioned weaknesses was no option.
