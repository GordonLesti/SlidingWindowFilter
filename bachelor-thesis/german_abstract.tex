\renewcommand{\abstractname}{Zusammenfassung.}
\renewcommand{\keywordname}{\textbf{Schlagw\"orter:}}
\begin{abstract} \addcontentsline{toc}{section}{Zusammenfassung}
    Viele unterschiedliche Ger\"ate im Leben des modernen Menschen produzieren untentwegt
    Daten in den verschiedenesten Formen. Offensichtliche Beispiele daf\"ur sind intelligente Mobiltelefone, tragbare
    Ger\"ate wie intelligente Uhren oder Titnessarmb\"ander. Ein betr\"achtlicher Anteil der Daten wird ununterbrochen
    produziert und kann als Zeitreihendatenstrom interpretiert werden. Zum Beispiel Beschleunigungs- oder
    Vitalit\"atsdaten. Eine Echtzeitanalyse dieses Zeitreihendatenstroms bedingt m\"oglicher Weise eine konstant
    ausgef\"uhrte Erster-N\"achster-Nachbar (1NN) Klassifizierung auf dem aktuellen Zeitreihenfenster mit einem
    Distanzma{\ss} wie Dynamic Time Warping (DTW). Die begrenzten Mittel von tragbaren Ger\"aten oder Mikrocontrollern
    zwingen zu einem sparsamen Umgang mit Speicher und Laufzeit.

    Diese Bachelorarbeit erkl\"art den Ansatz eines Filters f\"ur Zeitreihen mit linearer Komplexit\"at vor 1NN-DTW. Der
    Mehrwert eines Filters mit geringer Laufzeit ist dabei die Einsparung von kostenintensiven 1NN-DTW Ausf\"uhrungen im
    Falle von nicht klassifizierbaren Zeitreihenfenstern. Anhand von Ergebnissen eines beispielhaften Experiments mit
    Gestenerkennung durch Beschleunigungsdaten wird gezeigt, dass Filter mit nur geringem Genauigkeitsverlust die
    Ausf\"uhrung von 1NN-DTW reduzieren k\"onnen.
    \keywords{Zeitreihen, Distanzma{\ss}e, Klassifizierung}\\
\end{abstract}
