\renewcommand{\abstractname}{\normalsize Abstract.}
\begin{abstract} \addcontentsline{toc}{section}{Abstract}
    \normalsize
    Several devices with various sensors in the modern humans everyday life are producing a diverse kind of data
    constantly. Common examples are smartphones, wearable devices like smartwatches and fitness tracker bracelets. A
    considerable part of the data is produced continuously and can be interpreted as time series streams. Examples are
    acceleration or vitality data. A real time evaluation of the time series stream requires possibly a constantly
    running 1-Nearest-Neighbour (1NN) classification on the most recent time series window with a distance measure like
    Dynamic Time Warping (DTW). The limited resources of wearable devices or microcontrollers forces to use memory and
    running time economically.

    This bachelor thesis explains the approach of a filter with linear complexity for time series ahead of a 1NN-DTW.
    The added value of a filter with cost efficient running time is to reduce the execution of a cost intensive 1NN-DTW
    in the case of unclassifiable time series windows. On results of an experimental example with gesture detection via
    acceleration data will be shown that filters can reduce the execution of 1NN-DTW without loss of accuracy.
    \keywords{Time Series, Distance Measures, Classification}\\
\end{abstract}
