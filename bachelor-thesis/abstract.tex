\begin{abstract}
A lot of devices with various sensors in our everyday life are producing constantly a diverse kind of data. Obvious
examples are smartphones, wearable devices like smartwatches and fitness tracker braclets. A considerable part of the
produced data comes continuously and can be interpreted as time series streams. For example acceleration or vitality
data. A real time evaluation of the time series stream presupposes possibly a constantly running 1-Nearest-Neighbour
(1NN) classification on the most recent time series window with a distance measure like Dynamic Time Warping (DTW). The
limited resources of wearable devices or microcontrollers forces to a frugal usage of memory and running time.

This bachelor thesis explains the approach of a filter with linear complexity for time series in front of a 1NN
classification. The added value of a filter with cheap running time is to reduce the execution of a expensive
1NN classification in the case of unclassifiable time series windows. On results of an experimental
example with gesture detection via acceleration data will be shown that filters can reduce the execution of 1NN
classifications without great loss of accuracy.
\keywords{Time Series, Distance Measures, Classification}
\end{abstract}
